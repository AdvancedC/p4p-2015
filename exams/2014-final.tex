\documentclass[12pt]{article}
\usepackage{graphicx}
\usepackage{listings}
\usepackage{fullpage}
\usepackage{tikz}
\usepackage{url}
\usepackage{enumerate}
\usetikzlibrary{shapes,arrows,calc,automata}

\usepackage[listings]{tcolorbox}
\newtcbinputlisting{\codelisting}[2][]{
    listing file={#2},
    fonttitle=\bfseries,
    listing options={basicstyle=\ttfamily\footnotesize},
    listing only,
    #1
}

\begin{document}

\title{Programming for Performance (ECE459): Final}
\author{}
\renewcommand{\today}{}
\maketitle

 ~\\[-8em]

\begin{center}
{\Large April 22, 2014}
\end{center}

This open-book final has 12 non-cover pages and 6 questions, worth 25 points each. You may consult any printed
material (books, notes, etc).

\section{Nonblocking I/O}

This C code sends a file over a socket.

\codelisting{code/sendfile.c}

\paragraph{Part 1 (5 points).} Identify the lines in {\tt sendFile()}
which perform blocking I/O.

~\\

\noindent
Next, you'll convert {\tt sendFile()} to use nonblocking
I/O. Here is some state that you'll want to maintain.

\codelisting{code/sendfile_nbio.c}

Create two functions: a new {\tt sendFile()} function,
which opens the file and sets up the state in the {\tt sendfile\_state}
struct; and a {\tt handle\_io()} function, which may read a chunk
from the file and sends data over the socket. The {\tt handle\_io()} function
should return and 0 when work remains and a nonzero value
when the file has been completely sent over the network 
(The read may be blocking;
it is only the socket I/O that needs to be nonblocking.)
 
%Describe, in a few sentences, the structural transformation that you'd
%need to make to {\tt sendFile()} to make it use nonblocking I/O.
%(You'll probably want to use 2 functions. What do these functions do,
%and how do they get called?)

%\paragraph{Answer.}~\\[3em]

\paragraph{Part 2 (20 points).} Describe how the main function would call the functions I've described.
Provide implementations for {\tt sendFile()} and {\tt handle\_io()}.
You may assume the existence of 
function {\tt setNonblocking()}, which will set the mode of the socket to
non-blocking mode. You may omit error-handling code.

\paragraph{Answer.}
\newpage

\section{Dependencies and OpenMP}
Assume that {\tt A} and
{\tt C} are properly allocated arrays. After the completion of {\tt slow}, the
program reads the values of {\tt C}.
\codelisting{code/exam-openmp.cpp}

\paragraph{Part 1 (10 points).} For every {\tt for} loop in the {\tt work} function,
identify the loop-carried dependencies. (Write your answers on the code itself.)

\paragraph{Part 2 (15 points).} Assume you are only allowed to modify the {\tt work}
function. Use OpenMP and any reordering you wish to optimize the code. Argue
your function preserves the behaviour of the original.

\paragraph{Answer.}~
\newpage

\section{OpenMPI and MapReduce}
This OpenMPI code has a number of problems. One is sending too
much information.
\codelisting{code/logProcess-mpi.c}


\paragraph{Reducing Communications Overhead (10 points).}
Indicate changes to the code
which would reduce the communications overhead. For this part,
you do not need to say anything about returning the results to the
master thread, nor about combining the results.

\paragraph{MapReduce (15 points).} Instead of using OpenMPI,
we could reformulate this as a MapReduce problem. (For this part, imagine that the
code actually did collate the results from the worker threads and
produced a combined histogram.)  Describe a suitable mapper function
(explaining the format of its input, the algorithm, and the output)
and reducer function.

\paragraph{Answer.}~
\newpage

\section{GPU Computation: Password Cracking}
Cyclic Redundancy Checks (CRCs) are not an acceptable hash function
for passwords. However, for the purposes of this question, we'll
pretend that they are. You can think of a CRC as a checksum which is
resilient against transposition. 

\codelisting{code/crc.c}
{\scriptsize \verb+[Source: http://www.barrgroup.com/Embedded-Systems/How-To/CRC-Calculation-C-Code]+}

Assume that you have a file containing CRCs of passwords and that you
would like to recover the passwords using OpenCL. The
passwords are all 6 characters from \verb+[0-9A-Za-z]+, i.e. ASCII
range (48-57, 65-90, 97-122). Your approach is going to be: using the
GPU, compute all of the CRCs for 6-character passwords from the given
range and put passwords into a buffer indexed by CRC. In your kernel(s), you may assume the existence of {\tt decode\_char()} which converts from an int
in the range $(0, 62)$ to
a character in \verb+[0-9A-Za-z]+ and the existence of function {\tt atomic\_table\_write(buffer, index, c1, ..., c6)} which atomically writes the 6 characters of the
password to the appropriate index in {\tt buffer}.

\paragraph{Part 1 (6 marks).} Explain all the buffers that you would
declare in the host code and state whether each buffer should be
read-only, write-only, or read/write.

~\\[5em]

\paragraph{Part 2 (4 marks).} Describe the global range you'd pass
to your {\tt enqueueNDRangeKernel} call(s).

~\\[5em]

\paragraph{Part 3 (15 marks).} Write relevant
OpenCL kernel implementation(s). Include the important points; syntax
and exact types aren't the most important thing, problem structure is.
Argue that your implementation is thread-safe.

~\\[20em]


\section{Compiler Optimizations}

Consider the {\tt visit} function below.
\codelisting{code/foo.c}

\subsection*{Part 1 (5 marks)}
Assume that {\tt printf} has no cost and that {\tt root} is properly initialized to a list of sufficient length. Identify an obvious, yet unsafe, transformation that would halve the work done by {\tt visit}.

\paragraph{Transformation:} ~\\[2em]

\subsection*{Part 2 (10 marks)}
Now, assume that {\tt visit} belongs to a larger multi-threaded
program. Describe a potential action taken by another thread which
makes the transformation unsafe.

\paragraph{Why the transformation is unsafe:} ~\\[2em]

\subsection*{Part 3 (10 marks)}
(a) What if {\tt visit} belonged to a single-threaded program with
no interrupts---would your proposed transformation be safe?
Why or why not? (b) Would you expect a compiler to automatically
carry out this transformation? What would stop it?

\paragraph{Answers:} ~\\[5em]

\section{Race Conditions/Thread Safety}

Consider the following code:
\codelisting{code/swap.c}
For this question, assume the following global variables: {\tt int a = 1, b = 2, c = 3.}

\subsection*{Part 1 (3 marks)}

Let's say that we have two threads. One thread calls
\verb+swap(&b, &a)+ while the other thread calls \verb+swap(&b, &c)+.
Write down all possible expected outputs of the values of the
variables after the calls complete, assuming that {\tt swap} is
thread-safe.

\paragraph{Expected outputs:}

\newpage
\subsection*{Part 2 (6 marks)}

Assume each line of code is atomic. Write down an interleaving
of the code in two separate threads that shows that {\tt swap}
is not in fact thread-safe, along with the final output. That is,
your output should not match either expected result from Part 1.
You should only have code in one thread for each row of the table.
(Use a scrap piece of paper and write your final answer in the table.)

\begin{center}
\begin{tabular}{l|l|l|l|l}
  {\bf Thread 1: \verb+swap(&b, &a)+} & {\bf Thread 2: \verb+swap(&b, &c)+} & {\bf a} &{\bf b } & {\bf c} \\ \hline
  &&&& \\ \hline
  &&&& \\ \hline
  &&&& \\ \hline
  &&&& \\ \hline
  &&&& \\ \hline
  &&&& \\ \hline
\end{tabular}
\end{center}

\paragraph{Actual output:}

\subsection*{Part 3 (4 marks)}

Explain briefly (one sentence) what the {\tt restrict} keyword
does. Assume we label both parameters to {\tt swap} with
{\tt restrict}. Would this make the function thread-safe? Explain
why or why not. If it does make the function thread-safe, ignore the locking portion when completing Part 4.

\subsection*{Part 4 (12 marks)}
Assume that in your code you create a mutex, {\tt m},
and properly destroy it when the program ends. Write below where
you would lock and unlock the mutex to make the code thread-safe.
(Pseudocode like {\tt lock(m)} is OK.) Use the finest grain locking
possible with a single lock.

\codelisting{code/swap2.c}

Show what happens with the same interleaving (from Part 2) on
your lock-enabled code. Verify that you get one of the expected
outputs. Explain why the code is now thread-safe in all cases.


\begin{center}
\begin{tabular}{l|l|l|l|l}
  {\bf Thread 1: \verb+swap(&b, &a)+} & {\bf Thread 2: \verb+swap(&b, &c)+} & {\bf a} &{\bf b } & {\bf c} \\ \hline
  &&&& \\ \hline
  &&&& \\ \hline
  &&&& \\ \hline
  &&&& \\ \hline
  &&&& \\ \hline
  &&&& \\ \hline
  &&&& \\ \hline
  &&&& \\ \hline
  &&&& \\ \hline
  &&&& \\ \hline
  &&&& \\ \hline
  &&&& \\ \hline
  &&&& \\ \hline
  &&&& \\ \hline
\end{tabular}
\end{center}

\paragraph{Actual output:}

\paragraph{Explanation:}

\end{document}
