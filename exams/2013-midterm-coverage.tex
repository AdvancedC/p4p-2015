\documentclass[12pt]{article}

\usepackage[letterpaper, hmargin=0.75in, vmargin=0.75in]{geometry}
\usepackage{tikz}
\usepackage[colorlinks=true, urlcolor=black]{hyperref}

\pagestyle{empty}

\title{ECE 459: Programming for Performance\\Midterm Information}
\author{Patrick Lam}
\date{}

\begin{document}
\maketitle
\thispagestyle{empty}

First, the logistical information.

\begin{itemize}
\item {\bf Date:} Wednesday, February 27 
\item {\bf Time:} 19:00 to 20:20
\item {\bf Place:} RCH 301/309
\end{itemize}

\paragraph{Coverage.} I may ask questions about all of the material up to Lecture 12.
Lecture notes for Lectures 13 and 14 may also be helpful (I expand on
some of the material I've previously discussed), but all of the
answers should be in the notes for Lectures 1 through 12.

\paragraph{Format.} There are four questions on the exam. The first question is
short-answer, while the other three are multiple-part questions with
longer answers.  Some of the questions do involve writing code. Since
you don't have access to a compiler, I won't be picky about syntax,
but I do expect you to use the correct concepts.

\paragraph{Topics.} We saw the following concepts in class. Some will be on the exam.
\begin{itemize}
\item Bandwidth versus latency.
\item Calculating speedups and maximum speedups: Amdahl's Law and Gustafson's Law.
\item Concurrency vs Parallelism.
\item Threads: Pthreads, simple locks, synchronizing threads, race conditions.
\item Working with the compiler.
\item Dependencies.
\item Breaking dependencies: speculative execution, value speculation. 
\item Parallelization patterns.
\item Automatic parallelization.
\item OpenMP.
\item Memory models, reordering, fences and barriers, atomic instructions.
\end{itemize}

\end{document}
