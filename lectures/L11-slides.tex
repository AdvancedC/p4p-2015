\documentclass[aspectratio=43]{beamer}

% Text packages to stop warnings
\usepackage{lmodern}
\usepackage{textcomp}
\usepackage{listings}
\usepackage{multirow}

% Themes
\usetheme{Boadilla}
\setbeamertemplate{footline}[page number]{}
\setbeamertemplate{navigation symbols}{}

% Suppress the navigation bar
\beamertemplatenavigationsymbolsempty

\lstset{basicstyle=\scriptsize, frame=single}

\newenvironment{changemargin}[1]{% 
  \begin{list}{}{% 
    \setlength{\topsep}{0pt}% 
    \setlength{\leftmargin}{#1}% 
    \setlength{\rightmargin}{1em}
    \setlength{\listparindent}{\parindent}% 
    \setlength{\itemindent}{\parindent}% 
    \setlength{\parsep}{\parskip}% 
  }% 
  \item[]}{\end{list}} 

\title{Lecture 11---Dependencies}
\date{January 28, 2015}

\begin{document}

%%%%%%%%%%%%%%%%%%%%%%%%%%%%%%%%%%%%%%%%%%%%%%%%%%%%%%%%%%%%%%%%%%%%%%%%%%%%%%%%
\begin{frame}[plain]
  \titlepage
\end{frame}
%%%%%%%%%%%%%%%%%%%%%%%%%%%%%%%%%%%%%%%%%%%%%%%%%%%%%%%%%%%%%%%%%%%%%%%%%%%%%%%%

%%%%%%%%%%%%%%%%%%%%%%%%%%%%%%%%%%%%%%%%%%%%%%%%%%%%%%%%%%%%%%%%%%%%%%%%%%%%%%%%
\begin{frame}
  \frametitle{Roadmap}

  \Large
    \begin{changemargin}{2cm}
      Last Time: C++ atomics; C Compilers\\
      This Time: Dependencies
    \end{changemargin}
  
\end{frame}
%%%%%%%%%%%%%%%%%%%%%%%%%%%%%%%%%%%%%%%%%%%%%%%%%%%%%%%%%%%%%%%%%%%%%%%%%%%%%%%%

%%%%%%%%%%%%%%%%%%%%%%%%%%%%%%%%%%%%%%%%%%%%%%%%%%%%%%%%%%%%%%%%%%%%%%%%%%%%%%%%
\begin{frame}
  \frametitle{Atomics when not using C++11}

  \begin{changemargin}{2cm}
    Not really.\\[1em]

    {\tt gcc} supports atomics via extensions: \\
  \end{changemargin}
  \begin{changemargin}{1cm}
    {\scriptsize
      \url{https://gcc.gnu.org/onlinedocs/gcc/_005f_005fatomic-Builtins.html}}
  \end{changemargin}

  \begin{changemargin}{2cm}
    OS X has atomics via OS calls: \\
  \end{changemargin} {\scriptsize
  \begin{changemargin}{1cm}
    \url{https://developer.apple.com/library/mac/documentation/Cocoa/Conceptual/Multithreading/ThreadSafety/ThreadSafety.html}\\
  \end{changemargin}
  }
  \begin{changemargin}{2cm}
  etc\ldots \\[2em]
    Reference:\\
  \end{changemargin} {\scriptsize
  \begin{changemargin}{.5cm}
    \url{http://stackoverflow.com/questions/1130018/unix-portable-atomic-operations}
  \end{changemargin}
  }
\end{frame}
%%%%%%%%%%%%%%%%%%%%%%%%%%%%%%%%%%%%%%%%%%%%%%%%%%%%%%%%%%%%%%%%%%%%%%%%%%%%%%%%

\part{Dependencies}
\frame{\partpage}

%%%%%%%%%%%%%%%%%%%%%%%%%%%%%%%%%%%%%%%%%%%%%%%%%%%%%%%%%%%%%%%%%%%%%%%%%%%%%%%%
\begin{frame}
  \frametitle{Next topic: Dependencies}

  \Large
  \begin{changemargin}{2.5cm}
     \structure{Dependencies} are the main
      limitation to parallelization.\\[1em]
     Example: computation must be evaulated as {\tt XY} and not {\tt YX}.\\[1em]
  \end{changemargin}
\end{frame}

\begin{frame}
  \frametitle{Not synchronization}

  \Large
  \begin{changemargin}{2.5cm}
      Assume (for now) no synchronization problems.\\[1em]
      Only trying to identify code that is safe to run in
      parallel.

  \end{changemargin}
\end{frame}
%%%%%%%%%%%%%%%%%%%%%%%%%%%%%%%%%%%%%%%%%%%%%%%%%%%%%%%%%%%%%%%%%%%%%%%%%%%%%%%%

\part{Memory-carried Dependencies}
\frametitle{\partpage}

\begin{frame}
\frametitle{Dependencies: Analogies}
\begin{changemargin}{1cm}
Must extract bicycle from garage before closing garage door.\\[1em]

Must close washing machine door before starting the cycle.\\[1em]

Must be called on before answering questions? (sort of)\\[1em]

Students must submit assignment before course staff can mark the assignment.
\end{changemargin}
\end{frame}

%%%%%%%%%%%%%%%%%%%%%%%%%%%%%%%%%%%%%%%%%%%%%%%%%%%%%%%%%%%%%%%%%%%%%%%%%%%%%%%%
\begin{frame}[fragile]
\frametitle{Memory-carried Dependencies}
\begin{changemargin}{2.5cm}
\item Dependencies limit the amount of parallelization.
\vfill
Can we execute these 2 lines in parallel?
\begin{lstlisting}
x = 42
x = x + 1  
\end{lstlisting}
\pause
\alert{No.}\\[1em]
\begin{itemize}
\item Assume x initially 1. What are possible outcomes?
\pause \newline \structure{~~~x = 43 or x = 42}\\[1em]

\end{itemize}

Next, we'll classify dependencies.

\end{changemargin}

\end{frame}
%%%%%%%%%%%%%%%%%%%%%%%%%%%%%%%%%%%%%%%%%%%%%%%%%%%%%%%%%%%%%%%%%%%%%%%%%%%%%%%%

%%%%%%%%%%%%%%%%%%%%%%%%%%%%%%%%%%%%%%%%%%%%%%%%%%%%%%%%%%%%%%%%%%%%%%%%%%%%%%%%
\begin{frame}[fragile]
\frametitle{Read After Read (RAR)}
\begin{changemargin}{2.5cm}
Can we execute these 2 lines in parallel? (initially x is 2)
\begin{lstlisting}
y = x + 1
z = x + 5
\end{lstlisting}
\pause
\structure{Yes.}\\[1em]
\begin{itemize}
\item Variables y and z are independent.
\item Variable x is only read.
\end{itemize}

RAR dependency allows parallelization.

\end{changemargin}

\end{frame}
%%%%%%%%%%%%%%%%%%%%%%%%%%%%%%%%%%%%%%%%%%%%%%%%%%%%%%%%%%%%%%%%%%%%%%%%%%%%%%%%

%%%%%%%%%%%%%%%%%%%%%%%%%%%%%%%%%%%%%%%%%%%%%%%%%%%%%%%%%%%%%%%%%%%%%%%%%%%%%%%%
\begin{frame}[fragile]
\frametitle{Read After Write (RAW)}

\begin{changemargin}{2.5cm}
What about these 2 lines? (again, initially x is 2):
\begin{lstlisting}
x = 37
z = x + 5
\end{lstlisting}
\pause
\alert{No, z = 42 or z = 7.}\\[1em]

RAW inhibits parallelization: can't change ordering.\\
Also known as a \structure{true dependency}.
\end{changemargin}
\end{frame}
%%%%%%%%%%%%%%%%%%%%%%%%%%%%%%%%%%%%%%%%%%%%%%%%%%%%%%%%%%%%%%%%%%%%%%%%%%%%%%%%

%%%%%%%%%%%%%%%%%%%%%%%%%%%%%%%%%%%%%%%%%%%%%%%%%%%%%%%%%%%%%%%%%%%%%%%%%%%%%%%%
\begin{frame}[fragile]
\frametitle{Write After Read (WAR)}
\begin{changemargin}{2.5cm}
What if we change the order now? (again, initially x is 2)
\begin{lstlisting}
z = x + 5
x = 37
\end{lstlisting}
\pause
\alert{No. Again, z = 42 or z = 7.}\\[1em]
\begin{itemize}
\item WAR is also known as a \structure{anti-dependency}.
\item But, we can modify this code to enable parallelization.
\end{itemize}
\end{changemargin}
\end{frame}
%%%%%%%%%%%%%%%%%%%%%%%%%%%%%%%%%%%%%%%%%%%%%%%%%%%%%%%%%%%%%%%%%%%%%%%%%%%%%%%%

%%%%%%%%%%%%%%%%%%%%%%%%%%%%%%%%%%%%%%%%%%%%%%%%%%%%%%%%%%%%%%%%%%%%%%%%%%%%%%%%
\begin{frame}[fragile]
\frametitle{Removing Write After Read (WAR) Dependencies}
\begin{changemargin}{1.8cm}
Make a copy of the variable:
\begin{lstlisting}
x_copy = x
z = x_copy + 5
x = 37
\end{lstlisting}
\pause
\structure{We can now run the last 2 lines in parallel.}
\begin{itemize}
\item Induced a true dependency (RAW) between first 2 lines.
\item Isn't that bad?
\end{itemize}
\pause
Not always:
\begin{lstlisting}
z = very_long_function(x) + 5
x = very_long_calculation()
\end{lstlisting}
\end{changemargin}
\end{frame}
%%%%%%%%%%%%%%%%%%%%%%%%%%%%%%%%%%%%%%%%%%%%%%%%%%%%%%%%%%%%%%%%%%%%%%%%%%%%%%%%

%%%%%%%%%%%%%%%%%%%%%%%%%%%%%%%%%%%%%%%%%%%%%%%%%%%%%%%%%%%%%%%%%%%%%%%%%%%%%%%%
\begin{frame}[fragile]
\frametitle{Write After Write (WAW)}
\begin{changemargin}{2.5cm}
Can we run these lines in parallel? (initially x is 2)
\begin{lstlisting}
z = x + 5
z = x + 40
\end{lstlisting}
\pause
\alert{Nope, z = 42 or z = 7}.\\[1em]
\begin{itemize}
\item WAW is also known as an \structure{output dependency}.
\item We can remove this dependency (like WAR):
\end{itemize}
\pause
\begin{lstlisting}
z_copy = x + 5
z = x + 40
\end{lstlisting}
\end{changemargin}
\end{frame}
%%%%%%%%%%%%%%%%%%%%%%%%%%%%%%%%%%%%%%%%%%%%%%%%%%%%%%%%%%%%%%%%%%%%%%%%%%%%%%%%

%%%%%%%%%%%%%%%%%%%%%%%%%%%%%%%%%%%%%%%%%%%%%%%%%%%%%%%%%%%%%%%%%%%%%%%%%%%%%%%%
\begin{frame}
\frametitle{Summary of Memory-carried Dependencies}
\begin{center}
\begin{tabular}{ll|p{2.8cm}p{3.2cm}}
& & \multicolumn{2}{c}{Second Access} \\ 
&  & \bf Read & \bf Write \\ \hline
\multirow{2}{*}{First Access} & \bf Read & No Dependency Read After Read (RAR)  & Anti-dependency Write After Read (WAR) \\[0.5em]
& \bf Write & True Dependency Read After Write (RAW) & Output Dependency Write After Write (WAW) \\
\end{tabular}
\end{center}
\end{frame}
%%%%%%%%%%%%%%%%%%%%%%%%%%%%%%%%%%%%%%%%%%%%%%%%%%%%%%%%%%%%%%%%%%%%%%%%%%%%%%%%


\part{Loop-carried Dependencies}
\frame{\partpage}

%%%%%%%%%%%%%%%%%%%%%%%%%%%%%%%%%%%%%%%%%%%%%%%%%%%%%%%%%%%%%%%%%%%%%%%%%%%%%%%%
\begin{frame}[fragile]
\frametitle{Loop-carried Dependencies (1)}
\begin{changemargin}{2.5cm}
Can we run these lines in parallel? \\ ~~~(initially a[0] and a[1] are 1)
\begin{lstlisting}
a[4] = a[0] + 1
a[5] = a[1] + 2
\end{lstlisting}
\pause
\structure{Yes.}\\[1em]
\begin{itemize}
\item There are no dependencies between these lines.
\item However, this is not how we normally use arrays\ldots
\end{itemize}
\end{changemargin}
\end{frame}
%%%%%%%%%%%%%%%%%%%%%%%%%%%%%%%%%%%%%%%%%%%%%%%%%%%%%%%%%%%%%%%%%%%%%%%%%%%%%%%%

%%%%%%%%%%%%%%%%%%%%%%%%%%%%%%%%%%%%%%%%%%%%%%%%%%%%%%%%%%%%%%%%%%%%%%%%%%%%%%%%
\begin{frame}[fragile]
\frametitle{Loop-carried Dependencies (2)}
\begin{changemargin}{2.5cm}
What about this? (all elements initially 1)
\begin{lstlisting}
for (int i = 1; i < 12; ++i)
    a[i] = a[i-1] + 1
\end{lstlisting}
\pause
\alert{No, a[2] = 3 or a[2] = 2.}\\[1em]
\begin{itemize}
\item Statements depend on previous loop iterations.
\item An example of a \structure{loop-carried dependency}.
\end{itemize}
\end{changemargin}
\end{frame}
%%%%%%%%%%%%%%%%%%%%%%%%%%%%%%%%%%%%%%%%%%%%%%%%%%%%%%%%%%%%%%%%%%%%%%%%%%%%%%%%

%%%%%%%%%%%%%%%%%%%%%%%%%%%%%%%%%%%%%%%%%%%%%%%%%%%%%%%%%%%%%%%%%%%%%%%%%%%%%%%%
\begin{frame}[fragile]
\frametitle{Loop-carried Dependencies (3)}
\begin{changemargin}{2.5cm}
Can we parallelize this? (again, all elements initially 1)
\begin{lstlisting}
for (int i = 4; i < 12; ++i)
    a[i] = a[i-4] + 1
\end{lstlisting}
\pause
\structure{Yes, to a degree.}\\[1em]
\begin{itemize}
\item We can execute 4 statements in parallel:
\begin{itemize}
  \item a[4] = a[0] + 1, a[8] = a[4] + 1
  \item a[5] = a[1] + 1, a[9] = a[5] + 1
  \item a[6] = a[2] + 1, a[10] = a[6] + 1
  \item a[7] = a[3] + 1, a[11] = a[7] + 1
\end{itemize}  
\end{itemize}

\pause
\structure{Always consider dependencies between iterations.}
\end{changemargin}
\end{frame}
%%%%%%%%%%%%%%%%%%%%%%%%%%%%%%%%%%%%%%%%%%%%%%%%%%%%%%%%%%%%%%%%%%%%%%%%%%%%%%%%

%%%%%%%%%%%%%%%%%%%%%%%%%%%%%%%%%%%%%%%%%%%%%%%%%%%%%%%%%%%%%%%%%%%%%%%%%%%%%%%%
\begin{frame}[fragile]
\frametitle{Larger example: Loop-carried Dependencies}
{\small \begin{verbatim}
  // Repeatedly square input, return number of iterations before 
  // absolute value exceeds 4, or 1000, whichever is smaller.
  int inMandelbrot(double x0, double y0) {
    int iterations = 0;
    double x = x0, y = y0, x2 = x*x, y2 = y*y;
    while ((x2+y2 < 4) && (iterations < 1000)) {
      y = 2*x*y + y0;
      x = x2 - y2 + x0;
      x2 = x*x; y2 = y*y;
      iterations++;
    }
    return iterations;
  }
\end{verbatim} 
}
How can we parallelize this? \\
\pause
\begin{itemize}
\item Run {\tt inMandelbrot} sequentially for each point, but parallelize
different point computations.
\end{itemize}
\end{frame}
%%%%%%%%%%%%%%%%%%%%%%%%%%%%%%%%%%%%%%%%%%%%%%%%%%%%%%%%%%%%%%%%%%%%%%%%%%%%%%%%
\end{document}
