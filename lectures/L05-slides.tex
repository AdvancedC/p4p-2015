\documentclass[aspectratio=43]{beamer}

% Text packages to stop warnings
\usepackage{lmodern}
\usepackage{textcomp}
\usepackage{tikz}
\usepackage{listings}

% Themes
\usetheme{Boadilla}
\setbeamertemplate{footline}[page number]{}
\setbeamertemplate{navigation symbols}{}

% Suppress the navigation bar
\beamertemplatenavigationsymbolsempty

\lstset{basicstyle=\scriptsize,frame=single}

\newenvironment{changemargin}[1]{% 
  \begin{list}{}{% 
    \setlength{\topsep}{0pt}% 
    \setlength{\leftmargin}{#1}% 
    \setlength{\rightmargin}{1em}
    \setlength{\listparindent}{\parindent}% 
    \setlength{\itemindent}{\parindent}% 
    \setlength{\parsep}{\parskip}% 
  }% 
  \item[]}{\end{list}} 

\title{Lecture 05---Processes vs Threads II; Thread Usage}
\subtitle{ECE 459: Programming for Performance}
\author{Patrick Lam}
\institute{University of Waterloo}
\date{January 12, 2015}

\begin{document}

%%%%%%%%%%%%%%%%%%%%%%%%%%%%%%%%%%%%%%%%%%%%%%%%%%%%%%%%%%%%%%%%%%%%%%%%%%%%%%%%
\begin{frame}[plain]
  \titlepage
\end{frame}
%%%%%%%%%%%%%%%%%%%%%%%%%%%%%%%%%%%%%%%%%%%%%%%%%%%%%%%%%%%%%%%%%%%%%%%%%%%%%%%%

%%%%%%%%%%%%%%%%%%%%%%%%%%%%%%%%%%%%%%%%%%%%%%%%%%%%%%%%%%%%%%%%%%%%%%%%%%%%%%%%
\begin{frame}[fragile]
  \frametitle{Processes: {\tt fork} Usage Example (OS refresher)}

  \begin{changemargin}{3cm}
    \begin{minipage}{.5\textwidth}
\begin{lstlisting}
pid = fork();
if (pid < 0) {
        fork_error_function();
} else if (pid == 0) {
        child_function();
} else {
        parent_function();
}
\end{lstlisting}
    \end{minipage}
  \end{changemargin}

  \begin{changemargin}{1cm}
     {\tt fork} produces a second copy of the calling process, which starts
          execution after the call.\\[1em]

    The only difference between the copies is the return value: the parent gets the pid of
          the child, while the child gets 0.
  \end{changemargin}
\end{frame}
%%%%%%%%%%%%%%%%%%%%%%%%%%%%%%%%%%%%%%%%%%%%%%%%%%%%%%%%%%%%%%%%%%%%%%%%%%%%%%%%

%%%%%%%%%%%%%%%%%%%%%%%%%%%%%%%%%%%%%%%%%%%%%%%%%%%%%%%%%%%%%%%%%%%%%%%%%%%%%%%%
\begin{frame}[fragile]
  \frametitle{Live coding: overheads of threads vs fork}
\end{frame}
%%%%%%%%%%%%%%%%%%%%%%%%%%%%%%%%%%%%%%%%%%%%%%%%%%%%%%%%%%%%%%%%%%%%%%%%%%%%%%%%

  %%%%%%%%%%%%%%%%%%%%%%%%%%%%%%%%%%%%%%%%%%%%%%%%%%%%%%%%%%%%%%%%%%%%%%%%%%%%%%%%
\begin{frame}[fragile]
  \frametitle{Results: Threads Offer a Speedup of 6.5 over {\tt fork}}

  \begin{changemargin}{1cm}
  Here's a benchmark between {\tt fork} and Pthreads on a laptop, creating and
  destroying 50,000 threads:
  \vfill
  \begin{lstlisting}[basicstyle=\scriptsize]
jon@riker examples master % time ./create_fork 
0.18s user 4.14s system 34% cpu 12.484 total
jon@riker examples master % time ./create_pthread 
0.73s user 1.29s system 107% cpu 1.887 total
  \end{lstlisting}
  \vfill
  Clearly Pthreads incur much lower overhead than {\tt fork}.
  \end{changemargin}
\end{frame}
%%%%%%%%%%%%%%%%%%%%%%%%%%%%%%%%%%%%%%%%%%%%%%%%%%%%%%%%%%%%%%%%%%%%%%%%%%%%%%%%

\section{}
%%%%%%%%%%%%%%%%%%%%%%%%%%%%%%%%%%%%%%%%%%%%%%%%%%%%%%%%%%%%%%%%%%%%%%%%%%%%%%%%
\begin{frame}
  \frametitle{Assumptions}

  \begin{changemargin}{0.5cm}
  First, we'll see how to use threads on ``embarrassingly parallel problems''.
  \begin{itemize}
    \item mostly-independent sub-problems (little synchronization); and
    \item strong locality (little communication).
  \end{itemize}
  \vfill
  Later, we'll see:
  \begin{itemize}
    \item which problems can be parallelized (\structure{dependencies})
    \item alternative parallelization patterns\\(right now, just use one thread
          per sub-problem)
  \end{itemize}
  \end{changemargin}
\end{frame}
%%%%%%%%%%%%%%%%%%%%%%%%%%%%%%%%%%%%%%%%%%%%%%%%%%%%%%%%%%%%%%%%%%%%%%%%%%%%%%%%

\begin{frame}
  \frametitle{POSIX Threads}

  \begin{changemargin}{1.5cm}
  \begin{itemize}
    \item Available on most systems
    \vfill
    \item Windows has Pthreads Win32, but I wouldn't use it; \\use Linux for
          this course
    \vfill
    \item API available by {\tt \#include <pthread.h>}
    \vfill
    \item Compile with pthread flag \\ ({\tt gcc -pthread prog.c -o prog})
  \end{itemize}
  \end{changemargin}
\end{frame}

%%%%%%%%%%%%%%%%%%%%%%%%%%%%%%%%%%%%%%%%%%%%%%%%%%%%%%%%%%%%%%%%%%%%%%%%%%%%%%%%

\begin{frame}[fragile]
  \frametitle{C++ 11 Threads}

  \begin{changemargin}{1.5cm}
  \begin{itemize}
    \item Now part of the C++ standard (library)
    \vfill
    \item API available with {\tt \#include <thread>}
    \vfill
    \item Compile with flags: \\ (\verb!g++ -std=c++11 -pthread prog.c -o prog!)
  \end{itemize}
  \end{changemargin}
\end{frame}

\section{Management}
%%%%%%%%%%%%%%%%%%%%%%%%%%%%%%%%%%%%%%%%%%%%%%%%%%%%%%%%%%%%%%%%%%%%%%%%%%%%%%%%
\begin{frame}[fragile]
  \frametitle{Pthreads: Creating Threads}

  \begin{changemargin}{1.5cm}
  \begin{lstlisting}
int pthread_create(pthread_t* thread, 
                   const pthread_attr_t* attr,
                   void* (*start_routine)(void*),
                   void* arg);
  \end{lstlisting}
  \vfill
  {\bf thread}: creates a handle to a thread at pointer location

  {\bf attr}: thread attributes (NULL for defaults, more details later)

  {\bf start\_routine}: function to start execution

  {\bf arg}: value to pass to start\_routine
  \vfill
  returns 0 on success, error number otherwise\\(contents of *thread are
  undefined)
  \end{changemargin}
\end{frame}
%%%%%%%%%%%%%%%%%%%%%%%%%%%%%%%%%%%%%%%%%%%%%%%%%%%%%%%%%%%%%%%%%%%%%%%%%%%%%%%%

%%%%%%%%%%%%%%%%%%%%%%%%%%%%%%%%%%%%%%%%%%%%%%%%%%%%%%%%%%%%%%%%%%%%%%%%%%%%%%%%
\begin{frame}[fragile]
  \frametitle{Creating Threads---Pthreads Example}

  \begin{changemargin}{1.5cm}
\begin{lstlisting}
#include <pthread.h>
#include <stdio.h>

void* run(void*) {
  printf("In run\n");
}

int main() {
  pthread_t thread;
  pthread_create(&thread, NULL, &run, NULL);
  printf("In main\n");
}
\end{lstlisting}
  \vfill
  Simply creates a thread and terminates\\(usage isn't really right, as we'll
  see.)
  \end{changemargin}
\end{frame}
%%%%%%%%%%%%%%%%%%%%%%%%%%%%%%%%%%%%%%%%%%%%%%%%%%%%%%%%%%%%%%%%%%%%%%%%%%%%%%%%

%%%%%%%%%%%%%%%%%%%%%%%%%%%%%%%%%%%%%%%%%%%%%%%%%%%%%%%%%%%%%%%%%%%%%%%%%%%%%%%%
\begin{frame}[fragile]
  \frametitle{Creating Threads---C++11 Example}

  \begin{changemargin}{1.5cm}
\begin{lstlisting}
#include <thread>
#include <iostream>

void run() {
  std::cout << "In run\n";
}

int main() {
  std::thread t1(run);
  std::cout << "In main\n";
  t1.join(); // hang in there...
}
\end{lstlisting}
  \end{changemargin}
\end{frame}
%%%%%%%%%%%%%%%%%%%%%%%%%%%%%%%%%%%%%%%%%%%%%%%%%%%%%%%%%%%%%%%%%%%%%%%%%%%%%%%%


%%%%%%%%%%%%%%%%%%%%%%%%%%%%%%%%%%%%%%%%%%%%%%%%%%%%%%%%%%%%%%%%%%%%%%%%%%%%%%%%
\begin{frame}[fragile]
  \frametitle{Waiting for Threads}

  \begin{changemargin}{1cm}
  \begin{lstlisting}
int pthread_join(pthread_t thread,
                 void** retval)
  \end{lstlisting}
  \vfill
  {\bf thread}: wait for this thread to terminate (thread must be~joinable).

  {\bf retval}: stores exit status of thread (set by {\tt pthread\_exit}) to
                 the location pointed by *retval. If cancelled, returns
                 {\tt PTHREAD\_CANCELED}. {\tt NULL} is ignored.
  \vfill
  returns 0 on success, error number otherwise.
  \vfill
  {\bf Only call this one time per thread!} Multiple calls on the same thread
  leads to undefined behaviour.
  \end{changemargin}
\end{frame}
%%%%%%%%%%%%%%%%%%%%%%%%%%%%%%%%%%%%%%%%%%%%%%%%%%%%%%%%%%%%%%%%%%%%%%%%%%%%%%%%

%%%%%%%%%%%%%%%%%%%%%%%%%%%%%%%%%%%%%%%%%%%%%%%%%%%%%%%%%%%%%%%%%%%%%%%%%%%%%%%%
\begin{frame}[fragile]
  \frametitle{Waiting for Threads---Pthreads example}

\begin{changemargin}{1.5cm}
\begin{lstlisting}
#include <pthread.h>
#include <stdio.h>

void* run(void*) {
  printf("In run\n");
}

int main() {
  pthread_t thread;
  pthread_create(&thread, NULL, &run, NULL);
  printf("In main\n");
  pthread_join(thread, NULL);
}
\end{lstlisting}
  \vfill
  This now waits for the newly created thread to terminate.
\end{changemargin}
\end{frame}
%%%%%%%%%%%%%%%%%%%%%%%%%%%%%%%%%%%%%%%%%%%%%%%%%%%%%%%%%%%%%%%%%%%%%%%%%%%%%%%%

%%%%%%%%%%%%%%%%%%%%%%%%%%%%%%%%%%%%%%%%%%%%%%%%%%%%%%%%%%%%%%%%%%%%%%%%%%%%%%%%
\begin{frame}[fragile]
  \frametitle{Creating Threads---C++11 Example}

  \begin{changemargin}{1.5cm}
\begin{lstlisting}
#include <thread>
#include <iostream>

void run() {
  std::cout << "In run\n";
}

int main() {
  std::thread t1(run);
  std::cout << "In main\n";
  t1.join(); // aha!
}
\end{lstlisting}
  \end{changemargin}
\end{frame}
%%%%%%%%%%%%%%%%%%%%%%%%%%%%%%%%%%%%%%%%%%%%%%%%%%%%%%%%%%%%%%%%%%%%%%%%%%%%%%%%

%%%%%%%%%%%%%%%%%%%%%%%%%%%%%%%%%%%%%%%%%%%%%%%%%%%%%%%%%%%%%%%%%%%%%%%%%%%%%%%%
\begin{frame}[fragile]
  \frametitle{Passing Data to Pthreads threads\ldots Wrongly}

\begin{changemargin}{1.5cm}
Consider this snippet:
\vfill
\begin{lstlisting}
int i;
for (i = 0; i < 10; ++i)
  pthread_create(&thread[i], NULL, &run, (void*)&i);
\end{lstlisting}
  \vfill
  This is a \alert{terrible} idea. Why?
  \vfill
  \begin{enumerate}
    \item<2-> The value of {\tt i} will probably change before the thread executes
    \item<2-> The memory for {\tt i} may be out of scope, and therefore invalid by
          the time the thread executes
  \end{enumerate}
\end{changemargin}

\end{frame}
%%%%%%%%%%%%%%%%%%%%%%%%%%%%%%%%%%%%%%%%%%%%%%%%%%%%%%%%%%%%%%%%%%%%%%%%%%%%%%%%

%%%%%%%%%%%%%%%%%%%%%%%%%%%%%%%%%%%%%%%%%%%%%%%%%%%%%%%%%%%%%%%%%%%%%%%%%%%%%%%%
\begin{frame}[fragile]
  \frametitle{Passing Data to Pthreads threads}

\begin{changemargin}{1.5cm}
What about:
\begin{lstlisting}
int i;
for (i = 0; i < 10; ++i)
  pthread_create(&thread[i], NULL, &run, (void*)i);

...

void* run(void* arg) {
  int id = (int)arg;
\end{lstlisting}
  \vfill
  This is suggested in the book, but should carry a warning:

  \vfill
  \begin{itemize}
    \item<2-> Beware size mismatches between arguments: 
      no guarantee that a pointer is the same size as an int, so your data
      may overflow.
    \item<2-> Sizes of data types change between systems. For maximum
      portability, just use pointers you got {\tt from malloc}.
  \end{itemize}
\end{changemargin}

\end{frame}
%%%%%%%%%%%%%%%%%%%%%%%%%%%%%%%%%%%%%%%%%%%%%%%%%%%%%%%%%%%%%%%%%%%%%%%%%%%%%%%%

%%%%%%%%%%%%%%%%%%%%%%%%%%%%%%%%%%%%%%%%%%%%%%%%%%%%%%%%%%%%%%%%%%%%%%%%%%%%%%%%
\begin{frame}[fragile]
  \frametitle{Passing Data to C++11 threads}
\begin{changemargin}{1.5cm}
It's easier to get data to threads in C++11:
\begin{lstlisting}
#include <thread>
#include <iostream>

void run(int i) {
  std::cout << "In run " << i << "\n";
}

int main() {
  for (int i = 0; i < 10; ++i) {
    std::thread t1(run, i);
    t1.detach(); // see the next slide...
  }
}
\end{lstlisting}
\end{changemargin}
  
\end{frame}
%%%%%%%%%%%%%%%%%%%%%%%%%%%%%%%%%%%%%%%%%%%%%%%%%%%%%%%%%%%%%%%%%%%%%%%%%%%%%%%%

%%%%%%%%%%%%%%%%%%%%%%%%%%%%%%%%%%%%%%%%%%%%%%%%%%%%%%%%%%%%%%%%%%%%%%%%%%%%%%%%
\begin{frame}[fragile]
  \frametitle{Getting Data from C++11 threads}
  \begin{changemargin}{1.5cm}
    \ldots but it's harder to get data back.\\
    Use {\tt async} and {\tt future} abstractions:
    \begin{lstlisting}
#include <thread>
#include <iostream>
#include <future>

int run() {
  return 42;
}

int main() {
  std::future<int> t1_retval =
                  std::async(std::launch::async, run);
  std::cout << t1_retval.get();
}
\end{lstlisting}
\end{changemargin}
  
\end{frame}
%%%%%%%%%%%%%%%%%%%%%%%%%%%%%%%%%%%%%%%%%%%%%%%%%%%%%%%%%%%%%%%%%%%%%%%%%%%%%%%%

%%%%%%%%%%%%%%%%%%%%%%%%%%%%%%%%%%%%%%%%%%%%%%%%%%%%%%%%%%%%%%%%%%%%%%%%%%%%%%%%
\begin{frame}[fragile]
  \frametitle{Detached Threads}

\begin{changemargin}{1.5cm}
  {\it Joinable} threads (the default) wait for someone to call
  {\tt pthread\_join} before they release their resources.
  \vfill
  {\it Detached} threads release their resources when they terminate, without
  being joined.
  \vfill
  \begin{lstlisting}
int pthread_detach(pthread_t thread);
  \end{lstlisting}
  \vfill
  {\bf thread}: marks the thread as detached
  \vfill
  returns 0 on success, error number otherwise.
  \vfill
  Calling {\tt pthread\_detach} on an already detached thread results in undefined
  behaviour.
\end{changemargin}

\end{frame}
%%%%%%%%%%%%%%%%%%%%%%%%%%%%%%%%%%%%%%%%%%%%%%%%%%%%%%%%%%%%%%%%%%%%%%%%%%%%%%%%

%%%%%%%%%%%%%%%%%%%%%%%%%%%%%%%%%%%%%%%%%%%%%%%%%%%%%%%%%%%%%%%%%%%%%%%%%%%%%%%%
\begin{frame}[fragile]
  \frametitle{Thread Termination}

\begin{changemargin}{1.5cm}
  \begin{lstlisting}
void pthread_exit(void *retval);
  \end{lstlisting}
  \vfill
  {\bf retval}: return value passed to function that calls {\tt pthread\_join}
  \vfill
  start\_routine returning is equivalent to calling {\tt pthread\_exit} with
  that return value;
  \vfill
  {\tt pthread\_exit} is called implicitly when the {\tt start\_routine} of a
  thread returns.
  \vfill
  There is no C++11 equivalent.
\end{changemargin}

\end{frame}
%%%%%%%%%%%%%%%%%%%%%%%%%%%%%%%%%%%%%%%%%%%%%%%%%%%%%%%%%%%%%%%%%%%%%%%%%%%%%%%%

%%%%%%%%%%%%%%%%%%%%%%%%%%%%%%%%%%%%%%%%%%%%%%%%%%%%%%%%%%%%%%%%%%%%%%%%%%%%%%%%
\begin{frame}
  \frametitle{Attributes}

\begin{changemargin}{1.5cm}
  By default, threads are {\it joinable} on Linux, but a more portable way to
  know what you're getting is to set thread attributes. You can change:
  \begin{itemize}
    \item Detached or joinable state
    \item Scheduling inheritance
    \item Scheduling policy
    \item Scheduling parameters
    \item Scheduling contention scope
    \item Stack size
    \item Stack address
    \item Stack guard (overflow) size
  \end{itemize}
\end{changemargin}

\end{frame}
%%%%%%%%%%%%%%%%%%%%%%%%%%%%%%%%%%%%%%%%%%%%%%%%%%%%%%%%%%%%%%%%%%%%%%%%%%%%%%%%

%%%%%%%%%%%%%%%%%%%%%%%%%%%%%%%%%%%%%%%%%%%%%%%%%%%%%%%%%%%%%%%%%%%%%%%%%%%%%%%%
\begin{frame}[fragile]
  \frametitle{Attributes---Example}

\begin{changemargin}{1.5cm}
  \begin{lstlisting}
size_t stacksize;
pthread_attr_t attributes;
pthread_attr_init(&attributes);
pthread_attr_getstacksize(&attributes, &stacksize);
printf("Stack size = %i\n", stacksize);
pthread_attr_destroy(&attributes);
  \end{lstlisting}
Running this on a laptop produces:
  \begin{lstlisting}
jon@riker examples master % ./stack_size 
Stack size = 8388608
  \end{lstlisting}
  Setting a thread state to joinable:
  \begin{lstlisting}
pthread_attr_setdetachstate(&attributes,
                            PTHREAD_CREATE_JOINABLE);
  \end{lstlisting}
\end{changemargin}

\end{frame}
%%%%%%%%%%%%%%%%%%%%%%%%%%%%%%%%%%%%%%%%%%%%%%%%%%%%%%%%%%%%%%%%%%%%%%%%%%%%%%%%

%%%%%%%%%%%%%%%%%%%%%%%%%%%%%%%%%%%%%%%%%%%%%%%%%%%%%%%%%%%%%%%%%%%%%%%%%%%%%%%%
\begin{frame}[fragile]
  \frametitle{Detached Threads: Warning!}

\begin{changemargin}{1.5cm}
\begin{lstlisting}
#include <pthread.h>
#include <stdio.h>

void* run(void*) {
  printf("In run\n");
}

int main() {
  pthread_t thread;
  pthread_create(&thread, NULL, &run, NULL);
  pthread_detach(thread);
  printf("In main\n");
}
\end{lstlisting}

  When I run it, it just prints ``In main'', why?
\end{changemargin}

\end{frame}
%%%%%%%%%%%%%%%%%%%%%%%%%%%%%%%%%%%%%%%%%%%%%%%%%%%%%%%%%%%%%%%%%%%%%%%%%%%%%%%%

%%%%%%%%%%%%%%%%%%%%%%%%%%%%%%%%%%%%%%%%%%%%%%%%%%%%%%%%%%%%%%%%%%%%%%%%%%%%%%%%
\begin{frame}[fragile]
  \frametitle{Detached Threads: Solution to Problem}

\begin{changemargin}{1.5cm}
  \begin{lstlisting}
#include <pthread.h>
#include <stdio.h>

void* run(void*) {
  printf("In run\n");
}

int main() {
  pthread_t thread;
  pthread_create(&thread, NULL, &run, NULL);
  pthread_detach(thread);
  printf("In main\n");
  pthread_exit(NULL); // This waits for all detached
                      // threads to terminate
}
  \end{lstlisting}

  Make the final call {\tt pthread\_exit} if you have any detached threads. (There is no C++11 equivalent.)
\end{changemargin}
\end{frame}
%%%%%%%%%%%%%%%%%%%%%%%%%%%%%%%%%%%%%%%%%%%%%%%%%%%%%%%%%%%%%%%%%%%%%%%%%%%%%%%%

%%%%%%%%%%%%%%%%%%%%%%%%%%%%%%%%%%%%%%%%%%%%%%%%%%%%%%%%%%%%%%%%%%%%%%%%%%%%%%%%
\begin{frame}
  \frametitle{Threading Challenges}

\begin{changemargin}{1.5cm}
  \begin{itemize}
    \item Be aware of scheduling (you can also set affinity with pthreads on
      Linux).
    \vfill
    \item Make sure the libraries you use are {\bf thread-safe}:
      \begin{itemize}
        \item Means that the library protects its shared data.
      \end{itemize}
    \vfill
    \item glibc reentrant functions are also safe: a program can have more than one
      thread calling these functions concurrently.
    \vfill
    \item {\bf Example:} {\tt rand\_r} versus
      {\tt rand}.
  \end{itemize}
\end{changemargin}

\end{frame}
%%%%%%%%%%%%%%%%%%%%%%%%%%%%%%%%%%%%%%%%%%%%%%%%%%%%%%%%%%%%%%%%%%%%%%%%%%%%%%%%


\end{document}
