\documentclass[aspectratio=43]{beamer}

% Text packages to stop warnings
\usepackage{lmodern}
\usepackage{textcomp}
\usepackage{listings}
\usepackage{multirow}
\usepackage{tikz}

\usetikzlibrary{arrows,automata,shapes,positioning}
\tikzstyle{block} = [rectangle, draw, fill=blue!20, 
    text width=2.5em, text centered, rounded corners, minimum height=2em]
\tikzstyle{bw} = [rectangle, draw, fill=blue!20, 
    text width=3.5em, text centered, rounded corners, minimum height=2em]

% Themes
\usetheme{Boadilla}
\setbeamertemplate{footline}[page number]{}
\setbeamertemplate{navigation symbols}{}

% Suppress the navigation bar
\beamertemplatenavigationsymbolsempty

\lstset{basicstyle=\scriptsize, frame=single}

\newenvironment{changemargin}[1]{% 
  \begin{list}{}{% 
    \setlength{\topsep}{0pt}% 
    \setlength{\leftmargin}{#1}% 
    \setlength{\rightmargin}{1em}
    \setlength{\listparindent}{\parindent}% 
    \setlength{\itemindent}{\parindent}% 
    \setlength{\parsep}{\parskip}% 
  }% 
  \item[]}{\end{list}} 

\title{Lecture 12---Loop-carried Dependencies; Speculation}
\subtitle{ECE 459: Programming for Performance}
\date{January 30, 2015}

\begin{document}

%%%%%%%%%%%%%%%%%%%%%%%%%%%%%%%%%%%%%%%%%%%%%%%%%%%%%%%%%%%%%%%%%%%%%%%%%%%%%%%%
\begin{frame}[plain]
  \titlepage
\end{frame}
%%%%%%%%%%%%%%%%%%%%%%%%%%%%%%%%%%%%%%%%%%%%%%%%%%%%%%%%%%%%%%%%%%%%%%%%%%%%%%%%

%%%%%%%%%%%%%%%%%%%%%%%%%%%%%%%%%%%%%%%%%%%%%%%%%%%%%%%%%%%%%%%%%%%%%%%%%%%%%%%%
\begin{frame}
  \frametitle{Last Time}

  \begin{changemargin}{1.5cm}
    Dependencies:
  \end{changemargin}
\begin{center}
\begin{tabular}{ll|p{2.8cm}p{3.2cm}}
& & \multicolumn{2}{c}{Second Access} \\ 
&  & \bf Read & \bf Write \\ \hline
\multirow{2}{*}{First Access} & \bf Read & No Dependency Read After Read (RAR)  & Anti-dependency Write After Read (WAR) \\[0.5em]
& \bf Write & True Dependency Read After Write (RAW) & Output Dependency Write After Write (WAW) \\
\end{tabular}
\end{center}
  \begin{changemargin}{1.5cm}
     ~\\
     We also saw how to break WAR and WAW dependencies.
  \end{changemargin}
\end{frame}
%%%%%%%%%%%%%%%%%%%%%%%%%%%%%%%%%%%%%%%%%%%%%%%%%%%%%%%%%%%%%%%%%%%%%%%%%%%%%%%%

\part{Loop-carried Dependencies}
\frame{\partpage}

%%%%%%%%%%%%%%%%%%%%%%%%%%%%%%%%%%%%%%%%%%%%%%%%%%%%%%%%%%%%%%%%%%%%%%%%%%%%%%%%
\begin{frame}[fragile]
\frametitle{Loop-carried Dependencies (1)}
\begin{changemargin}{2.5cm}
Can we run these lines in parallel? \\ ~~~(initially a[0] and a[1] are 1)
\begin{lstlisting}
a[4] = a[0] + 1
a[5] = a[1] + 2
\end{lstlisting}
\pause
\structure{Yes.}\\[1em]
\begin{itemize}
\item There are no dependencies between these lines.
\item However, this is not how we normally use arrays\ldots
\end{itemize}
\end{changemargin}
\end{frame}
%%%%%%%%%%%%%%%%%%%%%%%%%%%%%%%%%%%%%%%%%%%%%%%%%%%%%%%%%%%%%%%%%%%%%%%%%%%%%%%%

%%%%%%%%%%%%%%%%%%%%%%%%%%%%%%%%%%%%%%%%%%%%%%%%%%%%%%%%%%%%%%%%%%%%%%%%%%%%%%%%
\begin{frame}[fragile]
\frametitle{Loop-carried Dependencies (2)}
\begin{changemargin}{2.5cm}
What about this? (all elements initially 1)
\begin{lstlisting}
for (int i = 1; i < 12; ++i)
    a[i] = a[i-1] + 1
\end{lstlisting}
\pause
\alert{No, a[2] = 3 or a[2] = 2.}\\[1em]
\begin{itemize}
\item Statements depend on previous loop iterations.
\item An example of a \structure{loop-carried dependency}.
\end{itemize}
\end{changemargin}
\end{frame}
%%%%%%%%%%%%%%%%%%%%%%%%%%%%%%%%%%%%%%%%%%%%%%%%%%%%%%%%%%%%%%%%%%%%%%%%%%%%%%%%

%%%%%%%%%%%%%%%%%%%%%%%%%%%%%%%%%%%%%%%%%%%%%%%%%%%%%%%%%%%%%%%%%%%%%%%%%%%%%%%%
\begin{frame}[fragile]
\frametitle{Loop-carried Dependencies (3)}
\begin{changemargin}{2.5cm}
Can we parallelize this? (again, all elements initially 1)
\begin{lstlisting}
for (int i = 4; i < 12; ++i)
    a[i] = a[i-4] + 1
\end{lstlisting}
\pause
\structure{Yes, to a degree.}\\[1em]
\begin{itemize}
\item We can execute 4 statements in parallel:
\begin{itemize}
  \item a[4] = a[0] + 1, a[8] = a[4] + 1
  \item a[5] = a[1] + 1, a[9] = a[5] + 1
  \item a[6] = a[2] + 1, a[10] = a[6] + 1
  \item a[7] = a[3] + 1, a[11] = a[7] + 1
\end{itemize}  
\end{itemize}

\pause
\structure{Always consider dependencies between iterations.}
\end{changemargin}
\end{frame}
%%%%%%%%%%%%%%%%%%%%%%%%%%%%%%%%%%%%%%%%%%%%%%%%%%%%%%%%%%%%%%%%%%%%%%%%%%%%%%%%

%%%%%%%%%%%%%%%%%%%%%%%%%%%%%%%%%%%%%%%%%%%%%%%%%%%%%%%%%%%%%%%%%%%%%%%%%%%%%%%%
\begin{frame}[fragile]
\frametitle{Larger example: Loop-carried Dependencies}
{\small \begin{verbatim}
  // Repeatedly square input, return number of iterations before 
  // absolute value exceeds 4, or 1000, whichever is smaller.
  int inMandelbrot(double x0, double y0) {
    int iterations = 0;
    double x = x0, y = y0, x2 = x*x, y2 = y*y;
    while ((x2+y2 < 4) && (iterations < 1000)) {
      y = 2*x*y + y0;
      x = x2 - y2 + x0;
      x2 = x*x; y2 = y*y;
      iterations++;
    }
    return iterations;
  }
\end{verbatim} 
}
How can we parallelize this? \\
\pause
\begin{itemize}
\item Run {\tt inMandelbrot} sequentially for each point, but parallelize
different point computations.
\end{itemize}
\end{frame}
%%%%%%%%%%%%%%%%%%%%%%%%%%%%%%%%%%%%%%%%%%%%%%%%%%%%%%%%%%%%%%%%%%%%%%%%%%%%%%%%

\begin{frame}[fragile]
  \frametitle{Live Coding Demo: Parallelizing Mandelbrot}

  \begin{changemargin}{1.5cm}
    Refactor the code; create array for output.\\[1em]
    Add a struct to pass offset, stride to thread.\\[1em]
    Create \& join threads.
  \end{changemargin}

\end{frame}

%%%%%%%%%%%%%%%%%%%%%%%%%%%%%%%%%%%%%%%%%%%%%%%%%%%%%%%%%%%%%%%%%%%%%%%%%%%%%%%%

\part{Breaking Dependencies with Speculation}
\frame{\partpage}

%%%%%%%%%%%%%%%%%%%%%%%%%%%%%%%%%%%%%%%%%%%%%%%%%%%%%%%%%%%%%%%%%%%%%%%%%%%%%%%%
\begin{frame}
  \frametitle{Breaking Dependencies}

  \begin{changemargin}{2cm}
  \structure{Speculation}: architects use it to predict
      branch targets.
  \begin{itemize}
    \item Need not wait for the branch to be evaluated.
  \end{itemize}~\\[1em]
  We'll use speculation at a coarser-grained level:
      speculatively parallelize source code.\\[1em]

      Two ways: \structure{speculative execution} and
      \structure{value speculation}.
  \end{changemargin}
\end{frame}
%%%%%%%%%%%%%%%%%%%%%%%%%%%%%%%%%%%%%%%%%%%%%%%%%%%%%%%%%%%%%%%%%%%%%%%%%%%%%%%%


\section{Speculation}
%%%%%%%%%%%%%%%%%%%%%%%%%%%%%%%%%%%%%%%%%%%%%%%%%%%%%%%%%%%%%%%%%%%%%%%%%%%%%%%%
\begin{frame}[fragile]
  \frametitle{Speculative Execution: Example}

  \begin{changemargin}{2cm}
Consider the following code:
  
  \begin{lstlisting}
void doWork(int x, int y) {
    int value = longCalculation(x, y);
    if (value > threshold) {
      return value + secondLongCalculation(x, y);
    }
    else {
      return value;
    }
}
  \end{lstlisting}

  Will we need to run {\tt secondLongCalculation}?
  \vfill  
  \begin{itemize}
    \item<2> OK, so: could we execute {\tt longCalculation} and {\tt secondLongCalculation}
      in parallel if we didn't have the conditional?
  \end{itemize}
  \end{changemargin}
\end{frame}
%%%%%%%%%%%%%%%%%%%%%%%%%%%%%%%%%%%%%%%%%%%%%%%%%%%%%%%%%%%%%%%%%%%%%%%%%%%%%%%%

%%%%%%%%%%%%%%%%%%%%%%%%%%%%%%%%%%%%%%%%%%%%%%%%%%%%%%%%%%%%%%%%%%%%%%%%%%%%%%%%
\begin{frame}[fragile]
  \frametitle{Speculative Execution: Assume No Conditional}

  \begin{changemargin}{1.5cm}
  Yes, we could parallelize them. Consider this code:
    
    \begin{lstlisting}
  void doWork(int x, int y) {
    thread_t t1, t2;
    point p(x,y);
    int v1, v2;
    thread_create(&t1, NULL, &longCalculation, &p);
    thread_create(&t2, NULL, &secondLongCalculation, &p);
    thread_join(t1, &v1);
    thread_join(t2, &v2);
    if (v1 > threshold) {
      return v1 + v2;
    } else {
      return v1;
    }
  }
    \end{lstlisting}

  We do both the calculations in parallel and return the same result as before.

  \begin{itemize}
    \item What are we assuming about {\tt longCalculation} and
{\tt secondLongCalculation}?
  \end{itemize}
  \end{changemargin}
\end{frame}
%%%%%%%%%%%%%%%%%%%%%%%%%%%%%%%%%%%%%%%%%%%%%%%%%%%%%%%%%%%%%%%%%%%%%%%%%%%%%%%%

%%%%%%%%%%%%%%%%%%%%%%%%%%%%%%%%%%%%%%%%%%%%%%%%%%%%%%%%%%%%%%%%%%%%%%%%%%%%%%%%
\begin{frame}
  \frametitle{Estimating Impact of Speculative Execution}

  \begin{changemargin}{2.5cm}
  $T_1$: time to run {\tt longCalculatuion}.

  $T_2$: time to run {\tt secondLongCalculation}.

  $p$: probability that {\tt secondLongCalculation} executes.\\[1em]

  In the normal case we have:
    \[T_{\mbox{\scriptsize normal}} = T_1 + pT_2.\]

  $S$: synchronization overhead.\\
  Our speculative code takes:
    \[ T_{\mbox{\scriptsize speculative}} = \mbox{max}(T_1, T_2) + S.\]

    \structure{Exercise.} When is speculative code faster? Slower? \\ How could you improve it?

  \end{changemargin}
\end{frame}
%%%%%%%%%%%%%%%%%%%%%%%%%%%%%%%%%%%%%%%%%%%%%%%%%%%%%%%%%%%%%%%%%%%%%%%%%%%%%%%%

%%%%%%%%%%%%%%%%%%%%%%%%%%%%%%%%%%%%%%%%%%%%%%%%%%%%%%%%%%%%%%%%%%%%%%%%%%%%%%%%
\begin{frame}[fragile]
  \frametitle{Shortcomings of Speculative Execution}

  \begin{changemargin}{2cm}
  Consider the following code:
  
  \begin{lstlisting}
void doWork(int x, int y) {
    int value = longCalculation(x, y);
    return secondLongCalculation(value);
}
  \end{lstlisting}

  Now we have a true dependency; can't use speculative~execution.\\[1em]

  But: if the value is predictable, we can execute
      {\tt secondLongCalculation} using the predicted value.\\[1em]

  This is \structure{value speculation}.
  \end{changemargin}
\end{frame}
%%%%%%%%%%%%%%%%%%%%%%%%%%%%%%%%%%%%%%%%%%%%%%%%%%%%%%%%%%%%%%%%%%%%%%%%%%%%%%%%

%%%%%%%%%%%%%%%%%%%%%%%%%%%%%%%%%%%%%%%%%%%%%%%%%%%%%%%%%%%%%%%%%%%%%%%%%%%%%%%%
\begin{frame}[fragile]
  \frametitle{Value Speculation Implementation}

% note: v1 is meant to be the result you get from longCalculation this time, while last_value is what you got last time. (The code doesn't show that). If you get the same result from longCalculation this time as you did last time, then secondLongCalculation is correct and you don't need to redo it.
  \begin{changemargin}{2cm}
  This Pthread code does value speculation:
  
  \begin{lstlisting}
void doWork(int x, int y) {
    thread_t t1, t2;
    point p(x,y);
    int v1, v2, last_value;
    thread_create(&t1, NULL, &longCalculation, &p);
    thread_create(&t2, NULL, &secondLongCalculation,
                  &last_value);
    thread_join(t1, &v1);
    thread_join(t2, &v2);
    if (v1 == last_value) {
      return v2;
    } else {
      last_value = v1;
      return secondLongCalculation(v1);
    }
}
  \end{lstlisting}

  Note: this is like memoization (plus parallelization).
  \end{changemargin}
\end{frame}
%%%%%%%%%%%%%%%%%%%%%%%%%%%%%%%%%%%%%%%%%%%%%%%%%%%%%%%%%%%%%%%%%%%%%%%%%%%%%%%%

%%%%%%%%%%%%%%%%%%%%%%%%%%%%%%%%%%%%%%%%%%%%%%%%%%%%%%%%%%%%%%%%%%%%%%%%%%%%%%%%
\begin{frame}
  \frametitle{Estimating Impact of Value Speculation}

  \begin{changemargin}{2.5cm}
  $T_1$: time to run {\tt longCalculatuion}.

  $T_2$: time to run {\tt secondLongCalculation}.

  $p$: probability that {\tt secondLongCalculation} executes.

  $S$: synchronization overhead.\\[1em]

  In the normal case, we again have:
    \[ T = T_1 +pT_2.\]

  This speculative code takes:
    \[ T = \mbox{max}(T_1, T_2) + S + pT_2.\]

    \structure{Exercise.} Again, when is speculative code faster? Slower? How could you improve it?

  \end{changemargin}
\end{frame}
%%%%%%%%%%%%%%%%%%%%%%%%%%%%%%%%%%%%%%%%%%%%%%%%%%%%%%%%%%%%%%%%%%%%%%%%%%%%%%%%

%%%%%%%%%%%%%%%%%%%%%%%%%%%%%%%%%%%%%%%%%%%%%%%%%%%%%%%%%%%%%%%%%%%%%%%%%%%%%%%%
\begin{frame}
  \frametitle{When Can We Speculate?}

  \begin{changemargin}{2.5cm}
  Required conditions for safety:

  \begin{itemize}
    \item {\tt longCalculation} and {\tt secondLongCalculation} must not call
      each other.
    \item {\tt secondLongCalculation} must not depend on
      any values set or modified by {\tt longCalculation}.
    \item The return value of {\tt longCalculation} must be deterministic.
  \end{itemize}

  General warning: Consider \structure{side effects} of function calls.
  \end{changemargin}
\end{frame}
%%%%%%%%%%%%%%%%%%%%%%%%%%%%%%%%%%%%%%%%%%%%%%%%%%%%%%%%%%%%%%%%%%%%%%%%%%%%%%%%

\end{document}
