\documentclass[aspectratio=169,10pt,compress]{beamer}

\usetheme{m}

\usepackage{booktabs}

%% \usepackage{minted}
%% \usemintedstyle{trac}

% Text packages to stop warnings


%% \usepackage{lmodern}
%% \usepackage{textcomp}
%% \usepackage[utf8]{inputenc}
%% \usepackage[T1]{fontenc}
%% \usepackage{ulem}
%% \usepackage{listings}

% Themes
%% \useoutertheme[subsection=false, footline=institutetitle]{miniframes}
%% \usecolortheme{uwaterloo}
%% \useinnertheme{circles}

%% % Suppress the navigation bar
%% \beamertemplatenavigationsymbolsempty

%% \lstset{basicstyle=\scriptsize, frame=single}

\title{Assignment 3 Background}
\subtitle{ECE 459: Programming for Performance}
\author{Jon Eyolfson}
\institute{University of Waterloo}
\date{March 6, 2015}

\begin{document}

\maketitle

\plain{Travelling Salesman Problem}

%%%%%%%%%%%%%%%%%%%%%%%%%%%%%%%%%%%%%%%%%%%%%%%%%%%%%%%%%%%%%%%%%%%%%%%%%%%%%%%%
\begin{frame}
  \frametitle{Problem Statement}

  {\bf Given:} a list of cities, with distances between each pair of cities

  \vspace{2em}

  {\bf Goal:} find the shortest route that visits each city exactly once and
  returns to the first city
\end{frame}
%%%%%%%%%%%%%%%%%%%%%%%%%%%%%%%%%%%%%%%%%%%%%%%%%%%%%%%%%%%%%%%%%%%%%%%%%%%%%%%%

\section{Genetic Algorithm}

%%%%%%%%%%%%%%%%%%%%%%%%%%%%%%%%%%%%%%%%%%%%%%%%%%%%%%%%%%%%%%%%%%%%%%%%%%%%%%%%
\begin{frame}
  \frametitle{High-level Overview}

  Generalizes heuristics from genetics

  \vspace{2em}

  Describes how to apply different random operations to improve our answer

  \vspace{2em}

  We only need to define a small number of operations that describe the problem
\end{frame}
%%%%%%%%%%%%%%%%%%%%%%%%%%%%%%%%%%%%%%%%%%%%%%%%%%%%%%%%%%%%%%%%%%%%%%%%%%%%%%%%

%%%%%%%%%%%%%%%%%%%%%%%%%%%%%%%%%%%%%%%%%%%%%%%%%%%%%%%%%%%%%%%%%%%%%%%%%%%%%%%%
\begin{frame}
  \frametitle{Applications}

  \begin{itemize}
    \item Course scheduling
    \item Many other NP-hard problems
    \item Playing games
  \end{itemize}
\end{frame}
%%%%%%%%%%%%%%%%%%%%%%%%%%%%%%%%%%%%%%%%%%%%%%%%%%%%%%%%%%%%%%%%%%%%%%%%%%%%%%%%

%%%%%%%%%%%%%%%%%%%%%%%%%%%%%%%%%%%%%%%%%%%%%%%%%%%%%%%%%%%%%%%%%%%%%%%%%%%%%%%%
\begin{frame}
  \frametitle{Terms}

  \begin{description}
    \item[Individual] A tour of cities, e.g. 1, 4, 3, 2
    \item[Population] A collection of individuals
    \item[Distance] The total units of distance to complete a tour
  \end{description}
\end{frame}
%%%%%%%%%%%%%%%%%%%%%%%%%%%%%%%%%%%%%%%%%%%%%%%%%%%%%%%%%%%%%%%%%%%%%%%%%%%%%%%%

%%%%%%%%%%%%%%%%%%%%%%%%%%%%%%%%%%%%%%%%%%%%%%%%%%%%%%%%%%%%%%%%%%%%%%%%%%%%%%%%
\begin{frame}
  \frametitle{Pseudocode}

  \begin{center}
    \fbox{
      \begin{minipage}{0.9\linewidth}
        \small

        create an initial population $pop(0) = X_1, ..., X_N$

        $t \leftarrow$ 0

        repeat

        $\quad$ assign each $X_i$ in $pop(t)$ a probability
        f($X_i$)/($\sum$ f($X_i$))

        $\quad$ for i $\leftarrow$ 1 to N do

        $\quad \quad$ a $\leftarrow$ random selection of individual from
        $pop(t)$

        $\quad \quad$ b $\leftarrow$ random selection of individual from
        $pop(t)$

        $\quad \quad$ child $\leftarrow$ reproduce(a, b)

        $\quad \quad$ with small probability mutate child
        
        $\quad \quad$ add child to $pop(t+1)$

        $\quad t \leftarrow t + 1$

        until stopping criteria

        return most fit individual
      \end{minipage}
    }
  \end{center}
\end{frame}
%%%%%%%%%%%%%%%%%%%%%%%%%%%%%%%%%%%%%%%%%%%%%%%%%%%%%%%%%%%%%%%%%%%%%%%%%%%%%%%%

%%%%%%%%%%%%%%%%%%%%%%%%%%%%%%%%%%%%%%%%%%%%%%%%%%%%%%%%%%%%%%%%%%%%%%%%%%%%%%%%
\begin{frame}
  \frametitle{Our Job}

  That's it, the operations we have to define are:
  \begin{itemize}
    \item Creating an initial population
    \item Creating a fitness function ({\bf higher} is better)
    \item Creating a selection function (not problem specific)
    \item Creating a crossover function (or reproduce)
    \item Creating a mutation function
  \end{itemize}

  \vfill

  There are also two parameters, which we will keep constant:
  \begin{description}
    \item[Population size] 100
    \item[Mutation probability] 1\%
  \end{description}
\end{frame}
%%%%%%%%%%%%%%%%%%%%%%%%%%%%%%%%%%%%%%%%%%%%%%%%%%%%%%%%%%%%%%%%%%%%%%%%%%%%%%%%

%%%%%%%%%%%%%%%%%%%%%%%%%%%%%%%%%%%%%%%%%%%%%%%%%%%%%%%%%%%%%%%%%%%%%%%%%%%%%%%%
\begin{frame}
  \frametitle{Initial Population}

  {\bf We represent our tours as a sequence of the city indexes}

  \begin{itemize}
    \item Start with a base tour which is all the cities in order, e.g. 1, 2, 3, 4, 5
    \item Randomly shuffle the tours around for each individual
  \end{itemize}

  \begin{center}
    \begin{tabular}{ll}
      \toprule
      ID & Tour\\
      \midrule
      P0 & 1, 5, 2, 4, 3\\
      P1 & 1, 4, 5, 2, 3\\
      P2 & 1, 3, 2, 5, 4\\
      P3 & 1, 2, 4, 5, 3\\
      \bottomrule
    \end{tabular}
  \end{center}

  {\bf For consistency our tour begins and ends at the first city}
\end{frame}
%%%%%%%%%%%%%%%%%%%%%%%%%%%%%%%%%%%%%%%%%%%%%%%%%%%%%%%%%%%%%%%%%%%%%%%%%%%%%%%%

%%%%%%%%%%%%%%%%%%%%%%%%%%%%%%%%%%%%%%%%%%%%%%%%%%%%%%%%%%%%%%%%%%%%%%%%%%%%%%%%
\begin{frame}
  \frametitle{Fitness Function}

  An evaluation of an individual in the population

  \vspace{2em}

  Most obvious metric is to just use the distance of the tour

  In this case, lower is better and we need higher is better

  \vspace{2em}

  \begin{enumerate}
    \item Find the maximum distance in the population
    \item Subtract a individual's distance from that value to obtain the fitness
  \end{enumerate}

  \vspace{2em}

  {\bf Therefore the individual with the highest distance has a fitness of 0}
\end{frame}
%%%%%%%%%%%%%%%%%%%%%%%%%%%%%%%%%%%%%%%%%%%%%%%%%%%%%%%%%%%%%%%%%%%%%%%%%%%%%%%%

%%%%%%%%%%%%%%%%%%%%%%%%%%%%%%%%%%%%%%%%%%%%%%%%%%%%%%%%%%%%%%%%%%%%%%%%%%%%%%%%
\begin{frame}
  \frametitle{Selection Function}

  Intuition is to have a better chance of picking more fit individuals

  \vspace{2em}
  
  Find the fitness of each individual, and normalize the values
  \begin{itemize}
    \item The sum of all the normalized fitness values should equal 1
  \end{itemize}

  \vspace{2em}
  
  Sort, accumulate, and pick random values, R, between 0 and 1

  \vspace{2em}

  Select the individual whose accumulated normalized value is greater than R
\end{frame}
%%%%%%%%%%%%%%%%%%%%%%%%%%%%%%%%%%%%%%%%%%%%%%%%%%%%%%%%%%%%%%%%%%%%%%%%%%%%%%%%

%%%%%%%%%%%%%%%%%%%%%%%%%%%%%%%%%%%%%%%%%%%%%%%%%%%%%%%%%%%%%%%%%%%%%%%%%%%%%%%%
\begin{frame}
  \frametitle{Selection Function Example}

  \begin{center}
    \begin{tabular}{llrr}
      \toprule
      ID & Tour & Distance & Fitness\\
      \midrule
      P0 & 1, 5, 2, 4, 3 & 100 &  0\\
      P1 & 1, 4, 5, 2, 3 &  80 & 20\\
      P2 & 1, 3, 2, 5, 4 &  50 & 50\\
      P3 & 1, 2, 4, 5, 3 &  70 & 30\\
      \bottomrule
    \end{tabular}
  \end{center}

  \begin{enumerate}
    \item Normalize the fitness values (P0 = 0, P1 = 0.2, P2 = 0.5, P3 = 0.3)
    \item Sort population by descending values (P2 = 0.5, P3 = 0.3, P1 = 0.2,
      P0 = 0)
    \item Accumulate the values (P2 = 0.5, P3 = 0.8, P1 = 1, P0 = 1)
    \item Pick a random value, R, and select individual (if R = 0.4, pick P2)
  \end{enumerate}

  Repeat step 4 for as many selections as you need
\end{frame}
%%%%%%%%%%%%%%%%%%%%%%%%%%%%%%%%%%%%%%%%%%%%%%%%%%%%%%%%%%%%%%%%%%%%%%%%%%%%%%%%

%%%%%%%%%%%%%%%%%%%%%%%%%%%%%%%%%%%%%%%%%%%%%%%%%%%%%%%%%%%%%%%%%%%%%%%%%%%%%%%%
\begin{frame}
  \frametitle{Crossover Function}

  This is how we combine two individuals to create another individual

  \begin{itemize}
    \item We will use a simple ordered 
      \begin{itemize}
        \item Select a random subtour from the first parent and copy it into the
          child (in order, all the cities at the same spot in the tour)
        \item Copy the remaining cities, not already in the child, in the order
          they appear in the second parent
      \end{itemize}
  \end{itemize}

\end{frame}
%%%%%%%%%%%%%%%%%%%%%%%%%%%%%%%%%%%%%%%%%%%%%%%%%%%%%%%%%%%%%%%%%%%%%%%%%%%%%%%%

%%%%%%%%%%%%%%%%%%%%%%%%%%%%%%%%%%%%%%%%%%%%%%%%%%%%%%%%%%%%%%%%%%%%%%%%%%%%%%%%
\begin{frame}
  \frametitle{Crossover Function Example}

  \begin{center}
    \begin{tabular}{ll}
      \toprule
      ID & Tour\\
      \midrule
      P2 & 1, 3, 2, 5, 4\\
      P3 & 1, 2, 4, 5, 3\\
      \bottomrule
    \end{tabular}
  \end{center}

  Suppose we chose P2 as the first parent and P3 as the second

  \begin{enumerate}
    \item Create a new child (C = 1, ?, ?, ?, ?)
    \item Select a subtour from first parent to copy to child

      Suppose we chose indexes 2 to 3 (C = 1, ?, 2, 5, ?)
    \item Fill in the missing values from the second parent in order
      (C = 1, 4, 2, 5, 3)
  \end{enumerate}
\end{frame}
%%%%%%%%%%%%%%%%%%%%%%%%%%%%%%%%%%%%%%%%%%%%%%%%%%%%%%%%%%%%%%%%%%%%%%%%%%%%%%%%

%%%%%%%%%%%%%%%%%%%%%%%%%%%%%%%%%%%%%%%%%%%%%%%%%%%%%%%%%%%%%%%%%%%%%%%%%%%%%%%%
\begin{frame}
  \frametitle{Mutation Function}

  How we may change an individual based off no other factors

  \vspace{2em}

  In our case we randomly change around a tour
\end{frame}
%%%%%%%%%%%%%%%%%%%%%%%%%%%%%%%%%%%%%%%%%%%%%%%%%%%%%%%%%%%%%%%%%%%%%%%%%%%%%%%%

%%%%%%%%%%%%%%%%%%%%%%%%%%%%%%%%%%%%%%%%%%%%%%%%%%%%%%%%%%%%%%%%%%%%%%%%%%%%%%%%
\begin{frame}
  \frametitle{Mutation Function Example}

  \begin{center}
    \begin{tabular}{ll}
      \toprule
      ID & Tour\\
      \midrule
      C & 1, 4, 2, 5, 3\\
      \bottomrule
    \end{tabular}
  \end{center}

  \begin{enumerate}
    \item Select a subtour and reverse the order

      Suppose we chose indexes 1 to 3 (C = 1, 5, 2, 4, 3)
  \end{enumerate}
\end{frame}
%%%%%%%%%%%%%%%%%%%%%%%%%%%%%%%%%%%%%%%%%%%%%%%%%%%%%%%%%%%%%%%%%%%%%%%%%%%%%%%%

%%%%%%%%%%%%%%%%%%%%%%%%%%%%%%%%%%%%%%%%%%%%%%%%%%%%%%%%%%%%%%%%%%%%%%%%%%%%%%%%
\begin{frame}
  \frametitle{Pseudocode Again}

  \begin{center}
    \fbox{
      \begin{minipage}{0.9\linewidth}
        \small

        create an initial population $pop(0) = X_1, ..., X_N$

        $t \leftarrow$ 0

        repeat

        $\quad$ assign each $X_i$ in $pop(t)$ a probability
        f($X_i$)/($\sum$ f($X_i$))

        $\quad$ for i $\leftarrow$ 1 to N do

        $\quad \quad$ a $\leftarrow$ random selection of individual from
        $pop(t)$

        $\quad \quad$ b $\leftarrow$ random selection of individual from
        $pop(t)$

        $\quad \quad$ child $\leftarrow$ reproduce(a, b)

        $\quad \quad$ with small probability mutate child
        
        $\quad \quad$ add child to $pop(t+1)$

        $\quad t \leftarrow t + 1$

        until stopping criteria

        return most fit individual
      \end{minipage}
    }
  \end{center}
\end{frame}
%%%%%%%%%%%%%%%%%%%%%%%%%%%%%%%%%%%%%%%%%%%%%%%%%%%%%%%%%%%%%%%%%%%%%%%%%%%%%%%%

%%%%%%%%%%%%%%%%%%%%%%%%%%%%%%%%%%%%%%%%%%%%%%%%%%%%%%%%%%%%%%%%%%%%%%%%%%%%%%%%
\begin{frame}
  \frametitle{Summary}

  That's all the operations we need for our genetic algorithm

  {\bf YOU MAY NOT CHANGE THEIR HIGH-LEVEL OPERATION}

  \vspace{2em}
  
  The interface to your algorithm is the following:
  \begin{itemize}
    \item A constructor call (will have the initial population)
    \item An iteration call, repeated (selection, crossover, mutate)
    \item A call to get the best individual found
  \end{itemize}

  {\bf YOU MAY NOT CHANGE THIS INTERFACE}

  \vspace{2em}

  Can't parallelize calls to iteration function, parallelize the iteration
  function itself
\end{frame}
%%%%%%%%%%%%%%%%%%%%%%%%%%%%%%%%%%%%%%%%%%%%%%%%%%%%%%%%%%%%%%%%%%%%%%%%%%%%%%%%

\section{Implementation}
%%%%%%%%%%%%%%%%%%%%%%%%%%%%%%%%%%%%%%%%%%%%%%%%%%%%%%%%%%%%%%%%%%%%%%%%%%%%%%%%
\begin{frame}
  \frametitle{Introduction}

  The code is more or less a direct translation of the high level functions

  \vspace{2em}

  Written in C++11, so we have complete control with nice abstractions
  \begin{itemize}
    \item The language should not be the main hurdle, ask if anything is
      confusing
  \end{itemize}

  \vspace{2em}

  Only used standard library functions, link to documentation is in the handout

  \vspace{2em}

  Used {\ttfamily typedef}s so you should be able to change data structures if
  you want
  \begin{itemize}
    \item {\ttfamily string} indexes, etc.
  \end{itemize}
\end{frame}
%%%%%%%%%%%%%%%%%%%%%%%%%%%%%%%%%%%%%%%%%%%%%%%%%%%%%%%%%%%%%%%%%%%%%%%%%%%%%%%%

%%%%%%%%%%%%%%%%%%%%%%%%%%%%%%%%%%%%%%%%%%%%%%%%%%%%%%%%%%%%%%%%%%%%%%%%%%%%%%%%
\begin{frame}
  \frametitle{Built-in Data Structures}

  \begin{description}
    \item[\ttfamily vector] basically an array
    \item[\ttfamily unordered\_map] basically a hash table, key and associated
      value

      Hint: {\tt unordered\_set} (no associated values) may be useful
    \item[\ttfamily pair] class with two fields ({\ttfamily first} and {\ttfamily second})
      with associated types
    \item[\ttfamily iterator] abstraction for pointers on containers
  \end{description}
\end{frame}
%%%%%%%%%%%%%%%%%%%%%%%%%%%%%%%%%%%%%%%%%%%%%%%%%%%%%%%%%%%%%%%%%%%%%%%%%%%%%%%%

%%%%%%%%%%%%%%%%%%%%%%%%%%%%%%%%%%%%%%%%%%%%%%%%%%%%%%%%%%%%%%%%%%%%%%%%%%%%%%%%
\begin{frame}
  \frametitle{Our Data Structures}

  \begin{description}
    \item[\ttfamily distance\_map] field {\ttfamily distances}, a lookup table
      for distances between cities, implemented with a 2D hash map (distances
      calculated in constructor)
    \vspace{1em}
    \item[\ttfamily individual] field {\ttfamily best\_individual}, consists of
      a tour and a metadata which may represent distance, fitness, normalized
      fitness or accumulated fitness
      \begin{description}
        \item[\ttfamily tour\_container] field {\ttfamily tour}, a vector of
          indexes

          Note, does not include the first index, that is defined by field
          {\ttfamily first\_index}
        \item[\ttfamily union] field {\ttfamily metadata},
          a union to all doubles, since we don't need
          distance/fitness/etc. values all at the same time
      \end{description}
    \vspace{1em}
    \item[\ttfamily population\_container] field {\ttfamily population}, a
      vector of individuals
  \end{description}
\end{frame}
%%%%%%%%%%%%%%%%%%%%%%%%%%%%%%%%%%%%%%%%%%%%%%%%%%%%%%%%%%%%%%%%%%%%%%%%%%%%%%%%

%%%%%%%%%%%%%%%%%%%%%%%%%%%%%%%%%%%%%%%%%%%%%%%%%%%%%%%%%%%%%%%%%%%%%%%%%%%%%%%%
\begin{frame}
  \frametitle{Our Functions}

  \begin{description}
    \item[\ttfamily iteration] performs one iteration of the genetic algorithm, it
      replaces {\tt population} with a new population
      \begin{itemize}
        \item Before iteration is called, it is assumed all individuals in the
          current population have a valid value of {\tt distance} for its
          metadata
      \end{itemize}
    \item [\ttfamily distance] calculates the distance of a {\tt tour}
    \item [\ttfamily selection] returns 100 (population size) pairs of iterators to
      individuals in the current population to use for crossovers
    \item [\ttfamily crossover] same as high-level explaination, returns a new
      individual
    \item [\ttfamily mutate] same as high-level explainatoin, modifies an individual
  \end{description}
\end{frame}
%%%%%%%%%%%%%%%%%%%%%%%%%%%%%%%%%%%%%%%%%%%%%%%%%%%%%%%%%%%%%%%%%%%%%%%%%%%%%%%%

%%%%%%%%%%%%%%%%%%%%%%%%%%%%%%%%%%%%%%%%%%%%%%%%%%%%%%%%%%%%%%%%%%%%%%%%%%%%%%%%
\begin{frame}
  \frametitle{Built-in Algorithms}

  All these standard C++ algorithms work with anything using the container
  interface

  \begin{description}
    \item[\ttfamily max\_element] returns an iterator the the largest element, you
      can use your own comparator function
    \item[\ttfamily min\_element] same as above, but the smallest element
    \item[\ttfamily sort] sorts a container, you can use your own comparator function
    \item[\ttfamily upper\_bound] returns an iterator to the first element greater
      than the value, only works on a {\bf sorted} container (if the default
      comparator isn't used, you have to use the same one used to sort the
      container)
    \item[\ttfamily random\_shuffle] does {\tt n} random swaps of the elements in
      the container
  \end{description}
\end{frame}
%%%%%%%%%%%%%%%%%%%%%%%%%%%%%%%%%%%%%%%%%%%%%%%%%%%%%%%%%%%%%%%%%%%%%%%%%%%%%%%%

\section{Tips}

%%%%%%%%%%%%%%%%%%%%%%%%%%%%%%%%%%%%%%%%%%%%%%%%%%%%%%%%%%%%%%%%%%%%%%%%%%%%%%%%
\begin{frame}
  \frametitle{Some Hurdles}

  Confusing C++ error messages involving templates
  \begin{itemize}
    \item Use {\ttfamily clang++} to compile your program for clearer error
      messages
  \end{itemize}

  \vspace{2em}

  For the profiler reports, it might get pretty bad

  \begin{itemize}
    \item Look for one of the main functions (selection, crossover, etc.)
    \item If the name is very weird, look where it's called from
    \item If you run into mangled names use {\ttfamily c++filt}
  \end{itemize}
\end{frame}
%%%%%%%%%%%%%%%%%%%%%%%%%%%%%%%%%%%%%%%%%%%%%%%%%%%%%%%%%%%%%%%%%%%%%%%%%%%%%%%%

%%%%%%%%%%%%%%%%%%%%%%%%%%%%%%%%%%%%%%%%%%%%%%%%%%%%%%%%%%%%%%%%%%%%%%%%%%%%%%%%
\begin{frame}[fragile]
  \frametitle{Example Profiler Output}

  {\ttfamily \tiny
    [32] std::\_Hashtable<unsigned int, std::pair<unsigned int const,
    std::unordered\_map<unsigned int, double, std::hash<unsigned int>,
    std::equal\_to<unsigned int>, std::allocator<std::pair<unsigned int const,
    double> > > >, std::allocator<std::pair<unsigned int const,
    std::unordered\_map<unsigned int, double, std::hash<unsigned int>,
    std::equal\_to<unsigned int>, std::allocator<std::pair<unsigned int const,
    double> > > > >, std::\_Select1st<std::pair<unsigned int const,
    std::unordered\_map<unsigned int, double, std::hash<unsigned int>,
    std::equal\_to<unsigned int>, std::allocator<std::pair<unsigned int const,
    double> > > > >, std::equal\_to<unsigned int>, std::hash<unsigned int>,
    std::\_\_detail::\_Mod\_range\_hashing,
    std::\_\_detail::\_Default\_ranged\_hash,
    std::\_\_detail::\_Prime\_rehash\_policy, false, false, true>::clear()
  }

  \vfill

  is actually {\tt distance\_map.clear()}, which is automatically called by the
  destructor
\end{frame}
%%%%%%%%%%%%%%%%%%%%%%%%%%%%%%%%%%%%%%%%%%%%%%%%%%%%%%%%%%%%%%%%%%%%%%%%%%%%%%%%

%%%%%%%%%%%%%%%%%%%%%%%%%%%%%%%%%%%%%%%%%%%%%%%%%%%%%%%%%%%%%%%%%%%%%%%%%%%%%%%%
\begin{frame}
  \frametitle{Things You Can Do}

  Well, it's the most basic implementation, so there should be a lot you can do

  \begin{itemize}
    \item Introduce threads, using pthreads, C++11 threads, OpenMP, etc.

    \vspace{1em}
   
    \item Experiment around with compiler options

    \vspace{1em}

    \item Use better algorithms or data structures
  \end{itemize}
\end{frame}
%%%%%%%%%%%%%%%%%%%%%%%%%%%%%%%%%%%%%%%%%%%%%%%%%%%%%%%%%%%%%%%%%%%%%%%%%%%%%%%%


\begin{frame}
  \frametitle{Things You Need to Do}

  {\bfseries Profile!}

  \vspace{1em}

  Keep the number of iterations constant between all profiles so they're
  comparable

  \vspace{1em}
    
  Start with a baseline profile (no changes)

  \vspace{1em}

  Pick your two best performance changes to add to the report
  \begin{itemize}
    \item Include a profile before and after the change (and only that
      change!)
    \item More specific instructions in the handout
  \end{itemize}

  \vspace{1em}

  You don't have to profile both changes from the baseline
\end{frame}
%%%%%%%%%%%%%%%%%%%%%%%%%%%%%%%%%%%%%%%%%%%%%%%%%%%%%%%%%%%%%%%%%%%%%%%%%%%%%%%%

%%%%%%%%%%%%%%%%%%%%%%%%%%%%%%%%%%%%%%%%%%%%%%%%%%%%%%%%%%%%%%%%%%%%%%%%%%%%%%%%
\begin{frame}
  \frametitle{Things To Notice}

  As you increase the number of iterations, your best answer should get better

  \vspace{1em}
 
  The faster your program, the more iterations you can do in the time limit

  \vspace{1em}

  The optimal answer is 7542

  \vspace{2em}
      
  Your program will be run on an unloaded equivalent of {\ttfamily ece459-1}

  \vspace{1em}

  The scores on the leaderboard are how many iterations you can do in 10 seconds
\end{frame}
%%%%%%%%%%%%%%%%%%%%%%%%%%%%%%%%%%%%%%%%%%%%%%%%%%%%%%%%%%%%%%%%%%%%%%%%%%%%%%%%

\plain{GLHF}

\end{document}
