\documentclass[aspectratio=43]{beamer}

% Text packages to stop warnings
\usepackage{lmodern}
\usepackage{textcomp}
\usepackage{listings}

% Themes
\usetheme{Boadilla}
\setbeamertemplate{footline}[page number]{}
\setbeamertemplate{navigation symbols}{}

% Suppress the navigation bar
\beamertemplatenavigationsymbolsempty

\lstset{basicstyle=\scriptsize, frame=single}

\newenvironment{changemargin}[1]{% 
  \begin{list}{}{% 
    \setlength{\topsep}{0pt}% 
    \setlength{\leftmargin}{#1}% 
    \setlength{\rightmargin}{1em}
    \setlength{\listparindent}{\parindent}% 
    \setlength{\itemindent}{\parindent}% 
    \setlength{\parsep}{\parskip}% 
  }% 
  \item[]}{\end{list}} 

\title{Lecture 08---Race Conditions; Mutexes}
\date{January 21, 2015}

\begin{document}

%%%%%%%%%%%%%%%%%%%%%%%%%%%%%%%%%%%%%%%%%%%%%%%%%%%%%%%%%%%%%%%%%%%%%%%%%%%%%%%%
\begin{frame}[plain]
  \titlepage
\end{frame}
%%%%%%%%%%%%%%%%%%%%%%%%%%%%%%%%%%%%%%%%%%%%%%%%%%%%%%%%%%%%%%%%%%%%%%%%%%%%%%%%

%%%%%%%%%%%%%%%%%%%%%%%%%%%%%%%%%%%%%%%%%%%%%%%%%%%%%%%%%%%%%%%%%%%%%%%%%%%%%%%%
\begin{frame}
  \frametitle{Roadmap}

  \Large
    \begin{changemargin}{2cm}
  Past: Non-blocking I/O;\\[1em]

  Now: Race Conditions, Locking.
    \end{changemargin}
  
\end{frame}
%%%%%%%%%%%%%%%%%%%%%%%%%%%%%%%%%%%%%%%%%%%%%%%%%%%%%%%%%%%%%%%%%%%%%%%%%%%%%%%%

%%%%%%%%%%%%%%%%%%%%%%%%%%%%%%%%%%%%%%%%%%%%%%%%%%%%%%%%%%%%%%%%%%%%%%%%%%%%%%%%
\part{Race Conditions}
\frame{\partpage}
%%%%%%%%%%%%%%%%%%%%%%%%%%%%%%%%%%%%%%%%%%%%%%%%%%%%%%%%%%%%%%%%%%%%%%%%%%%%%%%%

%%%%%%%%%%%%%%%%%%%%%%%%%%%%%%%%%%%%%%%%%%%%%%%%%%%%%%%%%%%%%%%%%%%%%%%%%%%%%%%%
\begin{frame}
  \frametitle{Race Conditions}

  \begin{changemargin}{2.5cm}
  \begin{itemize}
    \item A race occurs when you have two concurrent accesses to the
      same memory location, at least one of which is a {\bf write}.
  \end{itemize}~\\

   When there's a race, the final state may not be the same as
      running one access to completion and then the other.\\[1em]
   Race conditions arise between variables which
      are shared between threads.
  \end{changemargin}

\end{frame}
%%%%%%%%%%%%%%%%%%%%%%%%%%%%%%%%%%%%%%%%%%%%%%%%%%%%%%%%%%%%%%%%%%%%%%%%%%%%%%%%

%%%%%%%%%%%%%%%%%%%%%%%%%%%%%%%%%%%%%%%%%%%%%%%%%%%%%%%%%%%%%%%%%%%%%%%%%%%%%%%%
\begin{frame}[fragile]
  \frametitle{Example Data Race (Part 1)}

  \begin{changemargin}{2.5cm}
  \begin{lstlisting}
#include <stdlib.h>
#include <stdio.h>
#include <pthread.h>

void* run1(void* arg)
{
    int* x = (int*) arg;
    *x += 1;
}

void* run2(void* arg)
{
    int* x = (int*) arg;
    *x += 2;
}
  \end{lstlisting}
\end{changemargin}
\end{frame}
%%%%%%%%%%%%%%%%%%%%%%%%%%%%%%%%%%%%%%%%%%%%%%%%%%%%%%%%%%%%%%%%%%%%%%%%%%%%%%%%

%%%%%%%%%%%%%%%%%%%%%%%%%%%%%%%%%%%%%%%%%%%%%%%%%%%%%%%%%%%%%%%%%%%%%%%%%%%%%%%%
\begin{frame}[fragile]
  \frametitle{Example Data Race (Part 2)}

  \begin{changemargin}{2.5cm}
  \begin{lstlisting}
int main(int argc, char *argv[])
{
    int* x = malloc(sizeof(int));
    *x = 1;
    pthread_t t1, t2;
    pthread_create(&t1, NULL, &run1, x);
    pthread_join(t1, NULL);
    pthread_create(&t2, NULL, &run2, x);
    pthread_join(t2, NULL);
    printf("%d\n", *x);
    free(x);
    return EXIT_SUCCESS;
}
  \end{lstlisting}

     Do we have a data race? Why or why not?
  
  \begin{itemize}
    \item<2-> No, we don't. Only one thread is active at a time.
  \end{itemize}
   \end{changemargin}


\end{frame}
%%%%%%%%%%%%%%%%%%%%%%%%%%%%%%%%%%%%%%%%%%%%%%%%%%%%%%%%%%%%%%%%%%%%%%%%%%%%%%%%

%%%%%%%%%%%%%%%%%%%%%%%%%%%%%%%%%%%%%%%%%%%%%%%%%%%%%%%%%%%%%%%%%%%%%%%%%%%%%%%%
\begin{frame}[fragile]
  \frametitle{Example Data Race (Part 2B)}

  \begin{changemargin}{2.5cm}
  \begin{lstlisting}
int main(int argc, char *argv[])
{
    int* x = malloc(sizeof(int));
    *x = 1;
    pthread_t t1, t2;
    pthread_create(&t1, NULL, &run1, x);
    pthread_create(&t2, NULL, &run2, x);
    pthread_join(t1, NULL);
    pthread_join(t2, NULL);
    printf("%d\n", *x);
    free(x);
    return EXIT_SUCCESS;
}
  \end{lstlisting}

    Do we have a data race now? Why or why not?
  \begin{itemize}
    \item <2-> Yes, we do. We have 2 threads concurrently accessing the same data.
  \end{itemize}
   \end{changemargin}

\end{frame}
%%%%%%%%%%%%%%%%%%%%%%%%%%%%%%%%%%%%%%%%%%%%%%%%%%%%%%%%%%%%%%%%%%%%%%%%%%%%%%%%

%%%%%%%%%%%%%%%%%%%%%%%%%%%%%%%%%%%%%%%%%%%%%%%%%%%%%%%%%%%%%%%%%%%%%%%%%%%%%%%%
\begin{frame}[fragile]
  \frametitle{Tracing our Example Data Race}

  \begin{changemargin}{2.5cm}
   What are the possible outputs? (initially {\tt *x} is 1).

  \begin{lstlisting}[numbers=left]
run1                          run2   
D.1 = *x;                     D.1 = *x;
D.2 = D.1 + 1;                D.2 = D.1 + 2
*x = D.2;                     *x = D.2;
  \end{lstlisting}

  \begin{itemize}
    \item Memory reads and writes are key in data~races.
  \end{itemize}
  \end{changemargin}

\end{frame}
%%%%%%%%%%%%%%%%%%%%%%%%%%%%%%%%%%%%%%%%%%%%%%%%%%%%%%%%%%%%%%%%%%%%%%%%%%%%%%%%

%%%%%%%%%%%%%%%%%%%%%%%%%%%%%%%%%%%%%%%%%%%%%%%%%%%%%%%%%%%%%%%%%%%%%%%%%%%%%%%%
\begin{frame}
  \frametitle{Outcome of Example Data Race}

  \begin{changemargin}{2cm}
  \begin{itemize}
    \item Let's call the read and write from {\tt run1} R1 and W1; R2 and W2
      from {\tt run2}.
    \item Assuming a sane\footnote{sequentially consistent} memory model, $R_n$ must precede $W_n$. \alert{C/C++ don't guarantee that model.}
  \end{itemize}
  \vfill
  All possible orderings:
  \begin{center}
    \begin{tabular}{llll|l}
\multicolumn{4}{c|}{Order} & {\tt *x}\\
\hline
R1 & W1 & R2 & W2 & 4 \\
R1 & R2 & W1 & W2 & 3 \\
R1 & R2 & W2 & W1 & 2 \\
R2 & W2 & R1 & W1 & 4 \\
R2 & R1 & W2 & W1 & 2 \\
R2 & R1 & W1 & W2 & 3 \\
    \end{tabular}
  \end{center}
  \alert{WARNING: see the provisos in the written notes!}
  \end{changemargin}

\end{frame}
%%%%%%%%%%%%%%%%%%%%%%%%%%%%%%%%%%%%%%%%%%%%%%%%%%%%%%%%%%%%%%%%%%%%%%%%%%%%%%%%

%%%%%%%%%%%%%%%%%%%%%%%%%%%%%%%%%%%%%%%%%%%%%%%%%%%%%%%%%%%%%%%%%%%%%%%%%%%%%%%%
\begin{frame}
  \frametitle{Detecting Data Races Automatically}  

  \begin{changemargin}{2.5cm}
    Dynamic and static tools can help find data races in your program.
  \begin{itemize}
    \item {\tt helgrind} is one such tool. It runs your program 
      and analyzes it (and causes a large slowdown).
  \end{itemize}
    Run with {\tt valgrind --tool=helgrind <prog>}.\\[1em]
    It will warn you of possible data races along with locations.\\[1em]
    For useful debugging information, compile with debugging information
      ({\tt -g} flag for {\tt gcc}).
  \end{changemargin}
\end{frame}
%%%%%%%%%%%%%%%%%%%%%%%%%%%%%%%%%%%%%%%%%%%%%%%%%%%%%%%%%%%%%%%%%%%%%%%%%%%%%%%%

%%%%%%%%%%%%%%%%%%%%%%%%%%%%%%%%%%%%%%%%%%%%%%%%%%%%%%%%%%%%%%%%%%%%%%%%%%%%%%%%
\begin{frame}[fragile]
  \frametitle{Helgrind Output for Example}

  \begin{changemargin}{.8cm}
  \begin{lstlisting}
==5036== Possible data race during read of size 4 at
         0x53F2040 by thread #3
==5036== Locks held: none
==5036==    at 0x400710: run2 (in datarace.c:14)
...
==5036== 
==5036== This conflicts with a previous write of size 4 by
         thread #2
==5036== Locks held: none
==5036==    at 0x400700: run1 (in datarace.c:8)
...
==5036== 
==5036== Address 0x53F2040 is 0 bytes inside a block of size
         4 alloc'd
...
==5036==    by 0x4005AE: main (in datarace.c:19)
  \end{lstlisting}
  \end{changemargin}
\end{frame}
%%%%%%%%%%%%%%%%%%%%%%%%%%%%%%%%%%%%%%%%%%%%%%%%%%%%%%%%%%%%%%%%%%%%%%%%%%%%%%%%


\section{Synchronization}
%%%%%%%%%%%%%%%%%%%%%%%%%%%%%%%%%%%%%%%%%%%%%%%%%%%%%%%%%%%%%%%%%%%%%%%%%%%%%%%%
\begin{frame}
  \frametitle{Mutual Exclusion}

\begin{changemargin}{1.5cm}
    Mutexes are the most basic type of synchronization.
    \vfill
    \begin{itemize}
    \item Only one thread can execute code protected by a mutex at a time.
    \vfill
    \item All other threads must wait until the mutex is free before they can
      execute protected code.
    \end{itemize}
\end{changemargin}

\end{frame}
%%%%%%%%%%%%%%%%%%%%%%%%%%%%%%%%%%%%%%%%%%%%%%%%%%%%%%%%%%%%%%%%%%%%%%%%%%%%%%%%

%%%%%%%%%%%%%%%%%%%%%%%%%%%%%%%%%%%%%%%%%%%%%%%%%%%%%%%%%%%%%%%%%%%%%%%%%%%%%%%%
\begin{frame}
  \frametitle{Live Coding Example: Mutual Exclusion}
\end{frame}
%%%%%%%%%%%%%%%%%%%%%%%%%%%%%%%%%%%%%%%%%%%%%%%%%%%%%%%%%%%%%%%%%%%%%%%%%%%%%%%%

%%%%%%%%%%%%%%%%%%%%%%%%%%%%%%%%%%%%%%%%%%%%%%%%%%%%%%%%%%%%%%%%%%%%%%%%%%%%%%%%
\begin{frame}[fragile]
  \frametitle{Creating Mutexes---Pthreads Example}

\begin{changemargin}{1.5cm}
  \begin{lstlisting}
pthread_mutex_t m1 = PTHREAD_MUTEX_INITIALIZER;
pthread_mutex_t m2;

pthread_mutex_init(&m2, NULL);
...
pthread_mutex_destroy(&m1);
pthread_mutex_destroy(&m2);
  \end{lstlisting}

  \begin{itemize}
    \item Two ways to initialize mutexes: statically and dynamically
    \item If you want to include attributes, you need to use the dynamic version
  \end{itemize}
\end{changemargin}

\end{frame}
%%%%%%%%%%%%%%%%%%%%%%%%%%%%%%%%%%%%%%%%%%%%%%%%%%%%%%%%%%%%%%%%%%%%%%%%%%%%%%%%

%%%%%%%%%%%%%%%%%%%%%%%%%%%%%%%%%%%%%%%%%%%%%%%%%%%%%%%%%%%%%%%%%%%%%%%%%%%%%%%%
\begin{frame}[fragile]
  \frametitle{Creating Mutexes---C++ Example}

\begin{changemargin}{1.5cm}
  \begin{lstlisting}
mutex m1;
mutex *m2;

m2 = new mutex();
// ...

delete(m2);
  \end{lstlisting}
\end{changemargin}

\end{frame}
%%%%%%%%%%%%%%%%%%%%%%%%%%%%%%%%%%%%%%%%%%%%%%%%%%%%%%%%%%%%%%%%%%%%%%%%%%%%%%%%

%%%%%%%%%%%%%%%%%%%%%%%%%%%%%%%%%%%%%%%%%%%%%%%%%%%%%%%%%%%%%%%%%%%%%%%%%%%%%%%%
\begin{frame}
  \frametitle{Mutex Attributes}

\begin{changemargin}{1.5cm}
  \begin{itemize}
    \item {\bf Protocol}: specifies the protocol used to prevent priority
      inversions for a mutex
    \item {\bf Prioceiling}: specifies the priority ceiling of a mutex
    \item {\bf Process-shared}: specifies the process sharing of a mutex
  \end{itemize}
  You can specify a mutex as {\it process shared} so that you can access it
  between processes. In that case, you need to use shared memory and {\tt mmap},
  which we won't get into.
\end{changemargin}

\end{frame}
%%%%%%%%%%%%%%%%%%%%%%%%%%%%%%%%%%%%%%%%%%%%%%%%%%%%%%%%%%%%%%%%%%%%%%%%%%%%%%%%

%%%%%%%%%%%%%%%%%%%%%%%%%%%%%%%%%%%%%%%%%%%%%%%%%%%%%%%%%%%%%%%%%%%%%%%%%%%%%%%%
\begin{frame}[fragile]
  \frametitle{Using Mutexes: Pthreads Example}

\begin{changemargin}{1.5cm}
  \begin{lstlisting}
// code
pthread_mutex_lock(&m1);
// protected code
pthread_mutex_unlock(&m1);
// more code
  \end{lstlisting}
  \vfill
  \begin{itemize}
    \item Everything within the {\tt lock} and {\tt unlock} is protected.
    \item Be careful to avoid deadlocks if you are using multiple mutexes.
    \item Also you can use {\tt pthread\_mutex\_trylock}, if needed.
  \end{itemize}
\end{changemargin}
\end{frame}
%%%%%%%%%%%%%%%%%%%%%%%%%%%%%%%%%%%%%%%%%%%%%%%%%%%%%%%%%%%%%%%%%%%%%%%%%%%%%%%%

%%%%%%%%%%%%%%%%%%%%%%%%%%%%%%%%%%%%%%%%%%%%%%%%%%%%%%%%%%%%%%%%%%%%%%%%%%%%%%%%
\begin{frame}[fragile]
  \frametitle{Using Mutexes: C++11 Threads Example}

\begin{changemargin}{1.5cm}
  \begin{lstlisting}
// code
m1.lock(); 
// protected code
m1.unlock();
// more code
  \end{lstlisting}
\end{changemargin}
\end{frame}
%%%%%%%%%%%%%%%%%%%%%%%%%%%%%%%%%%%%%%%%%%%%%%%%%%%%%%%%%%%%%%%%%%%%%%%%%%%%%%%%

% XXX For next year: add mention of reentrant/recursive locks (coulda sworn that was already there; mentioned in L16)

%%%%%%%%%%%%%%%%%%%%%%%%%%%%%%%%%%%%%%%%%%%%%%%%%%%%%%%%%%%%%%%%%%%%%%%%%%%%%%%%
\begin{frame}[fragile]
  \frametitle{Data Race Example}

\begin{changemargin}{1.5cm}
  Recall that \structure{dataraces} occur when two concurrent actions access the same
  variable and at least one of them is a {\bf write}.


  \begin{lstlisting}
...
static int counter = 0;

void* run(void* arg) {
    for (int i = 0; i < 100; ++i) {
        ++counter;
    }
}

int main(int argc, char *argv[])
{
    // Create 8 threads
    // Join 8 threads
    printf("counter = %i\n", counter);
}
  \end{lstlisting}

Is there a datarace in this example? If so, how would we fix it?
\end{changemargin}

\end{frame}
%%%%%%%%%%%%%%%%%%%%%%%%%%%%%%%%%%%%%%%%%%%%%%%%%%%%%%%%%%%%%%%%%%%%%%%%%%%%%%%%

%%%%%%%%%%%%%%%%%%%%%%%%%%%%%%%%%%%%%%%%%%%%%%%%%%%%%%%%%%%%%%%%%%%%%%%%%%%%%%%%
\begin{frame}[fragile]
  \frametitle{Example Problem Solution}
\begin{changemargin}{1.5cm}
  \begin{lstlisting}[escapechar=!]
...
!\structure{static pthread\_mutex\_t mutex = PTHREAD\_MUTEX\_INITIALIZER;}!
static int counter = 0;

void* run(void* arg) {
    for (int i = 0; i < 100; ++i) {
        !\structure{pthread\_mutex\_lock(\&mutex);}!
        ++counter;
        !\structure{pthread\_mutex\_unlock(\&mutex);}!
    }
}

int main(int argc, char *argv[])
{
    // Create 8 threads
    // Join 8 threads
    !\structure{pthread\_mutex\_destroy(\&mutex);}!
    printf("counter = %i\n", counter);
}
  \end{lstlisting}
\end{changemargin}

\end{frame}
%%%%%%%%%%%%%%%%%%%%%%%%%%%%%%%%%%%%%%%%%%%%%%%%%%%%%%%%%%%%%%%%%%%%%%%%%%%%%%%%

\end{document}
