\documentclass[aspectratio=43]{beamer}

% Text packages to stop warnings
\usepackage{lmodern}
\usepackage{textcomp}
\usepackage{listings}
\usepackage{multirow}

% Themes
\usetheme{Boadilla}
\setbeamertemplate{footline}[page number]{}
\setbeamertemplate{navigation symbols}{}

% Suppress the navigation bar
\beamertemplatenavigationsymbolsempty

\lstset{basicstyle=\scriptsize, frame=single}

\newenvironment{changemargin}[1]{% 
  \begin{list}{}{% 
    \setlength{\topsep}{0pt}% 
    \setlength{\leftmargin}{#1}% 
    \setlength{\rightmargin}{1em}
    \setlength{\listparindent}{\parindent}% 
    \setlength{\itemindent}{\parindent}% 
    \setlength{\parsep}{\parskip}% 
  }% 
  \item[]}{\end{list}} 

\title{Lecture 09---More Synchronization}
\date{January 23, 2015}

\begin{document}

%%%%%%%%%%%%%%%%%%%%%%%%%%%%%%%%%%%%%%%%%%%%%%%%%%%%%%%%%%%%%%%%%%%%%%%%%%%%%%%%
\begin{frame}[plain]
  \titlepage
\end{frame}
%%%%%%%%%%%%%%%%%%%%%%%%%%%%%%%%%%%%%%%%%%%%%%%%%%%%%%%%%%%%%%%%%%%%%%%%%%%%%%%%

%%%%%%%%%%%%%%%%%%%%%%%%%%%%%%%%%%%%%%%%%%%%%%%%%%%%%%%%%%%%%%%%%%%%%%%%%%%%%%%%
\begin{frame}
  \frametitle{Roadmap}

  \Large
    \begin{changemargin}{2cm}
      Last Time: Race Conditions, Locking (mutexes)\\[1em]
      Now: More Synchronization Mechanisms
    \end{changemargin}
  
\end{frame}
%%%%%%%%%%%%%%%%%%%%%%%%%%%%%%%%%%%%%%%%%%%%%%%%%%%%%%%%%%%%%%%%%%%%%%%%%%%%%%%%


\part{More Synchronization}
\frame{\partpage}
%%%%%%%%%%%%%%%%%%%%%%%%%%%%%%%%%%%%%%%%%%%%%%%%%%%%%%%%%%%%%%%%%%%%%%%%%%%%%%%%
\begin{frame}
  \frametitle{Mutexes Recap}

  \begin{changemargin}{2.5cm}
     Our focus is on \structure{how to use mutexes correctly}:
  \begin{itemize}
    \item Call {\tt lock} on mutex {\tt m1}. Upon return from
      {\tt lock}, you have exclusive access to {\tt m1} until you
      {\tt unlock} it.
    \item Other calls to {\tt lock} {\tt m1} will not return
      until {\tt m1} is available.
    \end{itemize}
    For background on selection algorithms, look at
      \href{http://en.wikipedia.org/wiki/Lamport\%27s_bakery_algorithm}
      {Lamport's bakery algorithm}. \\ (Not in scope for this
      course.)
  \end{changemargin}
\end{frame}
%%%%%%%%%%%%%%%%%%%%%%%%%%%%%%%%%%%%%%%%%%%%%%%%%%%%%%%%%%%%%%%%%%%%%%%%%%%%%%%%

%%%%%%%%%%%%%%%%%%%%%%%%%%%%%%%%%%%%%%%%%%%%%%%%%%%%%%%%%%%%%%%%%%%%%%%%%%%%%%%%
\begin{frame}
  \frametitle{More on Mutexes}

  \begin{changemargin}{2.5cm}
     Can also ``try-lock'': grab lock if available, else return
     to caller (and do something else).\\[1em]

     Excessive use of locks can serialize programs. 
\begin{itemize}
\item Linux kernel used to
rely on a Big Kernel Lock protecting lots of resources in the 2.0 era.
\item Linux 2.2 improved performance on SMPs by cutting down on the use
of the BKL.
\end{itemize}~\\[1em]

Note: in Windows, ``mutex'' is an inter-process
communication mechanism. Windows ``critical sections'' are our mutexes.

  \end{changemargin}
\end{frame}
%%%%%%%%%%%%%%%%%%%%%%%%%%%%%%%%%%%%%%%%%%%%%%%%%%%%%%%%%%%%%%%%%%%%%%%%%%%%%%%%

%%%%%%%%%%%%%%%%%%%%%%%%%%%%%%%%%%%%%%%%%%%%%%%%%%%%%%%%%%%%%%%%%%%%%%%%%%%%%%%%
\begin{frame}
  \frametitle{Spinlocks}

  \begin{changemargin}{2.5cm}
    Functionally equivalent to {\tt mutex}.

  \begin{itemize}
    \item  {\tt pthread\_spinlock\_t},
      {\tt pthread\_spin\_lock}, {\tt pthread\_spin\_trylock} and friends
  \end{itemize}
 
    \vfill
    Implementation difference: spinlocks will repeatedly try the lock and will not put
      the thread to sleep.
    \vfill
    Good if your protected code is short.
    \vfill
    Mutexes may be implemented as a combination between spinning and sleeping
      (spin for a short time, then sleep).
  \end{changemargin}
\end{frame}
%%%%%%%%%%%%%%%%%%%%%%%%%%%%%%%%%%%%%%%%%%%%%%%%%%%%%%%%%%%%%%%%%%%%%%%%%%%%%%%%


%%%%%%%%%%%%%%%%%%%%%%%%%%%%%%%%%%%%%%%%%%%%%%%%%%%%%%%%%%%%%%%%%%%%%%%%%%%%%%%%
\begin{frame}
  \frametitle{Read-Write Locks}

  \begin{changemargin}{2.5cm}
  Two observations:
  \begin{itemize}
    \item If there are only reads, there's no datarace.
    \item Often, writes are relatively rare.
  \end{itemize}
  With mutexes/spinlocks, you have to lock the data, even for a read,
      since a write could happen.\\[1em]

  But, most of the time, reads can happen in parallel, as long as
      there's no write.\\[1em]

  Solution: Multiple threads can hold a read lock\\ \hspace*{2em} ({\tt pthread\_rwlock\_rdlock})\\
      but only one thread may hold the associated write lock\\ \hspace*{2em} ({\tt pthread\_rwlock\_wrlock});\\
      grabbing the write waits until  current readers are done.
  \end{changemargin}
\end{frame}
%%%%%%%%%%%%%%%%%%%%%%%%%%%%%%%%%%%%%%%%%%%%%%%%%%%%%%%%%%%%%%%%%%%%%%%%%%%%%%%%

%%%%%%%%%%%%%%%%%%%%%%%%%%%%%%%%%%%%%%%%%%%%%%%%%%%%%%%%%%%%%%%%%%%%%%%%%%%%%%%%
\begin{frame}
  \frametitle{Semaphores}

  \begin{changemargin}{2cm}
    Semaphores have a {\tt value}.
      You specify initial {\tt value}.\\[1em]
    Semaphores allow sharing of a \# of instances of a resource.\\[1em]
    Two fundamental operations: {\tt wait} and {\tt post}.\\[1em]
  \begin{itemize}
    \item {\tt wait} is like {\tt lock}; reserves the resource and decrements the value.
      \begin{itemize}
        \item If value is 0, sleep until value is greater than 0.
      \end{itemize}
    \item {\tt post} is like {\tt unlock}; releases the resource and increments the value.
  \end{itemize}
  \end{changemargin}
\end{frame}
%%%%%%%%%%%%%%%%%%%%%%%%%%%%%%%%%%%%%%%%%%%%%%%%%%%%%%%%%%%%%%%%%%%%%%%%%%%%%%%%

%%%%%%%%%%%%%%%%%%%%%%%%%%%%%%%%%%%%%%%%%%%%%%%%%%%%%%%%%%%%%%%%%%%%%%%%%%%%%%%%
\begin{frame}
  \frametitle{Barriers}

  \begin{changemargin}{2.5cm}
    Allows you to ensure that (some subset of) a collection 
    of threads all reach the barrier before finishing.\\[1em]

    Pthreads: A barrier is a {\tt pthread\_barrier\_t}.\\[1em]

    Functions: {\tt \_init()} (parameter: how many threads the barrier
    should wait for) and {\tt \_destroy()}.\\[1em]

    Also {\tt \_wait()}: similar to {\tt pthread\_join()}, but waits
      for the specified number of threads to arrive at the barrier
  \end{changemargin}
\end{frame}
%%%%%%%%%%%%%%%%%%%%%%%%%%%%%%%%%%%%%%%%%%%%%%%%%%%%%%%%%%%%%%%%%%%%%%%%%%%%%%%%

%%%%%%%%%%%%%%%%%%%%%%%%%%%%%%%%%%%%%%%%%%%%%%%%%%%%%%%%%%%%%%%%%%%%%%%%%%%%%%%%
\begin{frame}
  \frametitle{Lock-Free Algorithms}

  \begin{changemargin}{2.5cm}
    We'll talk more about this in a few weeks.\\[1em]

    Modern CPUs support atomic operations, such as compare-and-swap, which
enable experts to write lock-free code.\\[1em]

    Lock-free implementations are extremely complicated and must still contain certain synchronization constructs.
  \end{changemargin}
\end{frame}
%%%%%%%%%%%%%%%%%%%%%%%%%%%%%%%%%%%%%%%%%%%%%%%%%%%%%%%%%%%%%%%%%%%%%%%%%%%%%%%%

%%%%%%%%%%%%%%%%%%%%%%%%%%%%%%%%%%%%%%%%%%%%%%%%%%%%%%%%%%%%%%%%%%%%%%%%%%%%%%%%
\begin{frame}[fragile]
  \frametitle{Semaphores Usage}
  
  \begin{changemargin}{1cm}
  \begin{lstlisting}
#include <semaphore.h>

int sem_init(sem_t *sem, int pshared, unsigned int value);
int sem_destroy(sem_t *sem);
int sem_post(sem_t *sem);
int sem_wait(sem_t *sem);
int sem_trywait(sem_t *sem);
  \end{lstlisting}

  \begin{itemize}
    \item Also must link with {\tt -pthread} (or {\tt -lrt} on Solaris).
    \item All functions return {\tt 0} on success.
    \item Same usage as mutexes in terms of passing pointers.
  \end{itemize}~\\[1em]
    How could you use as {\tt semaphore} as a {\tt mutex}?
    \begin{itemize}
    \item<2-> If the initial {\tt value} is 1 and you use {\tt wait} to lock
      and {\tt post} to unlock, it's equivalent to a {\tt mutex}.
    \end{itemize}
  \end{changemargin}

\end{frame}
%%%%%%%%%%%%%%%%%%%%%%%%%%%%%%%%%%%%%%%%%%%%%%%%%%%%%%%%%%%%%%%%%%%%%%%%%%%%%%%%

%%%%%%%%%%%%%%%%%%%%%%%%%%%%%%%%%%%%%%%%%%%%%%%%%%%%%%%%%%%%%%%%%%%%%%%%%%%%%%%%
\begin{frame}[fragile]
  \frametitle{Semaphores for Signalling}

  \begin{changemargin}{1cm}
  Here's an example from the book. How would you make this always print
  ``Thread 1'' then ``Thread 2'' using semaphores?

  \begin{lstlisting}
#include <pthread.h>
#include <stdio.h>
#include <semaphore.h>
#include <stdlib.h>

void* p1 (void* arg) { printf("Thread 1\n"); }

void* p2 (void* arg) { printf("Thread 2\n"); }

int main(int argc, char *argv[])
{
    pthread_t thread[2];
    pthread_create(&thread[0], NULL, p1, NULL);
    pthread_create(&thread[1], NULL, p2, NULL);
    pthread_join(thread[0], NULL);
    pthread_join(thread[1], NULL);
    return EXIT_SUCCESS;
}
  \end{lstlisting}
\end{changemargin}
\end{frame}
%%%%%%%%%%%%%%%%%%%%%%%%%%%%%%%%%%%%%%%%%%%%%%%%%%%%%%%%%%%%%%%%%%%%%%%%%%%%%%%%

%%%%%%%%%%%%%%%%%%%%%%%%%%%%%%%%%%%%%%%%%%%%%%%%%%%%%%%%%%%%%%%%%%%%%%%%%%%%%%%%
\begin{frame}[fragile]
  \frametitle{Semaphores for Signalling}

  \begin{changemargin}{1cm}
  Here's their solution. Is it actually correct?

  \begin{lstlisting}
sem_t sem;
void* p1 (void* arg) {
  printf("Thread 1\n");
  sem_post(&sem);
}
void* p2 (void* arg) {
  sem_wait(&sem);
  printf("Thread 2\n");
}

int main(int argc, char *argv[])
{
    pthread_t thread[2];
    sem_init(&sem, 0, /* value: */ 1);
    pthread_create(&thread[0], NULL, p1, NULL);
    pthread_create(&thread[1], NULL, p2, NULL);
    pthread_join(thread[0], NULL);
    pthread_join(thread[1], NULL);
    sem_destroy(&sem);
}
  \end{lstlisting}
\end{changemargin}

\end{frame}
%%%%%%%%%%%%%%%%%%%%%%%%%%%%%%%%%%%%%%%%%%%%%%%%%%%%%%%%%%%%%%%%%%%%%%%%%%%%%%%%

%%%%%%%%%%%%%%%%%%%%%%%%%%%%%%%%%%%%%%%%%%%%%%%%%%%%%%%%%%%%%%%%%%%%%%%%%%%%%%%%
\begin{frame}[fragile]
  \frametitle{Semaphores for Signalling}

  
  \begin{changemargin}{1cm}
  \begin{enumerate}
    \item {\tt value} is initially 1.
    \vfill
    \item Say {\tt p2} hits its {\tt sem\_wait} first and succeeds.
    \vfill
    \item {\tt value} is now 0 and {\tt p2} prints ``Thread 2'' first.
  \end{enumerate}
  \begin{itemize}
    \item If {\tt p1} happens first, it would just increase
      {\tt value} to 2.\vfill
    \item<2-> Fix: set the initial {\tt value} to \alert{0}.\\[1em]
Then, if {\tt p2} hits its {\tt sem\_wait} first, it will
      not print until {\tt p1} posts (and prints ``Thread 1'') first.
  \end{itemize}~\\[1em]

  \end{changemargin}

\end{frame}
%%%%%%%%%%%%%%%%%%%%%%%%%%%%%%%%%%%%%%%%%%%%%%%%%%%%%%%%%%%%%%%%%%%%%%%%%%%%%%%%

%%%%%%%%%%%%%%%%%%%%%%%%%%%%%%%%%%%%%%%%%%%%%%%%%%%%%%%%%%%%%%%%%%%%%%%%%%%%%%%%
\begin{frame}[fragile]
  \frametitle{{\tt volatile} Keyword}

\begin{changemargin}{1.5cm}
  \begin{itemize}
    \item Used to notify the compiler that the variable may be changed by ``external forces''. For instance,
    %% \item Prevents the compiler from substituting values instead of evaluating
    %%   the value (with proper locking you'll never need this, why?)
  \end{itemize}

  \begin{lstlisting}
int i = 0;

while (i != 255) {
  ...
  \end{lstlisting}

{\tt volatile} prevents this from being optimized to:

  \begin{lstlisting}
int i = 0;

while (true) {
  ...
  \end{lstlisting}

  \begin{itemize}
    \item Variable will not actually be {\tt volatile} in the critical section
      and only prevents useful optimizations.
    \item Usually wrong unless there is a {\bf very} good reason for it.
  \end{itemize}
\end{changemargin}

\end{frame}
%%%%%%%%%%%%%%%%%%%%%%%%%%%%%%%%%%%%%%%%%%%%%%%%%%%%%%%%%%%%%%%%%%%%%%%%%%%%%%%%

\begin{frame}
  \frametitle{C++ atomics}

  Coming soon. Short version: wrap the type in {\tt atomic}, e.g.

  {\tt atomic<int> x;}
\end{frame}


\end{document}
