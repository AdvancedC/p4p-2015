\documentclass[aspectratio=43]{beamer}

% Text packages to stop warnings
\usepackage{lmodern}
\usepackage{textcomp}
\usepackage{listings}
\usepackage{multirow}

% Themes
\usetheme{Boadilla}
\setbeamertemplate{footline}[page number]{}
\setbeamertemplate{navigation symbols}{}

% Suppress the navigation bar
\beamertemplatenavigationsymbolsempty

\lstset{basicstyle=\scriptsize, frame=single}

\newenvironment{changemargin}[1]{% 
  \begin{list}{}{% 
    \setlength{\topsep}{0pt}% 
    \setlength{\leftmargin}{#1}% 
    \setlength{\rightmargin}{1em}
    \setlength{\listparindent}{\parindent}% 
    \setlength{\itemindent}{\parindent}% 
    \setlength{\parsep}{\parskip}% 
  }% 
  \item[]}{\end{list}} 

\title{Lecture 08---Race Conditions; More Synchronization}
\date{January 21, 2015}

\begin{document}

%%%%%%%%%%%%%%%%%%%%%%%%%%%%%%%%%%%%%%%%%%%%%%%%%%%%%%%%%%%%%%%%%%%%%%%%%%%%%%%%
\begin{frame}[plain]
  \titlepage
\end{frame}
%%%%%%%%%%%%%%%%%%%%%%%%%%%%%%%%%%%%%%%%%%%%%%%%%%%%%%%%%%%%%%%%%%%%%%%%%%%%%%%%

%%%%%%%%%%%%%%%%%%%%%%%%%%%%%%%%%%%%%%%%%%%%%%%%%%%%%%%%%%%%%%%%%%%%%%%%%%%%%%%%
\begin{frame}
  \frametitle{Roadmap}

  \Large
    \begin{changemargin}{2cm}
      Last Time: Race Conditions, Locking (mutexes)\\[1em]
      Now: More Synchronization Mechanisms
    \end{changemargin}
  
\end{frame}
%%%%%%%%%%%%%%%%%%%%%%%%%%%%%%%%%%%%%%%%%%%%%%%%%%%%%%%%%%%%%%%%%%%%%%%%%%%%%%%%


\part{More Synchronization}
\frame{\partpage}
%%%%%%%%%%%%%%%%%%%%%%%%%%%%%%%%%%%%%%%%%%%%%%%%%%%%%%%%%%%%%%%%%%%%%%%%%%%%%%%%
\begin{frame}
  \frametitle{Mutexes Recap}

  \begin{changemargin}{2.5cm}
     Our focus is on \structure{how to use mutexes correctly}:
  \begin{itemize}
    \item Call {\tt lock} on mutex {\tt m1}. Upon return from
      {\tt lock}, you have exclusive access to {\tt m1} until you
      {\tt unlock} it.
    \item Other calls to {\tt lock} {\tt m1} will not return
      until {\tt m1} is available.
    \end{itemize}
    For background on selection algorithms, look at
      \href{http://en.wikipedia.org/wiki/Lamport\%27s_bakery_algorithm}
      {Lamport's bakery algorithm}. \\ (Not in scope for this
      course.)
  \end{changemargin}
\end{frame}
%%%%%%%%%%%%%%%%%%%%%%%%%%%%%%%%%%%%%%%%%%%%%%%%%%%%%%%%%%%%%%%%%%%%%%%%%%%%%%%%

%%%%%%%%%%%%%%%%%%%%%%%%%%%%%%%%%%%%%%%%%%%%%%%%%%%%%%%%%%%%%%%%%%%%%%%%%%%%%%%%
\begin{frame}
  \frametitle{More on Mutexes}

  \begin{changemargin}{2.5cm}
     Can also ``try-lock'': grab lock if available, else return
     to caller (and do something else).\\[1em]

     Excessive use of locks can serialize programs. 
\begin{itemize}
\item Linux kernel used to
rely on a Big Kernel Lock protecting lots of resources in the 2.0 era.
\item Linux 2.2 improved performance on SMPs by cutting down on the use
of the BKL.
\end{itemize}~\\[1em]

Note: in Windows, ``mutex'' is an inter-process
communication mechanism. Windows ``critical sections'' are our mutexes.

  \end{changemargin}
\end{frame}
%%%%%%%%%%%%%%%%%%%%%%%%%%%%%%%%%%%%%%%%%%%%%%%%%%%%%%%%%%%%%%%%%%%%%%%%%%%%%%%%

%%%%%%%%%%%%%%%%%%%%%%%%%%%%%%%%%%%%%%%%%%%%%%%%%%%%%%%%%%%%%%%%%%%%%%%%%%%%%%%%
\begin{frame}
  \frametitle{Spinlocks}

  \begin{changemargin}{2.5cm}
    Functionally equivalent to {\tt mutex}.

  \begin{itemize}
    \item  {\tt pthread\_spinlock\_t},
      {\tt pthread\_spin\_lock}, {\tt pthread\_spin\_trylock} and friends
  \end{itemize}
 
    \vfill
    Implementation difference: spinlocks will repeatedly try the lock and will not put
      the thread to sleep.
    \vfill
    Good if your protected code is short.
    \vfill
    Mutexes may be implemented as a combination between spinning and sleeping
      (spin for a short time, then sleep).
  \end{changemargin}
\end{frame}
%%%%%%%%%%%%%%%%%%%%%%%%%%%%%%%%%%%%%%%%%%%%%%%%%%%%%%%%%%%%%%%%%%%%%%%%%%%%%%%%


%%%%%%%%%%%%%%%%%%%%%%%%%%%%%%%%%%%%%%%%%%%%%%%%%%%%%%%%%%%%%%%%%%%%%%%%%%%%%%%%
\begin{frame}
  \frametitle{Read-Write Locks}

  \begin{changemargin}{2.5cm}
  Two observations:
  \begin{itemize}
    \item If there are only reads, there's no datarace.
    \item Often, writes are relatively rare.
  \end{itemize}
  With mutexes/spinlocks, you have to lock the data, even for a read,
      since a write could happen.\\[1em]

  But, most of the time, reads can happen in parallel, as long as
      there's no write.\\[1em]

  Solution: Multiple threads can hold a read lock\\ \hspace*{2em} ({\tt pthread\_rwlock\_rdlock})\\
      but only one thread may hold the associated write lock\\ \hspace*{2em} ({\tt pthread\_rwlock\_wrlock});\\
      grabbing the write waits until  current readers are done.
  \end{changemargin}
\end{frame}
%%%%%%%%%%%%%%%%%%%%%%%%%%%%%%%%%%%%%%%%%%%%%%%%%%%%%%%%%%%%%%%%%%%%%%%%%%%%%%%%

%%%%%%%%%%%%%%%%%%%%%%%%%%%%%%%%%%%%%%%%%%%%%%%%%%%%%%%%%%%%%%%%%%%%%%%%%%%%%%%%
\begin{frame}
  \frametitle{Semaphores}

  \begin{changemargin}{2cm}
    Semaphores have a {\tt value}.
      You specify initial {\tt value}.\\[1em]
    Semaphores allow sharing of a \# of instances of a resource.\\[1em]
    Two fundamental operations: {\tt wait} and {\tt post}.\\[1em]
  \begin{itemize}
    \item {\tt wait} is like {\tt lock}; reserves the resource and decrements the value.
      \begin{itemize}
        \item If value is 0, sleep until value is greater than 0.
      \end{itemize}
    \item {\tt post} is like {\tt unlock}; releases the resource and increments the value.
  \end{itemize}
  \end{changemargin}
\end{frame}
%%%%%%%%%%%%%%%%%%%%%%%%%%%%%%%%%%%%%%%%%%%%%%%%%%%%%%%%%%%%%%%%%%%%%%%%%%%%%%%%

%%%%%%%%%%%%%%%%%%%%%%%%%%%%%%%%%%%%%%%%%%%%%%%%%%%%%%%%%%%%%%%%%%%%%%%%%%%%%%%%
\begin{frame}
  \frametitle{Barriers}

  \begin{changemargin}{2.5cm}
    Allows you to ensure that (some subset of) a collection 
    of threads all reach the barrier before finishing.\\[1em]

    Pthreads: A barrier is a {\tt pthread\_barrier\_t}.\\[1em]

    Functions: {\tt \_init()} (parameter: how many threads the barrier
    should wait for) and {\tt \_destroy()}.\\[1em]

    Also {\tt \_wait()}: similar to {\tt pthread\_join()}, but waits
      for the specified number of threads to arrive at the barrier
  \end{changemargin}
\end{frame}
%%%%%%%%%%%%%%%%%%%%%%%%%%%%%%%%%%%%%%%%%%%%%%%%%%%%%%%%%%%%%%%%%%%%%%%%%%%%%%%%

%%%%%%%%%%%%%%%%%%%%%%%%%%%%%%%%%%%%%%%%%%%%%%%%%%%%%%%%%%%%%%%%%%%%%%%%%%%%%%%%
\begin{frame}
  \frametitle{Lock-Free Algorithms}

  \begin{changemargin}{2.5cm}
    We'll talk more about this in a few weeks.\\[1em]

    Modern CPUs support atomic operations, such as compare-and-swap, which
enable experts to write lock-free code.\\[1em]

    Lock-free implementations are extremely complicated and must still contain certain synchronization constructs.
  \end{changemargin}
\end{frame}
%%%%%%%%%%%%%%%%%%%%%%%%%%%%%%%%%%%%%%%%%%%%%%%%%%%%%%%%%%%%%%%%%%%%%%%%%%%%%%%%

%%%%%%%%%%%%%%%%%%%%%%%%%%%%%%%%%%%%%%%%%%%%%%%%%%%%%%%%%%%%%%%%%%%%%%%%%%%%%%%%
\begin{frame}[fragile]
  \frametitle{Semaphores Usage}
  
  \begin{changemargin}{1cm}
  \begin{lstlisting}
#include <semaphore.h>

int sem_init(sem_t *sem, int pshared, unsigned int value);
int sem_destroy(sem_t *sem);
int sem_post(sem_t *sem);
int sem_wait(sem_t *sem);
int sem_trywait(sem_t *sem);
  \end{lstlisting}

  \begin{itemize}
    \item Also must link with {\tt -pthread} (or {\tt -lrt} on Solaris).
    \item All functions return {\tt 0} on success.
    \item Same usage as mutexes in terms of passing pointers.
  \end{itemize}~\\[1em]
    How could you use as {\tt semaphore} as a {\tt mutex}?
    \begin{itemize}
    \item<2-> If the initial {\tt value} is 1 and you use {\tt wait} to lock
      and {\tt post} to unlock, it's equivalent to a {\tt mutex}.
    \end{itemize}
  \end{changemargin}

\end{frame}
%%%%%%%%%%%%%%%%%%%%%%%%%%%%%%%%%%%%%%%%%%%%%%%%%%%%%%%%%%%%%%%%%%%%%%%%%%%%%%%%

%%%%%%%%%%%%%%%%%%%%%%%%%%%%%%%%%%%%%%%%%%%%%%%%%%%%%%%%%%%%%%%%%%%%%%%%%%%%%%%%
\begin{frame}[fragile]
  \frametitle{Semaphores for Signalling}

  \begin{changemargin}{1cm}
  Here's an example from the book. How would you make this always print
  ``Thread 1'' then ``Thread 2'' using semaphores?

  \begin{lstlisting}
#include <pthread.h>
#include <stdio.h>
#include <semaphore.h>
#include <stdlib.h>

void* p1 (void* arg) { printf("Thread 1\n"); }

void* p2 (void* arg) { printf("Thread 2\n"); }

int main(int argc, char *argv[])
{
    pthread_t thread[2];
    pthread_create(&thread[0], NULL, p1, NULL);
    pthread_create(&thread[1], NULL, p2, NULL);
    pthread_join(thread[0], NULL);
    pthread_join(thread[1], NULL);
    return EXIT_SUCCESS;
}
  \end{lstlisting}
\end{changemargin}
\end{frame}
%%%%%%%%%%%%%%%%%%%%%%%%%%%%%%%%%%%%%%%%%%%%%%%%%%%%%%%%%%%%%%%%%%%%%%%%%%%%%%%%

%%%%%%%%%%%%%%%%%%%%%%%%%%%%%%%%%%%%%%%%%%%%%%%%%%%%%%%%%%%%%%%%%%%%%%%%%%%%%%%%
\begin{frame}[fragile]
  \frametitle{Semaphores for Signalling}

  \begin{changemargin}{1cm}
  Here's their solution. Is it actually correct?

  \begin{lstlisting}
sem_t sem;
void* p1 (void* arg) {
  printf("Thread 1\n");
  sem_post(&sem);
}
void* p2 (void* arg) {
  sem_wait(&sem);
  printf("Thread 2\n");
}

int main(int argc, char *argv[])
{
    pthread_t thread[2];
    sem_init(&sem, 0, /* value: */ 1);
    pthread_create(&thread[0], NULL, p1, NULL);
    pthread_create(&thread[1], NULL, p2, NULL);
    pthread_join(thread[0], NULL);
    pthread_join(thread[1], NULL);
    sem_destroy(&sem);
}
  \end{lstlisting}
\end{changemargin}

\end{frame}
%%%%%%%%%%%%%%%%%%%%%%%%%%%%%%%%%%%%%%%%%%%%%%%%%%%%%%%%%%%%%%%%%%%%%%%%%%%%%%%%

%%%%%%%%%%%%%%%%%%%%%%%%%%%%%%%%%%%%%%%%%%%%%%%%%%%%%%%%%%%%%%%%%%%%%%%%%%%%%%%%
\begin{frame}[fragile]
  \frametitle{Semaphores for Signalling}

  
  \begin{changemargin}{1cm}
  \begin{enumerate}
    \item {\tt value} is initially 1.
    \vfill
    \item Say {\tt p2} hits its {\tt sem\_wait} first and succeeds.
    \vfill
    \item {\tt value} is now 0 and {\tt p2} prints ``Thread 2'' first.
  \end{enumerate}
  \begin{itemize}
    \item If {\tt p1} happens first, it would just increase
      {\tt value} to 2.\vfill
    \item<2-> Fix: set the initial {\tt value} to \alert{0}.\\[1em]
Then, if {\tt p2} hits its {\tt sem\_wait} first, it will
      not print until {\tt p1} posts (and prints ``Thread 1'') first.
  \end{itemize}~\\[1em]

  \end{changemargin}

\end{frame}
%%%%%%%%%%%%%%%%%%%%%%%%%%%%%%%%%%%%%%%%%%%%%%%%%%%%%%%%%%%%%%%%%%%%%%%%%%%%%%%%

\begin{frame}
  \frametitle{C++ atomics}

  Coming soon.
\end{frame}

%%%%%%%%%%%%%%%%%%%%%%%%%%%%%%%%%%%%%%%%%%%%%%%%%%%%%%%%%%%%%%%%%%%%%%%%%%%%%%%%
\part{Making C Compilers Work For You}
\frame{\partpage}

%%%%%%%%%%%%%%%%%%%%%%%%%%%%%%%%%%%%%%%%%%%%%%%%%%%%%%%%%%%%%%%%%%%%%%%%%%%%%%%%
\begin{frame}
  \frametitle{Three Address Code}

  \begin{changemargin}{2.5cm}
  \begin{itemize}
    \item An intermediate code used by compilers
      for analysis and optimization.
    \vfill
    \item Statements represent one fundamental operation---we
      can consider each operation \structure{atomic}.
    \vfill
    \item Statements have the form:\\
      $\qquad result := operand_1\:operator\:operand_2$
    \vfill
    \item Useful for reasoning about data races,\\ and easier to read than assembly. \\
            \hspace*{1cm} (separates out memory reads/writes).
  \end{itemize}
  \end{changemargin}
\end{frame}
%%%%%%%%%%%%%%%%%%%%%%%%%%%%%%%%%%%%%%%%%%%%%%%%%%%%%%%%%%%%%%%%%%%%%%%%%%%%%%%%

%%%%%%%%%%%%%%%%%%%%%%%%%%%%%%%%%%%%%%%%%%%%%%%%%%%%%%%%%%%%%%%%%%%%%%%%%%%%%%%%
\begin{frame}
  \frametitle{GIMPLE}

  \begin{changemargin}{2.5cm}
  \begin{itemize}
    \item GIMPLE is the three address code used by {\tt gcc}.
    \vfill
    \item To see the GIMPLE representation of your code use the
      {\tt -fdump-tree-gimple} flag.
    \vfill
    \item To see all of the three address code generated by the compiler use
      {\tt -fdump-tree-all}. You'll probably just be interested in the
      optimized version.
    \vfill
    \item Use GIMPLE to reason about your code at a low level without
      having to read assembly.
  \end{itemize}
  \end{changemargin}
\end{frame}
%%%%%%%%%%%%%%%%%%%%%%%%%%%%%%%%%%%%%%%%%%%%%%%%%%%%%%%%%%%%%%%%%%%%%%%%%%%%%%%%

%%%%%%%%%%%%%%%%%%%%%%%%%%%%%%%%%%%%%%%%%%%%%%%%%%%%%%%%%%%%%%%%%%%%%%%%%%%%%%%%
\begin{frame}
  \frametitle{Live Coding Demo: GIMPLE}
\end{frame}

%%%%%%%%%%%%%%%%%%%%%%%%%%%%%%%%%%%%%%%%%%%%%%%%%%%%%%%%%%%%%%%%%%%%%%%%%%%%%%%%
\begin{frame}[fragile]
  \frametitle{{\tt volatile} Keyword}

\begin{changemargin}{1.5cm}
  \begin{itemize}
    \item Used to notify the compiler that the variable may be changed by ``external forces''. For instance,
    %% \item Prevents the compiler from substituting values instead of evaluating
    %%   the value (with proper locking you'll never need this, why?)
  \end{itemize}

  \begin{lstlisting}
int i = 0;

while (i != 255) {
  ...
  \end{lstlisting}

{\tt volatile} prevents this from being optimized to:

  \begin{lstlisting}
int i = 0;

while (true) {
  ...
  \end{lstlisting}

  \begin{itemize}
    \item Variable will not actually be {\tt volatile} in the critical section
      and only prevents useful optimizations.
    \item Usually wrong unless there is a {\bf very} good reason for it.
  \end{itemize}
\end{changemargin}

\end{frame}
%%%%%%%%%%%%%%%%%%%%%%%%%%%%%%%%%%%%%%%%%%%%%%%%%%%%%%%%%%%%%%%%%%%%%%%%%%%%%%%%

%%%%%%%%%%%%%%%%%%%%%%%%%%%%%%%%%%%%%%%%%%%%%%%%%%%%%%%%%%%%%%%%%%%%%%%%%%%%%%%%
\begin{frame}[fragile]
  \frametitle{Branch Prediction Hints}

  \begin{changemargin}{1.5cm}
  As seen earlier in class, {\tt gcc} allows you to give branch
  prediction hints by calling this builtin function:

  \begin{center}
    \verb+long __builtin_expect (long exp, long c)+
  \end{center}

  The expected result is that {\tt exp} equals {\tt c}.\\[1em]
  Compiler reorders code \& tells CPU the prediction.
  
  \end{changemargin}
\end{frame}
%%%%%%%%%%%%%%%%%%%%%%%%%%%%%%%%%%%%%%%%%%%%%%%%%%%%%%%%%%%%%%%%%%%%%%%%%%%%%%%%

%%%%%%%%%%%%%%%%%%%%%%%%%%%%%%%%%%%%%%%%%%%%%%%%%%%%%%%%%%%%%%%%%%%%%%%%%%%%%%%%
\begin{frame}
  \frametitle{The {\tt restrict} Keyword}

   \begin{changemargin}{2.5cm}
   A new feature of C99: ``The restrict type qualifier allows programs to
      be written so that translators can produce significantly faster
      executables.''
  \begin{itemize}
    \item To request C99 in {\tt gcc}, use the {\tt -std=c99} flag.
  \end{itemize}
~\\
   {\tt restrict} means: you are promising the
      compiler that the pointer will never \structure{alias} (another pointer
      will not point to the same data) for the lifetime of the pointer.
  \end{changemargin}

\end{frame}
%%%%%%%%%%%%%%%%%%%%%%%%%%%%%%%%%%%%%%%%%%%%%%%%%%%%%%%%%%%%%%%%%%%%%%%%%%%%%%%%

%%%%%%%%%%%%%%%%%%%%%%%%%%%%%%%%%%%%%%%%%%%%%%%%%%%%%%%%%%%%%%%%%%%%%%%%%%%%%%%%
\begin{frame}[fragile]
  \frametitle{Example of {\tt restrict} (1)}

  \begin{changemargin}{2.5cm}
    Pointers declared with {\tt restrict} must
       never point to the same data.
~\\[1em]
    From Wikipedia:

  \begin{lstlisting}
void updatePtrs(int* ptrA, int* ptrB, int* val) {
    *ptrA += *val;
    *ptrB += *val;
}
  \end{lstlisting}
 Would declaring all these pointers as
      {\tt restrict} generate better code?
  \end{changemargin}

\end{frame}
%%%%%%%%%%%%%%%%%%%%%%%%%%%%%%%%%%%%%%%%%%%%%%%%%%%%%%%%%%%%%%%%%%%%%%%%%%%%%%%%

%%%%%%%%%%%%%%%%%%%%%%%%%%%%%%%%%%%%%%%%%%%%%%%%%%%%%%%%%%%%%%%%%%%%%%%%%%%%%%%%
\begin{frame}[fragile]
  \frametitle{Example of {\tt restrict} (2)}

  \begin{changemargin}{2.5cm}
    Let's look at the GIMPLE:

  \begin{lstlisting}[numbers=left]
void updatePtrs(int* ptrA, int* ptrB, int* val) {
 D.1609 = *ptrA;
 D.1610 = *val;
 D.1611 = D.1609 + D.1610;
 *ptrA = D.1611;
 D.1612 = *ptrB;
 D.1610 = *val;
 D.1613 = D.1612 + D.1610;
 *ptrB = D.1613;
}
  \end{lstlisting}

  \begin{itemize}
    \item Could any operation be left out if all the pointers
      didn't overlap?
  \end{itemize}  
  \end{changemargin}

\end{frame}
%%%%%%%%%%%%%%%%%%%%%%%%%%%%%%%%%%%%%%%%%%%%%%%%%%%%%%%%%%%%%%%%%%%%%%%%%%%%%%%%

%%%%%%%%%%%%%%%%%%%%%%%%%%%%%%%%%%%%%%%%%%%%%%%%%%%%%%%%%%%%%%%%%%%%%%%%%%%%%%%%
\begin{frame}[fragile]
  \frametitle{Example of {\tt restrict} (3)}

  \begin{changemargin}{2.5cm}
  \begin{lstlisting}[numbers=left]
void updatePtrs(int* ptrA, int* ptrB, int* val) {
 D.1609 = *ptrA;
 D.1610 = *val;
 D.1611 = D.1609 + D.1610;
 *ptrA = D.1611;
 D.1612 = *ptrB;
 D.1610 = *val;
 D.1613 = D.1612 + D.1610;
 *ptrB = D.1613;
}
  \end{lstlisting}
  \begin{itemize}
    \item If {\tt ptrA} and {\tt val} are not equal, you don't have to
      reload the data on {\bf line 7}.
    \item Otherwise, you would: there might be a call\\{\tt ~~~~updatePtrs(\&x, \&y,
      \&x);}
  \end{itemize}
  \end{changemargin}
\end{frame}
%%%%%%%%%%%%%%%%%%%%%%%%%%%%%%%%%%%%%%%%%%%%%%%%%%%%%%%%%%%%%%%%%%%%%%%%%%%%%%%%

%%%%%%%%%%%%%%%%%%%%%%%%%%%%%%%%%%%%%%%%%%%%%%%%%%%%%%%%%%%%%%%%%%%%%%%%%%%%%%%%
\begin{frame}[fragile]
  \frametitle{Example of {\tt restrict} (4)}
  \begin{changemargin}{2.5cm}
  Hence, this markup allows optimization:
  \begin{lstlisting}
void updatePtrs(int* restrict ptrA, 
                int* restrict ptrB,
                int* restrict val)
  \end{lstlisting}
  Note: you can get the optimization by just declaring {\tt ptrA} and
      {\tt val} as {\tt restrict}; {\tt ptrB} isn't needed for this optimization
  \end{changemargin}
\end{frame}
%%%%%%%%%%%%%%%%%%%%%%%%%%%%%%%%%%%%%%%%%%%%%%%%%%%%%%%%%%%%%%%%%%%%%%%%%%%%%%%%

%%%%%%%%%%%%%%%%%%%%%%%%%%%%%%%%%%%%%%%%%%%%%%%%%%%%%%%%%%%%%%%%%%%%%%%%%%%%%%%%
\begin{frame}[fragile]
  \frametitle{Summary of {\tt restrict}}

  \begin{changemargin}{2.5cm}
  \begin{itemize}
    \item Use {\tt restrict} whenever you know the pointer will not alias
      another pointer (also declared {\tt restrict})
  \end{itemize}

    It's hard for the compiler to infer pointer aliasing information;
    it's easier for you to specify it.\\[1em]

    $\Rightarrow$ compiler can better optimize your code (more perf!)\\[1em]

    Caveat: don't lie to the compiler, or  you will get
      \alert{undefined behaviour}.\\[1em]

    Aside: {\tt restrict} is not the same as {\tt const}. {\tt const} data can still be
      changed through an alias.
  \end{changemargin}
\end{frame}
%%%%%%%%%%%%%%%%%%%%%%%%%%%%%%%%%%%%%%%%%%%%%%%%%%%%%%%%%%%%%%%%%%%%%%%%%%%%%%%%

%%%%%%%%%%%%%%%%%%%%%%%%%%%%%%%%%%%%%%%%%%%%%%%%%%%%%%%%%%%%%%%%%%%%%%%%%%%%%%%%
\begin{frame}
  \frametitle{Next topic: Dependencies}

  \Large
  \begin{changemargin}{2.5cm}
     \structure{Dependencies} are the main
      limitation to parallelization.\\[1em]
     Example: computation must be evaulated as {\tt XY} and not {\tt YX}.\\[1em]
  \end{changemargin}
\end{frame}

\begin{frame}
  \frametitle{Not synchronization}

  \Large
  \begin{changemargin}{2.5cm}
      Assume (for now) no synchronization problems.\\[1em]
      Only trying to identify code that is safe to run in
      parallel.

  \end{changemargin}
\end{frame}
%%%%%%%%%%%%%%%%%%%%%%%%%%%%%%%%%%%%%%%%%%%%%%%%%%%%%%%%%%%%%%%%%%%%%%%%%%%%%%%%

\part{Memory-carried Dependencies}
\frametitle{\partpage}
%%%%%%%%%%%%%%%%%%%%%%%%%%%%%%%%%%%%%%%%%%%%%%%%%%%%%%%%%%%%%%%%%%%%%%%%%%%%%%%%
\begin{frame}[fragile]
\frametitle{Memory-carried Dependencies}
\begin{changemargin}{2.5cm}
\item Dependencies limit the amount of parallelization.
\vfill
Can we execute these 2 lines in parallel?
\begin{lstlisting}
x = 42
x = x + 1  
\end{lstlisting}
\pause
\alert{No.}\\[1em]
\begin{itemize}
\item Assume x initially 1. What are possible outcomes?
\pause \newline \structure{~~~x = 43 or x = 42}\\[1em]

\end{itemize}

Next, we'll classify dependencies.

\end{changemargin}

\end{frame}
%%%%%%%%%%%%%%%%%%%%%%%%%%%%%%%%%%%%%%%%%%%%%%%%%%%%%%%%%%%%%%%%%%%%%%%%%%%%%%%%

%%%%%%%%%%%%%%%%%%%%%%%%%%%%%%%%%%%%%%%%%%%%%%%%%%%%%%%%%%%%%%%%%%%%%%%%%%%%%%%%
\begin{frame}[fragile]
\frametitle{Read After Read (RAR)}
\begin{changemargin}{2.5cm}
Can we execute these 2 lines in parallel? (initially x is 2)
\begin{lstlisting}
y = x + 1
z = x + 5
\end{lstlisting}
\pause
\structure{Yes.}\\[1em]
\begin{itemize}
\item Variables y and z are independent.
\item Variable x is only read.
\end{itemize}

RAR dependency allows parallelization.

\end{changemargin}

\end{frame}
%%%%%%%%%%%%%%%%%%%%%%%%%%%%%%%%%%%%%%%%%%%%%%%%%%%%%%%%%%%%%%%%%%%%%%%%%%%%%%%%

%%%%%%%%%%%%%%%%%%%%%%%%%%%%%%%%%%%%%%%%%%%%%%%%%%%%%%%%%%%%%%%%%%%%%%%%%%%%%%%%
\begin{frame}[fragile]
\frametitle{Read After Write (RAW)}

\begin{changemargin}{2.5cm}
What about these 2 lines? (again, initially x is 2):
\begin{lstlisting}
x = 37
z = x + 5
\end{lstlisting}
\pause
\alert{No, z = 42 or z = 7.}\\[1em]

RAW inhibits parallelization: can't change ordering.\\
Also known as a \structure{true dependency}.
\end{changemargin}
\end{frame}
%%%%%%%%%%%%%%%%%%%%%%%%%%%%%%%%%%%%%%%%%%%%%%%%%%%%%%%%%%%%%%%%%%%%%%%%%%%%%%%%

%%%%%%%%%%%%%%%%%%%%%%%%%%%%%%%%%%%%%%%%%%%%%%%%%%%%%%%%%%%%%%%%%%%%%%%%%%%%%%%%
\begin{frame}[fragile]
\frametitle{Write After Read (WAR)}
\begin{changemargin}{2.5cm}
What if we change the order now? (again, initially x is 2)
\begin{lstlisting}
z = x + 5
x = 37
\end{lstlisting}
\pause
\alert{No. Again, z = 42 or z = 7.}\\[1em]
\begin{itemize}
\item WAR is also known as a \structure{anti-dependency}.
\item But, we can modify this code to enable parallelization.
\end{itemize}
\end{changemargin}
\end{frame}
%%%%%%%%%%%%%%%%%%%%%%%%%%%%%%%%%%%%%%%%%%%%%%%%%%%%%%%%%%%%%%%%%%%%%%%%%%%%%%%%

%%%%%%%%%%%%%%%%%%%%%%%%%%%%%%%%%%%%%%%%%%%%%%%%%%%%%%%%%%%%%%%%%%%%%%%%%%%%%%%%
\begin{frame}[fragile]
\frametitle{Removing Write After Read (WAR) Dependencies}
\begin{changemargin}{1.8cm}
Make a copy of the variable:
\begin{lstlisting}
x_copy = x
z = x_copy + 5
x = 37
\end{lstlisting}
\pause
\structure{We can now run the last 2 lines in parallel.}
\begin{itemize}
\item Induced a true dependency (RAW) between first 2 lines.
\item Isn't that bad?
\end{itemize}
\pause
Not always:
\begin{lstlisting}
z = very_long_function(x) + 5
x = very_long_calculation()
\end{lstlisting}
\end{changemargin}
\end{frame}
%%%%%%%%%%%%%%%%%%%%%%%%%%%%%%%%%%%%%%%%%%%%%%%%%%%%%%%%%%%%%%%%%%%%%%%%%%%%%%%%

%%%%%%%%%%%%%%%%%%%%%%%%%%%%%%%%%%%%%%%%%%%%%%%%%%%%%%%%%%%%%%%%%%%%%%%%%%%%%%%%
\begin{frame}[fragile]
\frametitle{Write After Write (WAW)}
\begin{changemargin}{2.5cm}
Can we run these lines in parallel? (initially x is 2)
\begin{lstlisting}
z = x + 5
z = x + 40
\end{lstlisting}
\pause
\alert{Nope, z = 42 or z = 7}.\\[1em]
\begin{itemize}
\item WAW is also known as an \structure{output dependency}.
\item We can remove this dependency (like WAR):
\end{itemize}
\pause
\begin{lstlisting}
z_copy = x + 5
z = x + 40
\end{lstlisting}
\end{changemargin}
\end{frame}
%%%%%%%%%%%%%%%%%%%%%%%%%%%%%%%%%%%%%%%%%%%%%%%%%%%%%%%%%%%%%%%%%%%%%%%%%%%%%%%%

%%%%%%%%%%%%%%%%%%%%%%%%%%%%%%%%%%%%%%%%%%%%%%%%%%%%%%%%%%%%%%%%%%%%%%%%%%%%%%%%
\begin{frame}
\frametitle{Summary of Memory-carried Dependencies}
\begin{center}
\begin{tabular}{ll|p{2.8cm}p{3.2cm}}
& & \multicolumn{2}{c}{Second Access} \\ 
&  & \bf Read & \bf Write \\ \hline
\multirow{2}{*}{First Access} & \bf Read & No Dependency Read After Read (RAR)  & Anti-dependency Write After Read (WAR) \\[0.5em]
& \bf Write & True Dependency Read After Write (RAW) & Output Dependency Write After Write (WAW) \\
\end{tabular}
\end{center}
\end{frame}
%%%%%%%%%%%%%%%%%%%%%%%%%%%%%%%%%%%%%%%%%%%%%%%%%%%%%%%%%%%%%%%%%%%%%%%%%%%%%%%%

\end{document}
