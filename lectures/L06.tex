\documentclass[11pt]{article}
\usepackage{listings}
\usepackage{tikz}
\usepackage{url}
%\usepackage{algorithm2e}
\usetikzlibrary{arrows,automata,shapes,positioning}
\tikzstyle{block} = [rectangle, draw, fill=blue!20, 
    text width=2.5em, text centered, rounded corners, minimum height=2em]
\tikzstyle{bw} = [rectangle, draw, fill=blue!20, 
    text width=3.5em, text centered, rounded corners, minimum height=2em]

\newcommand{\handout}[5]{
  \noindent
  \begin{center}
  \framebox{
    \vbox{
      \hbox to 5.78in { {\bf ECE459: Programming for Performance } \hfill #2 }
      \vspace{4mm}
      \hbox to 5.78in { {\Large \hfill #5  \hfill} }
      \vspace{2mm}
      \hbox to 5.78in { {\em #3 \hfill #4} }
    }
  }
  \end{center}
  \vspace*{4mm}
}

\lstset{basicstyle=\ttfamily \scriptsize}

\newcommand{\lecture}[4]{\handout{#1}{#2}{#3}{#4}{Lecture #1}}
\topmargin 0pt
\advance \topmargin by -\headheight
\advance \topmargin by -\headsep
\textheight 8.9in
\oddsidemargin 0pt
\evensidemargin \oddsidemargin
\marginparwidth 0.5in
\textwidth 6.5in

\parindent 0in
\parskip 1.5ex
%\renewcommand{\baselinestretch}{1.25}

\begin{document}

\lecture{6 --- January 15, 2015}{Winter 2015}{Patrick Lam}{version 0}

\section*{Synchronization}
You'll need some sort of synchronization to get sane results from
multithreaded programs. We'll start by talking about how to use
mutual exclusion in Pthreads.

\paragraph{Mutual Exclusion.} Mutexes are the most basic type of synchronization.
As a reminder:
    \begin{itemize}
    \item Only one thread can access code protected by a mutex at a time.
    \item All other threads must wait until the mutex is free before they can
      execute the protected code.
    \end{itemize}

    Here's two examples of using mutexes:
    {\small
  \begin{minipage}{.55\textwidth}
  \begin{lstlisting}
pthread_mutex_t m1_static = PTHREAD_MUTEX_INITIALIZER;
pthread_mutex_t m2_dynamic;

pthread_mutex_init(&m2_dynamic, NULL);
...
pthread_mutex_destroy(&m1_static);
pthread_mutex_destroy(&m2_dynamic);
  \end{lstlisting}
  \end{minipage}
  \begin{minipage}{.4\textwidth}
    \begin{lstlisting}[language=C++]
#include <mutex>

std::mutex m1, m2;
\end{lstlisting}
  \end{minipage}
}


You can initialize Pthreads mutexes statically (as with {\tt m1\_static}) or
dynamically ({\tt m2\_dynamic}). If you want to include attributes,
you need to use the dynamic version. C++11 mutexes don't need to be explicitly
destroyed; resources are freed when they go out of scope. They don't seem to have
attributes.

\paragraph{Mutex Attributes.} Both threads and mutexes use the notion of attributes.
We won't talk about mutex attributes in any detail, but here are the three standard Pthreads ones.
  \begin{itemize}
    \item {\bf Protocol}: specifies the protocol used to prevent priority
      inversions for a mutex.
    \item {\bf Prioceiling}: specifies the priority ceiling of a mutex.
    \item {\bf Process-shared}: specifies the process sharing of a mutex.
  \end{itemize}
  You can specify a mutex as {\it process shared} so that you can access it
  between processes. In that case, you need to use shared memory and {\tt mmap},
  which we won't get into.

\paragraph{Mutex Example.} Let's see how this looks in practice. It is fairly simple:
    {\small
  \begin{minipage}{.55\textwidth}
  \begin{lstlisting}
// code
pthread_mutex_lock(&m1);
// protected code
pthread_mutex_unlock(&m1);
// more code
  \end{lstlisting}
  \end{minipage}
  \begin{minipage}{.4\textwidth}
    \begin{lstlisting}[language=C++]
// code
m1.lock();
// protected code
m1.unlock();
// more code
    \end{lstlisting}
  \end{minipage}
  }
  \begin{itemize}
    \item Everything within the {\tt lock} and {\tt unlock} is protected.
    \item Be careful to avoid deadlocks if you are using multiple mutexes (always
acquire locks in the same order across threads).
    \item Another useful primitive is {\tt pthread\_mutex\_trylock}, also known as {\tt ::try\_lock()} in C++11 threads. We may come back to this
later.
  \end{itemize}

\subsection*{Data Races}
Why are we bothering with locks? Data races. A data race occurs when
two concurrent actions access the same variable and at least one of
them is a {\bf write}. (This shows up on Assignment 1!)

  \begin{lstlisting}
...
static int counter = 0;

void* run(void* arg) {
    for (int i = 0; i < 100; ++i) {
        ++counter;
    }
}

int main(int argc, char *argv[])
{
    // Create 8 threads
    // Join 8 threads
    printf("counter = %i\n", counter);
}
  \end{lstlisting}

Is there a datarace in this example? If so, how would we fix it?

\paragraph{Solution: use mutexes.}
  \begin{lstlisting}[escapechar=!]
...
!\bf{static pthread\_mutex\_t mutex = PTHREAD\_MUTEX\_INITIALIZER;}!
static int counter = 0;

void* run(void* arg) {
    for (int i = 0; i < 100; ++i) {
        !\bf{pthread\_mutex\_lock(\&mutex);}!
        ++counter;
        !\bf{pthread\_mutex\_unlock(\&mutex);}!
    }
}

int main(int argc, char *argv[])
{
    // Create 8 threads
    // Join 8 threads
    !\bf{pthread\_mutex\_destroy(\&mutex);}!
    printf("counter = %i\n", counter);
}
  \end{lstlisting}
(I'll leave expressing this in C++11 as an exercise to the reader.)

\end{document}
