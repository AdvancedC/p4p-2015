\documentclass[aspectratio=43]{beamer}

% Text packages to stop warnings
\usepackage{lmodern}
\usepackage{textcomp}
\usepackage{listings}

% Themes
\usetheme{Boadilla}
\setbeamertemplate{footline}[page number]{}
\setbeamertemplate{navigation symbols}{}

% Suppress the navigation bar
\beamertemplatenavigationsymbolsempty

\lstset{basicstyle=\scriptsize, frame=single}

\newenvironment{changemargin}[1]{% 
  \begin{list}{}{% 
    \setlength{\topsep}{0pt}% 
    \setlength{\leftmargin}{#1}% 
    \setlength{\rightmargin}{1em}
    \setlength{\listparindent}{\parindent}% 
    \setlength{\itemindent}{\parindent}% 
    \setlength{\parsep}{\parskip}% 
  }% 
  \item[]}{\end{list}} 

\title{Lecture 07---{\tt epoll}, async I/O, curl\_multi}
\date{January 19, 2015}

\begin{document}

%%%%%%%%%%%%%%%%%%%%%%%%%%%%%%%%%%%%%%%%%%%%%%%%%%%%%%%%%%%%%%%%%%%%%%%%%%%%%%%%
\begin{frame}[plain]
  \titlepage
\end{frame}
%%%%%%%%%%%%%%%%%%%%%%%%%%%%%%%%%%%%%%%%%%%%%%%%%%%%%%%%%%%%%%%%%%%%%%%%%%%%%%%%

%%%%%%%%%%%%%%%%%%%%%%%%%%%%%%%%%%%%%%%%%%%%%%%%%%%%%%%%%%%%%%%%%%%%%%%%%%%%%%%%
\begin{frame}
  \frametitle{Roadmap}

  \Large
    \begin{changemargin}{2cm}
  Past: Modern Hardware, Threads; \\[1em]

  Now: Non-blocking I/O;\\[1em]

  Next: Race Conditions, Locking.
    \end{changemargin}
  
\end{frame}
%%%%%%%%%%%%%%%%%%%%%%%%%%%%%%%%%%%%%%%%%%%%%%%%%%%%%%%%%%%%%%%%%%%%%%%%%%%%%%%%

%%%%%%%%%%%%%%%%%%%%%%%%%%%%%%%%%%%%%%%%%%%%%%%%%%%%%%%%%%%%%%%%%%%%%%%%%%%%%%%%
\begin{frame}
  \frametitle{Last Time}

  \begin{changemargin}{2.5cm}
  \begin{itemize}
    \item Assignment 1 walkthrough.
    \item Concept behind non-blocking I/O.
  \end{itemize}
  \end{changemargin}
\end{frame}
%%%%%%%%%%%%%%%%%%%%%%%%%%%%%%%%%%%%%%%%%%%%%%%%%%%%%%%%%%%%%%%%%%%%%%%%%%%%%%%%

\part{Async I/O with epoll}
\frame{\partpage}
%%%%%%%%%%%%%%%%%%%%%%%%%%%%%%%%%%%%%%%%%%%%%%%%%%%%%%%%%%%%%%%%%%%%%%%%%%%%%%%%
\begin{frame}
  \frametitle{Using {\tt epoll}}
  \begin{changemargin}{2em}
    Key idea: give {\tt epoll} a bunch of file descriptors;\\
     \hspace*{2em}wait for events to happen.\\[1em]

     Steps:
     \begin{enumerate}
       \item create an instance ({\tt epoll\_create1});
       \item populate it with file descriptors ({\tt epoll\_ctl});
       \item wait for events ({\tt epoll\_wait}).
     \end{enumerate}
  \end{changemargin}
\end{frame}
%%%%%%%%%%%%%%%%%%%%%%%%%%%%%%%%%%%%%%%%%%%%%%%%%%%%%%%%%%%%%%%%%%%%%%%%%%%%%%%%

%%%%%%%%%%%%%%%%%%%%%%%%%%%%%%%%%%%%%%%%%%%%%%%%%%%%%%%%%%%%%%%%%%%%%%%%%%%%%%%%
\begin{frame}[fragile]
  \frametitle{Creating an {\tt epoll} instance}
  \begin{changemargin}{2em}
    \begin{minipage}{.5\textwidth}
    \begin{lstlisting}
   int epfd = epoll_create1(0);
    \end{lstlisting}
    \end{minipage}

    {\tt efpd} doesn't represent any files; use it to talk to {\tt epoll}.\\[1em]

    0 represents the flags (only flag: {\tt EPOLL\_CLOEXEC}).
    
  \end{changemargin}
\end{frame}
%%%%%%%%%%%%%%%%%%%%%%%%%%%%%%%%%%%%%%%%%%%%%%%%%%%%%%%%%%%%%%%%%%%%%%%%%%%%%%%%

%%%%%%%%%%%%%%%%%%%%%%%%%%%%%%%%%%%%%%%%%%%%%%%%%%%%%%%%%%%%%%%%%%%%%%%%%%%%%%%%
\begin{frame}[fragile]
  \frametitle{Populating the {\tt epoll} instance}
  \begin{changemargin}{2em}
    To add {\tt fd} to the set of descriptors watched by {\tt epfd}:
    \begin{lstlisting}
   struct epoll_event event;
   int ret;
   event.data.fd = fd;
   event.events = EPOLLIN | EPOLLOUT;
   ret = epoll_ctl(epfd, EPOLL_CTL_ADD, fd, &event);
    \end{lstlisting}

    Can also modify and delete descriptors from {\tt epfd}.
  \end{changemargin}
\end{frame}
%%%%%%%%%%%%%%%%%%%%%%%%%%%%%%%%%%%%%%%%%%%%%%%%%%%%%%%%%%%%%%%%%%%%%%%%%%%%%%%%

%%%%%%%%%%%%%%%%%%%%%%%%%%%%%%%%%%%%%%%%%%%%%%%%%%%%%%%%%%%%%%%%%%%%%%%%%%%%%%%%
\begin{frame}[fragile]
  \frametitle{Waiting on an {\tt epoll} instance}
  \begin{changemargin}{2em}
    Now we're ready to wait for events on any file descriptor in {\tt epfd}.
    \begin{lstlisting}
  #define MAX_EVENTS 64

  struct epoll_event events[MAX_EVENTS];
  int nr_events;

  nr_events = epoll_wait(epfd, events, MAX_EVENTS, -1);
    \end{lstlisting}

-1: wait potentially forever; otherwise, milliseconds to wait.\\[1em]

Upon return from {\tt epoll\_wait}, we have {\tt nr\_events} events ready.

  \end{changemargin}
\end{frame}
%%%%%%%%%%%%%%%%%%%%%%%%%%%%%%%%%%%%%%%%%%%%%%%%%%%%%%%%%%%%%%%%%%%%%%%%%%%%%%%%

%%%%%%%%%%%%%%%%%%%%%%%%%%%%%%%%%%%%%%%%%%%%%%%%%%%%%%%%%%%%%%%%%%%%%%%%%%%%%%%%
\begin{frame}
  \frametitle{Level-Triggered and Edge-Triggered Events}
  \begin{changemargin}{2em}
    Default {\tt epoll} behaviour is \structure{level-triggered}:\\
    \quad return whenever data is ready.\\[1em]

    Can also specify (via {\tt epoll\_ctl}) \structure{edge-triggered} behaviour:\\
    \quad return whenever there is a change in readiness.\\[1em]
  \end{changemargin}
\end{frame}
%%%%%%%%%%%%%%%%%%%%%%%%%%%%%%%%%%%%%%%%%%%%%%%%%%%%%%%%%%%%%%%%%%%%%%%%%%%%%%%%

%%%%%%%%%%%%%%%%%%%%%%%%%%%%%%%%%%%%%%%%%%%%%%%%%%%%%%%%%%%%%%%%%%%%%%%%%%%%%%%%
\begin{frame}
  \frametitle{Live Coding: Level-Triggered vs Edge-Triggered}
\end{frame}
%%%%%%%%%%%%%%%%%%%%%%%%%%%%%%%%%%%%%%%%%%%%%%%%%%%%%%%%%%%%%%%%%%%%%%%%%%%%%%%%

%%%%%%%%%%%%%%%%%%%%%%%%%%%%%%%%%%%%%%%%%%%%%%%%%%%%%%%%%%%%%%%%%%%%%%%%%%%%%%%%
\begin{frame}
  \frametitle{Asynchronous I/O}
  \begin{changemargin}{2em}
    POSIX standard defines \structure{aio} calls.\\[1em]

    These work for disk as well as sockets.\\[1em]

    Key idea: you specify the action to occur when I/O is ready:
    \begin{itemize}
      \item nothing;
      \item start a new thread;
      \item raise a signal
    \end{itemize}

    Submit the requests using e.g. {\tt aio\_read} and {\tt aio\_write}.\\[1em]
    Can wait for I/O to happen using {\tt aio\_suspend}.
  \end{changemargin}
\end{frame}
%%%%%%%%%%%%%%%%%%%%%%%%%%%%%%%%%%%%%%%%%%%%%%%%%%%%%%%%%%%%%%%%%%%%%%%%%%%%%%%%

%%%%%%%%%%%%%%%%%%%%%%%%%%%%%%%%%%%%%%%%%%%%%%%%%%%%%%%%%%%%%%%%%%%%%%%%%%%%%%%%
\begin{frame}
  \frametitle{Nonblocking I/O with curl}
  \begin{changemargin}{2em}
    Similar idea to {\tt epoll}:
\begin{itemize}
\item build up a set of descriptors;
\item invoke the transfers and wait for them to finish;
\item see how things went.
\end{itemize}
  \end{changemargin}
\end{frame}
%%%%%%%%%%%%%%%%%%%%%%%%%%%%%%%%%%%%%%%%%%%%%%%%%%%%%%%%%%%%%%%%%%%%%%%%%%%%%%%%

\part{Using curl\_multi}
\frame{\partpage}

%%%%%%%%%%%%%%%%%%%%%%%%%%%%%%%%%%%%%%%%%%%%%%%%%%%%%%%%%%%%%%%%%%%%%%%%%%%%%%%%
\begin{frame}
  \frametitle{curl\_multi initialization}

  \begin{changemargin}{2cm}
    curl\_multi: work with multiple resources at once.\\[1em]

    How? Similar idea to {\tt epoll}:\\[1em]

    1. To use {\tt curl\_multi}, first create the individual requests \\ \hspace{2em} ({\tt curl\_easy\_init}).\\
    ~~~(Set options as needed on each handle).\\[1em]

    2. Then, combine them with:
    \begin{itemize}
      \item {\tt curl\_multi\_init()};
      \item {\tt curl\_multi\_add\_handle()}.
    \end{itemize}

  \end{changemargin}

\end{frame}
%%%%%%%%%%%%%%%%%%%%%%%%%%%%%%%%%%%%%%%%%%%%%%%%%%%%%%%%%%%%%%%%%%%%%%%%%%%%%%%%

%%%%%%%%%%%%%%%%%%%%%%%%%%%%%%%%%%%%%%%%%%%%%%%%%%%%%%%%%%%%%%%%%%%%%%%%%%%%%%%%
\begin{frame}
  \frametitle{curl\_multi\_perform: option 1, select-based interface}

  \begin{changemargin}{1.5cm}
    Main idea: put in requests and wait for results.\\[1em]

    {\tt curl\_multi\_perform} is a generalization of {\tt curl\_easy\_perform} to multiple resources.\\[1em]

    Handle completed transfers with {\tt curl\_multi\_info\_read}.

  \end{changemargin}

\end{frame}
%%%%%%%%%%%%%%%%%%%%%%%%%%%%%%%%%%%%%%%%%%%%%%%%%%%%%%%%%%%%%%%%%%%%%%%%%%%%%%%%

%%%%%%%%%%%%%%%%%%%%%%%%%%%%%%%%%%%%%%%%%%%%%%%%%%%%%%%%%%%%%%%%%%%%%%%%%%%%%%%%
\begin{frame}
  \frametitle{calling curl\_multi\_perform}

  \begin{changemargin}{2cm}
    perform interface requires use of {\tt select} (not {\tt epoll}).\\[1em]

    usage (once you've {\tt curl\_multi\_add\_handle}'d):\\[.5em]
    \hspace*{-2em} {\tt curl\_multi\_perform(multi\_handle, \&still\_running)}\\[.5em]
    performs a non-blocking read/write, and\\
    returns the number of still-active handles\\ \hspace*{2em} (with more data to come).

  \end{changemargin}

\end{frame}
%%%%%%%%%%%%%%%%%%%%%%%%%%%%%%%%%%%%%%%%%%%%%%%%%%%%%%%%%%%%%%%%%%%%%%%%%%%%%%%%

%%%%%%%%%%%%%%%%%%%%%%%%%%%%%%%%%%%%%%%%%%%%%%%%%%%%%%%%%%%%%%%%%%%%%%%%%%%%%%%%
\begin{frame}
  \frametitle{Next steps after curl\_multi\_perform}

  \begin{changemargin}{2cm}
    do
    \begin{itemize}
      \item organize a call to {\tt select}; and
      \item call {\tt curl\_multi\_perform} again
    \end{itemize}
    while there are still running transfers.\\[1em]

    After the {\tt curl\_multi\_perform}, you can also delete, alter,
    and re-add an {\tt curl\_easy\_handle} when a transfer finishes.
  \end{changemargin}

\end{frame}
%%%%%%%%%%%%%%%%%%%%%%%%%%%%%%%%%%%%%%%%%%%%%%%%%%%%%%%%%%%%%%%%%%%%%%%%%%%%%%%%


%%%%%%%%%%%%%%%%%%%%%%%%%%%%%%%%%%%%%%%%%%%%%%%%%%%%%%%%%%%%%%%%%%%%%%%%%%%%%%%%
\begin{frame}[fragile]
  \frametitle{Before calling {\tt select}}

  \begin{changemargin}{2cm}
    {\tt select} needs a timeout and an {\tt fdset}. \\
    \qquad ({\tt curl} provides both.)\\[1em]

    Initializing the {\tt fdset} from the {\tt multi\_handle}:
  \end{changemargin}

\begin{lstlisting}
    // zero the fd-sets
    FD_ZERO(&fdread); FD_ZERO(&fdwrite); FD_ZERO(&fdexcep);
    // retrieve the fds, check for error
    curl_multi_fdset(multi_handle, 
                     &fdread, &fdwrite, &fdexcep, &maxfd);
    if (maxfd < -1) abort_("multi_fdset: couldn't wait for fds");
\end{lstlisting}

  \begin{changemargin}{2cm}
    Retrieving the proper timeout:
  \end{changemargin}
\begin{lstlisting}
    curl_multi_timeout(multi_handle, &curl_timeout);
\end{lstlisting}
  \begin{changemargin}{2cm}
    (and then convert the {\tt long} to a {\tt struct timeval}).
  \end{changemargin}

\end{frame}
%%%%%%%%%%%%%%%%%%%%%%%%%%%%%%%%%%%%%%%%%%%%%%%%%%%%%%%%%%%%%%%%%%%%%%%%%%%%%%%%

%%%%%%%%%%%%%%%%%%%%%%%%%%%%%%%%%%%%%%%%%%%%%%%%%%%%%%%%%%%%%%%%%%%%%%%%%%%%%%%%
\begin{frame}[fragile]
  \frametitle{The call to {\tt select}}

\begin{lstlisting}
  rc = select(maxfd + 1, &fdread, &fdwrite, &fdexcep, &timeout);
  if (rc == -1) abort_("[main] select error");
\end{lstlisting}

  \begin{changemargin}{2cm}
    Wait for one of the fds to become ready,\\ ~~or for timeout to elapse. \\[1em]
    What next?\\
    \only<2>{Call {\tt curl\_multi\_perform} again to do the work.}
  \end{changemargin}
\end{frame}
%%%%%%%%%%%%%%%%%%%%%%%%%%%%%%%%%%%%%%%%%%%%%%%%%%%%%%%%%%%%%%%%%%%%%%%%%%%%%%%%

%%%%%%%%%%%%%%%%%%%%%%%%%%%%%%%%%%%%%%%%%%%%%%%%%%%%%%%%%%%%%%%%%%%%%%%%%%%%%%%%
\begin{frame}[fragile]
  \frametitle{Knowing what happened after {\tt curl\_multi\_perform}}

  \begin{changemargin}{2cm}
    {\tt curl\_multi\_info\_read} will tell you.
  \end{changemargin}
\begin{lstlisting}
  msg = curl_multi_info_read(multi_handle, &msgs_left);
\end{lstlisting}
  \begin{changemargin}{2cm}
    and also how many messages are left.\\[1em]
    {\tt msg->msg} can be {\tt CURLMSG\_DONE} or an error;\\
    {\tt msg->easy\_handle} tells you who is done.\\[1em]
  \end{changemargin}
  \begin{changemargin}{1.25cm}
    Some gotchas (thanks Desiye Collier):
    \begin{itemize}
    \item Checking \verb+msg->msg == CURLMSG_DONE+ is not sufficient to ensure that a curl request actually happened. You also need to check {\tt data.result}.

    \item (A1 hint:) To reset an individual handle in the {\tt multi\_handle}, you need to ``replace'' it. But you shouldn't use {\tt curl\_easy\_init()}.  In fact, you don't need a new handle at all.
    \end{itemize}
  \end{changemargin}

\end{frame}
%%%%%%%%%%%%%%%%%%%%%%%%%%%%%%%%%%%%%%%%%%%%%%%%%%%%%%%%%%%%%%%%%%%%%%%%%%%%%%%%



%%%%%%%%%%%%%%%%%%%%%%%%%%%%%%%%%%%%%%%%%%%%%%%%%%%%%%%%%%%%%%%%%%%%%%%%%%%%%%%%
\begin{frame}
  \frametitle{curl\_multi cleanup}

  \begin{changemargin}{2cm}
    Call {\tt curl\_multi\_cleanup} on the multi handle.\\[1em]
    Then, call {\tt curl\_easy\_cleanup} on each easy handle.\\[2em]
    If you replace {\tt curl\_easy\_init} by {\tt curl\_global\_init},\\
    then call {\tt curl\_global\_cleanup} also.
  \end{changemargin}

\end{frame}
%%%%%%%%%%%%%%%%%%%%%%%%%%%%%%%%%%%%%%%%%%%%%%%%%%%%%%%%%%%%%%%%%%%%%%%%%%%%%%%%


%%%%%%%%%%%%%%%%%%%%%%%%%%%%%%%%%%%%%%%%%%%%%%%%%%%%%%%%%%%%%%%%%%%%%%%%%%%%%%%%
\begin{frame}
  \frametitle{curl\_multi\_perform example}

  \begin{changemargin}{1.5cm}
    Not a great example:

\begin{center}
\url{http://curl.haxx.se/libcurl/c/multi-app.html}
\end{center}

    I'm not even sure it works verbatim.\\[1em]

    Nevertheless, you could use it as a solution template.\\
    You'll have to add more code to replace completed transfers.
  \end{changemargin}

\end{frame}
%%%%%%%%%%%%%%%%%%%%%%%%%%%%%%%%%%%%%%%%%%%%%%%%%%%%%%%%%%%%%%%%%%%%%%%%%%%%%%%%

%%%%%%%%%%%%%%%%%%%%%%%%%%%%%%%%%%%%%%%%%%%%%%%%%%%%%%%%%%%%%%%%%%%%%%%%%%%%%%%%
\begin{frame}
  \frametitle{A better choice: curl\_multi\_wait}

  \begin{changemargin}{1.5cm}
    Instead of using {\tt select()}, \\
    you can use {\tt curl\_multi\_wait()}.\\[2em]
    It's just better.\\
  \url{https://gist.github.com/clemensg/4960504}
  \end{changemargin}
\end{frame}
%%%%%%%%%%%%%%%%%%%%%%%%%%%%%%%%%%%%%%%%%%%%%%%%%%%%%%%%%%%%%%%%%%%%%%%%%%%%%%%%

%%%%%%%%%%%%%%%%%%%%%%%%%%%%%%%%%%%%%%%%%%%%%%%%%%%%%%%%%%%%%%%%%%%%%%%%%%%%%%%%
\begin{frame}
  \frametitle{curl\_multi, option 3: curl\_multi\_socket\_action}

  \begin{changemargin}{1.5cm}
    So, I couldn't quite figure out how this works. Sorry.\\[1em]

    Similar to the {\tt perform} interface, but you have more control.

    Advantage:

\begin{quote}
   2 - When the application discovers action on a single socket, it calls
       libcurl and informs that there was action on this particular socket and
       libcurl can then act on that socket/transfer only and not care about
       any other transfers. (The previous API always had to scan through all
       the existing transfers.)
\end{quote}

\url{http://curl.haxx.se/dev/readme-multi_socket.html}

  \end{changemargin}
\end{frame}
%%%%%%%%%%%%%%%%%%%%%%%%%%%%%%%%%%%%%%%%%%%%%%%%%%%%%%%%%%%%%%%%%%%%%%%%%%%%%%%%

%%%%%%%%%%%%%%%%%%%%%%%%%%%%%%%%%%%%%%%%%%%%%%%%%%%%%%%%%%%%%%%%%%%%%%%%%%%%%%%%
\begin{frame}

  \frametitle{multi\_socket usage}

\begin{changemargin}{1cm}
From the manpage:
\small
\begin{itemize}
\item Create a multi handle

\item Set the socket callback with {\tt CURLMOPT\_SOCKETFUNCTION}

\item Set the timeout callback with {\tt CURLMOPT\_TIMERFUNCTION}, to get to know what timeout value to use when waiting for socket activities.

\item Add easy handles with {\tt curl\_multi\_add\_handle()}

\item Provide some means to manage the sockets libcurl is using, so you can check them for activity. This can be done through your application code, or by way of an external library such as libevent or glib.

\item Call {\tt curl\_multi\_socket\_action(..., CURL\_SOCKET\_TIMEOUT, 0, ...)} to kickstart everything. To get one or more callbacks called.

\item Wait for activity on any of libcurl's sockets, use the timeout value your callback has been told.

\item When activity is detected, call {\tt curl\_multi\_socket\_action()} for the socket(s) that got action. If no activity is detected and the timeout expires, call {\tt curl\_multi\_socket\_action(3)} with {\tt CURL\_SOCKET\_TIMEOUT}. 
\end{itemize}
  \end{changemargin}

\end{frame}
%%%%%%%%%%%%%%%%%%%%%%%%%%%%%%%%%%%%%%%%%%%%%%%%%%%%%%%%%%%%%%%%%%%%%%%%%%%%%%%%

%%%%%%%%%%%%%%%%%%%%%%%%%%%%%%%%%%%%%%%%%%%%%%%%%%%%%%%%%%%%%%%%%%%%%%%%%%%%%%%%
\begin{frame}

  \frametitle{multi\_socket example}

\begin{changemargin}{2cm}
This example is even worse than the last one:
\begin{center}
\url{http://curl.haxx.se/libcurl/c/hiperfifo.html}
\end{center}~\\[1em]
It contains more moving parts than we need to understand the API, and gets
another library ({\tt libevent}) involved.
\end{changemargin}
\end{frame}
%%%%%%%%%%%%%%%%%%%%%%%%%%%%%%%%%%%%%%%%%%%%%%%%%%%%%%%%%%%%%%%%%%%%%%%%%%%%%%%%

\end{document}
