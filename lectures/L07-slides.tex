\documentclass[aspectratio=43]{beamer}

% Text packages to stop warnings
\usepackage{lmodern}
\usepackage{textcomp}
\usepackage{tikz}
\usepackage{listings}

% Themes
\usetheme{Boadilla}
\setbeamertemplate{footline}[page number]{}
\setbeamertemplate{navigation symbols}{}

% Suppress the navigation bar
\beamertemplatenavigationsymbolsempty

\lstset{basicstyle=\scriptsize,frame=single}

\newenvironment{changemargin}[1]{% 
  \begin{list}{}{% 
    \setlength{\topsep}{0pt}% 
    \setlength{\leftmargin}{#1}% 
    \setlength{\rightmargin}{1em}
    \setlength{\listparindent}{\parindent}% 
    \setlength{\itemindent}{\parindent}% 
    \setlength{\parsep}{\parskip}% 
  }% 
  \item[]}{\end{list}} 

\title{Lecture 06---More on Threads}
\subtitle{ECE 459: Programming for Performance}
\author{Patrick Lam}
\institute{University of Waterloo}
\date{January 16, 2015}

\begin{document}

%%%%%%%%%%%%%%%%%%%%%%%%%%%%%%%%%%%%%%%%%%%%%%%%%%%%%%%%%%%%%%%%%%%%%%%%%%%%%%%%
\begin{frame}[plain]
  \titlepage
\end{frame}
%%%%%%%%%%%%%%%%%%%%%%%%%%%%%%%%%%%%%%%%%%%%%%%%%%%%%%%%%%%%%%%%%%%%%%%%%%%%%%%%

\section{Synchronization}
%%%%%%%%%%%%%%%%%%%%%%%%%%%%%%%%%%%%%%%%%%%%%%%%%%%%%%%%%%%%%%%%%%%%%%%%%%%%%%%%
\begin{frame}
  \frametitle{Mutual Exclusion}

\begin{changemargin}{1.5cm}
    Mutexes are the most basic type of synchronization.
    \vfill
    \begin{itemize}
    \item Only one thread can access code protected by a mutex at a time.
    \vfill
    \item All other threads must wait until the mutex is free before they can
      execute the protected code.
    \end{itemize}
\end{changemargin}

\end{frame}
%%%%%%%%%%%%%%%%%%%%%%%%%%%%%%%%%%%%%%%%%%%%%%%%%%%%%%%%%%%%%%%%%%%%%%%%%%%%%%%%

%%%%%%%%%%%%%%%%%%%%%%%%%%%%%%%%%%%%%%%%%%%%%%%%%%%%%%%%%%%%%%%%%%%%%%%%%%%%%%%%
\begin{frame}
  \frametitle{Live Coding Example: Mutual Exclusion}
\end{frame}
%%%%%%%%%%%%%%%%%%%%%%%%%%%%%%%%%%%%%%%%%%%%%%%%%%%%%%%%%%%%%%%%%%%%%%%%%%%%%%%%

%%%%%%%%%%%%%%%%%%%%%%%%%%%%%%%%%%%%%%%%%%%%%%%%%%%%%%%%%%%%%%%%%%%%%%%%%%%%%%%%
\begin{frame}[fragile]
  \frametitle{Creating Mutexes---Pthreads Example}

\begin{changemargin}{1.5cm}
  \begin{lstlisting}
pthread_mutex_t m1 = PTHREAD_MUTEX_INITIALIZER;
pthread_mutex_t m2;

pthread_mutex_init(&m2, NULL);
...
pthread_mutex_destroy(&m1);
pthread_mutex_destroy(&m2);
  \end{lstlisting}

  \begin{itemize}
    \item Two ways to initialize mutexes: statically and dynamically
    \item If you want to include attributes, you need to use the dynamic version
  \end{itemize}
\end{changemargin}

\end{frame}
%%%%%%%%%%%%%%%%%%%%%%%%%%%%%%%%%%%%%%%%%%%%%%%%%%%%%%%%%%%%%%%%%%%%%%%%%%%%%%%%

%%%%%%%%%%%%%%%%%%%%%%%%%%%%%%%%%%%%%%%%%%%%%%%%%%%%%%%%%%%%%%%%%%%%%%%%%%%%%%%%
\begin{frame}[fragile]
  \frametitle{Creating Mutexes---C++11 Example}

\begin{changemargin}{1.5cm}
  \begin{lstlisting}
#include <mutex>

std::mutex m1, m2;
  \end{lstlisting}

  Resources released when objects go out of scope.
\end{changemargin}

\end{frame}
%%%%%%%%%%%%%%%%%%%%%%%%%%%%%%%%%%%%%%%%%%%%%%%%%%%%%%%%%%%%%%%%%%%%%%%%%%%%%%%%

%%%%%%%%%%%%%%%%%%%%%%%%%%%%%%%%%%%%%%%%%%%%%%%%%%%%%%%%%%%%%%%%%%%%%%%%%%%%%%%%
\begin{frame}
  \frametitle{Pthreads Mutex Attributes}

\begin{changemargin}{1.5cm}
  \begin{itemize}
    \item {\bf Protocol}: specifies the protocol used to prevent priority
      inversions for a mutex
    \item {\bf Prioceiling}: specifies the priority ceiling of a mutex
    \item {\bf Process-shared}: specifies the process sharing of a mutex
  \end{itemize}
  You can specify a mutex as {\it process shared} so that you can access it
  between processes. In that case, you need to use shared memory and {\tt mmap},
  which we won't get into.
\end{changemargin}

\end{frame}
%%%%%%%%%%%%%%%%%%%%%%%%%%%%%%%%%%%%%%%%%%%%%%%%%%%%%%%%%%%%%%%%%%%%%%%%%%%%%%%%

%%%%%%%%%%%%%%%%%%%%%%%%%%%%%%%%%%%%%%%%%%%%%%%%%%%%%%%%%%%%%%%%%%%%%%%%%%%%%%%%
\begin{frame}[fragile]
  \frametitle{Using Pthreads Mutexes: Example}

\begin{changemargin}{1.5cm}
  \begin{lstlisting}
// code
pthread_mutex_lock(&m1);
// protected code
pthread_mutex_unlock(&m1);
// more code
  \end{lstlisting}
  \vfill
  \begin{itemize}
    \item Everything within the {\tt lock} and {\tt unlock} is protected.
    \item Be careful to avoid deadlocks if you are using multiple mutexes.
    \item Also you can use {\tt pthread\_mutex\_trylock}, if needed.
  \end{itemize}
\end{changemargin}
\end{frame}
%%%%%%%%%%%%%%%%%%%%%%%%%%%%%%%%%%%%%%%%%%%%%%%%%%%%%%%%%%%%%%%%%%%%%%%%%%%%%%%%

%%%%%%%%%%%%%%%%%%%%%%%%%%%%%%%%%%%%%%%%%%%%%%%%%%%%%%%%%%%%%%%%%%%%%%%%%%%%%%%%
\begin{frame}[fragile]
  \frametitle{Using C++11 Mutexes: Example}

\begin{changemargin}{1.5cm}
  \begin{lstlisting}
// code
m1.lock()    
// protected code
m1.unlock()
// more code
  \end{lstlisting}
  \vfill
  \begin{itemize}
    \item Everything within the {\tt lock} and {\tt unlock} is protected.
    \item Be careful to avoid deadlocks if you are using multiple mutexes.
    \item Also you can use {\tt mutex::trylock()}, if needed.
  \end{itemize}
\end{changemargin}
\end{frame}
%%%%%%%%%%%%%%%%%%%%%%%%%%%%%%%%%%%%%%%%%%%%%%%%%%%%%%%%%%%%%%%%%%%%%%%%%%%%%%%%

%%%%%%%%%%%%%%%%%%%%%%%%%%%%%%%%%%%%%%%%%%%%%%%%%%%%%%%%%%%%%%%%%%%%%%%%%%%%%%%%
\begin{frame}[fragile]
  \frametitle{Data Race Example}

\begin{changemargin}{1.5cm}
  Recall that \structure{dataraces} occur when two concurrent actions access the same
  variable and at least one of them is a {\bf write}


  \begin{lstlisting}
...
static int counter = 0;

void* run(void* arg) {
    for (int i = 0; i < 100; ++i) {
        ++counter;
    }
}

int main(int argc, char *argv[])
{
    // Create 8 threads
    // Join 8 threads
    printf("counter = %i\n", counter);
}
  \end{lstlisting}

Is there a datarace in this example? If so, how would we fix it?
\end{changemargin}

\end{frame}
%%%%%%%%%%%%%%%%%%%%%%%%%%%%%%%%%%%%%%%%%%%%%%%%%%%%%%%%%%%%%%%%%%%%%%%%%%%%%%%%

%%%%%%%%%%%%%%%%%%%%%%%%%%%%%%%%%%%%%%%%%%%%%%%%%%%%%%%%%%%%%%%%%%%%%%%%%%%%%%%%
\begin{frame}[fragile]
  \frametitle{Example Problem Solution}
\begin{changemargin}{1.5cm}
  \begin{lstlisting}[escapechar=!]
...
!\structure{static pthread\_mutex\_t mutex = PTHREAD\_MUTEX\_INITIALIZER;}!
static int counter = 0;

void* run(void* arg) {
    for (int i = 0; i < 100; ++i) {
        !\structure{pthread\_mutex\_lock(\&mutex);}!
        ++counter;
        !\structure{pthread\_mutex\_unlock(\&mutex);}!
    }
}

int main(int argc, char *argv[])
{
    // Create 8 threads
    // Join 8 threads
    !\structure{pthread\_mutex\_destroy(\&mutex);}!
    printf("counter = %i\n", counter);
}
  \end{lstlisting}
\end{changemargin}

\end{frame}
%%%%%%%%%%%%%%%%%%%%%%%%%%%%%%%%%%%%%%%%%%%%%%%%%%%%%%%%%%%%%%%%%%%%%%%%%%%%%%%%
\end{document}
