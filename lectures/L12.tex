\documentclass[11pt]{article}
\usepackage{listings}
\usepackage{tikz}
\usepackage{url}
\usepackage{hyperref}
%\usepackage{algorithm2e}
\usetikzlibrary{arrows,automata,shapes,positioning}
\tikzstyle{block} = [rectangle, draw, fill=blue!20, 
    text width=2.5em, text centered, rounded corners, minimum height=2em]
\tikzstyle{bw} = [rectangle, draw, fill=blue!20, 
    text width=3.5em, text centered, rounded corners, minimum height=2em]

\newcommand{\handout}[5]{
  \noindent
  \begin{center}
  \framebox{
    \vbox{
      \hbox to 5.78in { {\bf ECE459: Programming for Performance } \hfill #2 }
      \vspace{4mm}
      \hbox to 5.78in { {\Large \hfill #5  \hfill} }
      \vspace{2mm}
      \hbox to 5.78in { {\em #3 \hfill #4} }
    }
  }
  \end{center}
  \vspace*{4mm}
}

\lstset{basicstyle=\ttfamily \scriptsize}

\newcommand{\lecture}[4]{\handout{#1}{#2}{#3}{#4}{Lecture #1}}
\topmargin 0pt
\advance \topmargin by -\headheight
\advance \topmargin by -\headsep
\textheight 8.9in
\oddsidemargin 0pt
\evensidemargin \oddsidemargin
\marginparwidth 0.5in
\textwidth 6.5in

\parindent 0in
\parskip 1.5ex
%\renewcommand{\baselinestretch}{1.25}

% http://gurmeet.net/2008/09/20/latex-tips-n-tricks-for-conference-papers/
\newcommand{\squishlist}{
 \begin{list}{$\bullet$}
  { \setlength{\itemsep}{0pt}
     \setlength{\parsep}{3pt}
     \setlength{\topsep}{3pt}
     \setlength{\partopsep}{0pt}
     \setlength{\leftmargin}{1.5em}
     \setlength{\labelwidth}{1em}
     \setlength{\labelsep}{0.5em} } }
\newcommand{\squishlisttwo}{
 \begin{list}{$\bullet$}
  { \setlength{\itemsep}{0pt}
     \setlength{\parsep}{0pt}
    \setlength{\topsep}{0pt}
    \setlength{\partopsep}{0pt}
    \setlength{\leftmargin}{2em}
    \setlength{\labelwidth}{1.5em}
    \setlength{\labelsep}{0.5em} } }
\newcommand{\squishend}{
\end{list}  }
\newtheorem{example}{Example}


\begin{document}

\lecture{12 --- January 30, 2015}{Winter 2015}{Patrick Lam}{version 2}

In class, I live-coded a parallelization
of the Mandelbrot code. I had to first refactor the code, creating
an array to hold the output. Then I added a struct to pass the offset and
stride to the thread. Finally, I invoked pthread to create and join threads.

\section*{Breaking Dependencies with Speculation}
Recall that computer architects often use speculation to predict branch
targets: the direction of the branch depends on the condition codes
when executing the branch code. To get around having to wait, the processor
speculatively executes one of the branch targets, and cleans up if it
has to.

We can also use speculation at a coarser-grained level and
speculatively parallelize code. We discuss two ways of doing so: one
which we'll call speculative execution, the other value speculation.

\subsection*{Speculative Execution for Threads.} The idea here is to
start up a thread to compute a result that you may or may not need.
Consider the following code:

{\small \begin{verbatim}
  void doWork(int x, int y) {
    int value = longCalculation(x, y);
    if (value > threshold) {
      return value + secondLongCalculation(x, y);
    }
    else {
      return value;
    }
  }
\end{verbatim} }
Without more information, you don't know whether you'll have to execute
{\tt secondLongCalculation} or not; it depends on the return value of
{\tt longCalculation}. 

Fortunately, the arguments to {\tt secondLongCalculation} do not
depend on {\tt longCalculation}, so we can call it at any point. 
Here's one way to speculatively thread the work:

{\small \begin{verbatim}
  void doWork(int x, int y) {
    thread_t t1, t2;
    point p(x,y);
    int v1, v2;
    thread_create(&t1, NULL, &longCalculation, &p);
    thread_create(&t2, NULL, &secondLongCalculation, &p);
    thread_join(t1, &v1);
    thread_join(t2, &v2);
    if (v1 > threshold) {
      return v1 + v2;
    } else {
      return v1;
    }
  }
\end{verbatim} }
We now execute both of the calculations in parallel and return the same
result as before.

{\sf Intuitively: when is this code faster? When is it slower? How
  could you improve the use of threads?}\\[4em]

We can model the above code by estimating the probability $p$ that
the second calculation needs to run, the time $T_1$ that it takes
to run {\tt longCalculation}, the time $T_2$ that it takes to run
{\tt secondLongCalculation}, and synchronization overhead $S$.
Then the original code takes time
\[ T = T_1 + p T_2, \]
while the speculative code takes time
\[ T_s = \max(T_1, T_2) + S.\]

\paragraph{Exercise.} Symbolically compute when it's profitable to do the
speculation as shown above. There are two cases: $T_1 > T_2$ and $T_1
< T_2$. (You can ignore $T_1 = T_2$.)

\end{document}
