\documentclass[aspectratio=43,11pt]{beamer}

\usetheme{Boadilla}
\setbeamertemplate{footline}[page number]{}
\setbeamertemplate{navigation symbols}{}

\usepackage{times}
\usepackage{tikz}
\usefonttheme{structurebold}
\usepackage{listings}
\usepackage{hyperref}
\usepackage{url}

\definecolor{commgreen}{rgb}{0,0.6,0}
% Suppress the navigation bar
\beamertemplatenavigationsymbolsempty

\lstset{language=C++,
        basicstyle=\ttfamily\tiny,
        keywordstyle=\color{blue},
        stringstyle=\color{red},
        commentstyle=\color{green},
        morecomment=[l][\color{magenta}]{\#},
        breaklines=true
}

\newenvironment{changemargin}[1]{% 
  \begin{list}{}{% 
    \setlength{\topsep}{0pt}% 
    \setlength{\leftmargin}{#1}% 
    \setlength{\rightmargin}{1em}
    \setlength{\listparindent}{\parindent}% 
    \setlength{\itemindent}{\parindent}% 
    \setlength{\parsep}{\parskip}% 
  }% 
  \item[]}{\end{list}} 

\title{Tutorial 01 -- POSIX Threads}
\subtitle{ECE 459: Programming for Performance}
\author{Saeed Nejati \\ \href{mailto:snejati@uwaterloo.ca}{snejati@uwaterloo.ca}}
\institute{University of Waterloo}
\date{January 23, 2015}

\begin{document}

%%%%%%%%%%%%%%%%%%%%%%%%%%%%%%%%%%%%%%%%%%%%%%%%%%%%%%%%%%%%%%%%%%%%%%%%%%%%%%%%
\begin{frame}[plain]
  \titlepage
\end{frame}

%%%%%%%%%%%%%%%%%%%%%%%%%%%%%%%%%%%%%%%%%%%%%%%%%%%%%%%%%%%%%%%%%%%%%%%%%%%%%%%%
\begin{frame}[fragile]
  \frametitle{Quick Reference}
  
  \begin{itemize}
    \item \textbf{API header:} {\tt \#include <pthread.h>}
    \item \textbf{Compiling:} {\tt gcc -pthread source.c -o exec}
    \item \textbf{Main functions:}
    \begin{itemize}
      \item \textbf{Creating threads:} {\tt pthread\_create}
      \item \textbf{Waiting for another thread to finish:} {\tt pthread\_join}
      \item \textbf{Exiting a thread:} {\tt pthread\_exit}
      \item \textbf{Initializing a Mutex: } {\tt pthread\_mutex\_init}
      \item \textbf{Locking a Mutex:} {\tt pthread\_mutex\_lock}
      \item \textbf{Unlocking a Mutex:} {\tt pthread\_mutex\_unlock}
    \end{itemize}
  \end{itemize}
  \begin{lstlisting}
  int pthread_create(pthread_t *thread, const pthread_attr_t *attr,
                     void *(*start_routine) (void *), void *arg);
  int pthread_join(pthread_t thread, void **retval);
  void pthread_exit(void *retval);
  int pthread_mutex_init(pthread_mutex_t *mutex,
                         const pthread_mutexattr_t *mutexattr);
  int pthread_mutex_lock(pthread_mutex_t *mutex);
  int pthread_mutex_unlock(pthread_mutex_t *mutex);
  \end{lstlisting}
\end{frame}
%%%%%%%%%%%%%%%%%%%%%%%%%%%%%%%%%%%%%%%%%%%%%%%%%%%%%%%%%%%%%%%%%%%%%%%%%%%%%%%%
\section{An example use of Pthread}
%%%%%%%%%%%%%%%%%%%%%%%%%%%%%%%%%%%%%%%%%%%%%%%%%%%%%%%%%%%%%%%%%%%%%%%%%%%%%%%%
\begin{frame}
  \sectionpage
\end{frame}
%%%%%%%%%%%%%%%%%%%%%%%%%%%%%%%%%%%%%%%%%%%%%%%%%%%%%%%%%%%%%%%%%%%%%%%%%%%%%%%%
\begin{frame}
  \frametitle{An example: Computing Sum of $\phi(n)$}
  
  \begin{itemize}
    \item \textbf{Task:} Compute the sum of $\phi$ of numbers in a given range [$l$, $h$]
    \item $\phi$ is the euler's totient function
    \item We use a naive implementation for $\phi$ and don't care about optimizing the function itself
  \end{itemize}
\end{frame}
%%%%%%%%%%%%%%%%%%%%%%%%%%%%%%%%%%%%%%%%%%%%%%%%%%%%%%%%%%%%%%%%%%%%%%%%%%%%%%%%
\begin{frame}
  \frametitle{Sum of $\phi(n)$, version 1}
  
  \begin{itemize}
    \item This is a simple serial implementation of the task.
    \item We considered the range to be [2, 30000]
  \end{itemize}
  
  \begin{changemargin}{0.5cm}
  \lstinputlisting{codes/serial.cpp}
  \end{changemargin}
  \let\thefootnote\relax\footnotetext{Complete code can be found in {\it codes/serial.cpp}}
\end{frame}
%%%%%%%%%%%%%%%%%%%%%%%%%%%%%%%%%%%%%%%%%%%%%%%%%%%%%%%%%%%%%%%%%%%%%%%%%%%%%%%%
%\begin{frame}
%\lstinputlisting[firstline=22, lastline=54]{prime_sum/para1.cpp}
%\end{frame}
%%%%%%%%%%%%%%%%%%%%%%%%%%%%%%%%%%%%%%%%%%%%%%%%%%%%%%%%%%%%%%%%%%%%%%%%%%%%%%%%
\begin{frame}
  \frametitle{Sum of $\phi(n)$, version 2}

  \begin{enumerate}
    \item Calculation of $\phi(n)$ for each $n$ could be done independent of each other (look into {\tt main})
    \item Thus we can do it in parallel
    \item We consider that we have a fixed number of threads (8).
    \item Now we need to split the task and send it to threads
  \end{enumerate}  
\end{frame}
%%%%%%%%%%%%%%%%%%%%%%%%%%%%%%%%%%%%%%%%%%%%%%%%%%%%%%%%%%%%%%%%%%%%%%%%%%%%%%%%
\begin{frame}
  \frametitle{Sum of $\phi(n)$, version 2}
  
  \begin{itemize}
    \item Let's split the interval into equal sized sub-intervals
    \item Thread routines accept one argument, so we represent a sub-interval as a
    pair object: [first, second]
  \end{itemize}
  
  \lstinputlisting[firstline=32, lastline=56]{codes/parallel1.cpp}
\end{frame}
%%%%%%%%%%%%%%%%%%%%%%%%%%%%%%%%%%%%%%%%%%%%%%%%%%%%%%%%%%%%%%%%%%%%%%%%%%%%%%%%
\begin{frame}
  \frametitle{Sum of $\phi(n)$, version 2}
  
  \begin{itemize}
    \item We should sum up the results and because all threads are going to update this
    sum variable, we need to protect it as well
    \item Consider these two variables are in a scope that are accessible by all threads
    \lstinputlisting[firstline=6, lastline=7]{codes/parallel1.cpp}
    \item and before starting threads, we initialize these two like this:
    \lstinputlisting[firstline=58, lastline=59]{codes/parallel1.cpp}
    \item Now we can fire up our threads and wait for all of them to complete their tasks
    \lstinputlisting[firstline=61, lastline=66]{codes/parallel1.cpp}
  \end{itemize}
\end{frame}
%%%%%%%%%%%%%%%%%%%%%%%%%%%%%%%%%%%%%%%%%%%%%%%%%%%%%%%%%%%%%%%%%%%%%%%%%%%%%%%%
\begin{frame}
  \frametitle{Sum of $\phi(n)$, version 2}
  
  \begin{itemize}
    \item At last, our {\tt threadRoutine} would be something like this:
    \lstinputlisting[firstline=20, lastline=30]{codes/parallel1.cpp}
    \item \textbf{Note:} we cast the {\tt void*} pointer to argument to the appropriate
    type and then reference it to get the value of the job we need to work on
  \end{itemize}
  
  \let\thefootnote\relax\footnotetext{Complete code can be found in {\it codes/parallel1.cpp}}
\end{frame}
%%%%%%%%%%%%%%%%%%%%%%%%%%%%%%%%%%%%%%%%%%%%%%%%%%%%%%%%%%%%%%%%%%%%%%%%%%%%%%%%
\begin{frame}
  \frametitle{Quick Reminder}
  
  UNIX {\tt time} command: {\tt time [options] command [arguments...]}
  \begin{itemize}
    \item \textbf{real:} The elapsed real time between invocation and termination of process
    \item \textbf{user:} The user CPU time
    \item \textbf{sys:} The system CPU time (the time spent in system calls on behalf of process)
  \end{itemize}
\end{frame}
%%%%%%%%%%%%%%%%%%%%%%%%%%%%%%%%%%%%%%%%%%%%%%%%%%%%%%%%%%%%%%%%%%%%%%%%%%%%%%%%
\begin{frame}
  \frametitle{Execution Results}

  The results of running these two versions on a machine with
  Intel Core i7 CPU, 16M ram was:
  \begin{columns}
  \column{.5\textwidth}
  \begin{itemize}
    \item Version 1: \\
    {\tt real 0m38.191s \\
         user 0m38.180s \\
         sys  0m0.002s}  
  \end{itemize}
  \column{.5\textwidth}
  \begin{itemize}
    \item Version 2: \\
    {\tt real 0m38.214s \\
         user 0m38.708s \\
         sys  0m0.075s}  
  \end{itemize}
  \end{columns}
  \vfill
  But why the parallel version didn't make any improvements?  
\end{frame}
%%%%%%%%%%%%%%%%%%%%%%%%%%%%%%%%%%%%%%%%%%%%%%%%%%%%%%%%%%%%%%%%%%%%%%%%%%%%%%%%
\begin{frame}
  \frametitle{Sum of $\phi(n)$, version 2, look-back}

  \begin{itemize}
    \item<1-> Take a look at this part of code (in {\tt threadRoutine} function):
    \lstinputlisting[firstline=24, lastline=28]{codes/parallel1.cpp}
    
    \item<2-> We are protecting the {\tt sum} variable, which is a correct thing to do
    \item<3-> BUT, we are locking the main heavy computation part in the critical section
    \item<4-> No two threads go into Critical sections at the same time
    \item<5-> Therefore Critical sections are executed in a serial fashion
    \item<6-> Our parallel version is no better than a serial code + thread creation and mutex handling overhead!
    (look at the difference in {\tt sys} time again)
  \end{itemize}
\end{frame}
%%%%%%%%%%%%%%%%%%%%%%%%%%%%%%%%%%%%%%%%%%%%%%%%%%%%%%%%%%%%%%%%%%%%%%%%%%%%%%%%
\begin{frame}
  \frametitle{Sum of $\phi(n)$, version 2.1}
  
  \begin{itemize}
    \item \textbf{TIP:} Do as much as possible with your thread local data, then publish your results
    to global scope
    \item A fix to previous version could be:
    \lstinputlisting[firstline=20, lastline=33]{codes/parallel2.cpp}
  \end{itemize}
  \let\thefootnote\relax\footnotetext{The updated code can be found in {\it codes/parallel2.cpp}}
\end{frame}
%%%%%%%%%%%%%%%%%%%%%%%%%%%%%%%%%%%%%%%%%%%%%%%%%%%%%%%%%%%%%%%%%%%%%%%%%%%%%%%%
\begin{frame}
  \frametitle{Execution Results, again}

  \begin{columns}
  \column{.3\textwidth}
  \begin{itemize}
    \item Version 1: \\
    \fontsize{8pt}{7.2}\selectfont{\tt real 0m38.191s \\
         user 0m38.180s \\
         sys  0m0.002s}  
  \end{itemize}
  \column{.3\textwidth}
  \begin{itemize}
    \item Version 2: \\
    \fontsize{8pt}{7.2}\selectfont{\tt real 0m38.214s \\
         user 0m38.708s \\
         sys  0m0.075s}  
  \end{itemize}
  \column{.3\textwidth}
  \begin{itemize}
    \item Version 2.1: \\
    \fontsize{8pt}{7.2}\selectfont{\tt real 0m9.975s \\
           user 0m41.427s \\
           sys  0m0.007s}
  \end{itemize}
  \end{columns}
  \vfill
  \begin{itemize}
    \item<1-> The results seems much better, but the current speed up is about 3.8X (instead of close to ideal 8X).
    \item<2-> The threads are not doing the same amount of job
    \item<3-> Computing $\phi(n)$ takes more time as $n$ increases
    \item<4-> The 8th thread is doing more computation than the others
    \item<4-> It means that some threads are doing more than 1/8th of the total jobs
    \item<5-> How to balance the job splitting?
    \begin{itemize}
      \item Dependent on task
      \item Very hard to determine
    \end{itemize}
    \item<6-> \textbf{An Option:} Break the jobs into smaller piece and when the threads finished
    their job, let them pick another one (known as Dynamic Scheduling, we won't discuss it here).
  \end{itemize}
\end{frame}
%%%%%%%%%%%%%%%%%%%%%%%%%%%%%%%%%%%%%%%%%%%%%%%%%%%%%%%%%%%%%%%%%%%%%%%%%%%%%%%%
\begin{frame}
  \frametitle{Some performance notes}
  
  \begin{itemize}
    \item \textbf{Lock granularity:} How "big" (coarse) or "small" (fine) are your mutexes?
    \begin{itemize}
      \item lock your whole structure or fields of a structure?
      \item The more fine-grained, the more concurrency
      \item but at the cost of more overhead and potential deadlocks
    \end{itemize}
    \item \textbf{Lock ordering:} Make sure your locks are always locked in an agreed order
    \item \textbf{Lock frequency:} Are you locking too often? Reduce such occurrences to
    exploit concurrency and reduce sync overhead
    \item \textbf{Critical sections:} Take extra steps to minimize critical sections which
    can be potentially large bottlenecks
  \end{itemize}
  
\end{frame}
%%%%%%%%%%%%%%%%%%%%%%%%%%%%%%%%%%%%%%%%%%%%%%%%%%%%%%%%%%%%%%%%%%%%%%%%%%%%%%%%
\section{Helgrind, A thread error detector}
%%%%%%%%%%%%%%%%%%%%%%%%%%%%%%%%%%%%%%%%%%%%%%%%%%%%%%%%%%%%%%%%%%%%%%%%%%%%%%%%
\begin{frame}
  \sectionpage
\end{frame}
%%%%%%%%%%%%%%%%%%%%%%%%%%%%%%%%%%%%%%%%%%%%%%%%%%%%%%%%%%%%%%%%%%%%%%%%%%%%%%%%
\begin{frame}
  \frametitle{How to use it?}

  \begin{itemize}
    \item It is part of Valgrind tool set
    \item You can invoke it like: {\tt valgrind --tool=helgrind your\_program [your\_program\_options]}
    \item It is a tool for detecting synchronization errors in C, C++ and fortran (which use pthread)
    \item It can detect:
    \begin{itemize}
      \item Misuses of pthread API
      \item Potential deadlocks arising from lock ordering problems
      \item Data races -- Accessing memory without adequate locking or synchronization
    \end{itemize}
  \end{itemize}

\let\thefootnote\relax\footnotetext{For more information and complete reference, please refer to: \\ \url{http://valgrind.org/docs/manual/hg-manual.html}}
\end{frame}
%%%%%%%%%%%%%%%%%%%%%%%%%%%%%%%%%%%%%%%%%%%%%%%%%%%%%%%%%%%%%%%%%%%%%%%%%%%%%%%%
\begin{frame}
  \frametitle{A sample use}
  
  \begin{itemize}
    \item Consider the following code, which has a data race on variable {\tt var}
    \lstinputlisting{codes/race.cpp}
    \item If you compile the code with debugging info, helgrind would be able to address line of code
    with an error in it. \\
    {\tt g++ -pthread -g race.cpp}
  \end{itemize}
  
  \let\thefootnote\relax\footnotetext{The code is an example from Helgrind reference document and
  can be found at {\it codes/race.cpp} too}
\end{frame}
%%%%%%%%%%%%%%%%%%%%%%%%%%%%%%%%%%%%%%%%%%%%%%%%%%%%%%%%%%%%%%%%%%%%%%%%%%%%%%%%
\begin{frame}[fragile]
  \frametitle{A sample use}
  
  \begin{itemize}
    \item Now, this is the result of running the code with Helgrind: {\tt valgrind --tool=helgrind ./a.out}
    \item Helgrind reports two possible data races: one on read
  \end{itemize}

\fontsize{6pt}{7.2}\selectfont\begin{verbatim}
Possible data race during read of size 4 at 0x601050 by thread #1
Locks held: none
   at 0x4007C1: main (race.cpp:13)

This conflicts with a previous write of size 4 by thread #2
Locks held: none
   at 0x400791: child_fn(void*) (race.cpp:6)
   by 0x4A0A245: ??? (in /usr/lib64/valgrind/vgpreload_helgrind-amd64-linux.so)
   by 0x3147407C64: start_thread (in /usr/lib64/libpthread-2.17.so)
   by 0x3146CF5B9C: clone (in /usr/lib64/libc-2.17.so)
\end{verbatim}    
\end{frame}
%%%%%%%%%%%%%%%%%%%%%%%%%%%%%%%%%%%%%%%%%%%%%%%%%%%%%%%%%%%%%%%%%%%%%%%%%%%%%%%%
\begin{frame}[fragile]
  \frametitle{A sample use}
  
  \begin{itemize}
    \item and one on write
    \item It refers to line 13 and line 6 of the code
  \end{itemize}

\fontsize{6pt}{7.2}\selectfont\begin{verbatim}
Possible data race during write of size 4 at 0x601050 by thread #1
Locks held: none
   at 0x4007CA: main (race.cpp:13)

This conflicts with a previous write of size 4 by thread #2
Locks held: none
   at 0x400791: child_fn(void*) (race.cpp:6)
   by 0x4A0A245: ??? (in /usr/lib64/valgrind/vgpreload_helgrind-amd64-linux.so)
   by 0x3147407C64: start_thread (in /usr/lib64/libpthread-2.17.so)
   by 0x3146CF5B9C: clone (in /usr/lib64/libc-2.17.so)
\end{verbatim}    
\end{frame}
%%%%%%%%%%%%%%%%%%%%%%%%%%%%%%%%%%%%%%%%%%%%%%%%%%%%%%%%%%%%%%%%%%%%%%%%%%%%%%%%
\end{document}
